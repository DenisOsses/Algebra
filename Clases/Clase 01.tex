\documentclass[handout,aspectratio=169]{beamer}
\usepackage{graphicx}
\usepackage{ragged2e}
\usefonttheme{professionalfonts}
\usepackage[spanish]{babel}
\usepackage{advdate}
\usepackage{multicol}
\setbeamercovered{transparent}

% Document metadata
\title{Clase 1}
\subtitle{Proposiciones y conectivos lógicos}
\author{Profesor: Denis Osses}
% \date{\AdvanceDate[+1]\today}
\date{7 de marzo de 2025}

% Image for the title page (use includegraphics option to properly size/place it)
\titlegraphic{\includegraphics[height=\paperheight]{imagen5}}

\usetheme[sectionstyle=style2]{trigon}

% Define logos to use (comment if no logo)
\biglogo{logoFIC} % Used on titlepage only
%\smalllogo{gggg}% Used on top right corner of regular frames

% ------ If you want to change the theme default colors, do it here ------
%\definecolor{tPrim}{HTML}{00843B}   % Green
%\definecolor{tSec}{HTML}{289B38}    % Green light
%\definecolor{tAccent}{HTML}{F07F3C} % Orange


% ------ Packages and definitions used for this demo. Can be removed ------
\usepackage{appendixnumberbeamer} % To use \appendix command
\pdfstringdefDisableCommands{% Fix hyperref translate warning with \appendix
\def\translate#1{#1}%
}
\usepackage{pgf-pie} % For pie charts
\usepackage{caption} % For subfigures
\usepackage{subcaption} % For subfigures
\usepackage{xspace}
\newcommand{\themename}{\textbf{\textsc{Álgebra}}\xspace}
\usepackage[scale=2]{ccicons} % Icons for CC-BY-SA
\usepackage{booktabs} % Better tables


%==============================================================================
%                               BEGIN DOCUMENT
%==============================================================================
\begin{document}

%--------------------------------------
% Create title frame
\titleframe

%--------------------------------------
% Table of contents
\begin{frame}{Temario}
  \setbeamertemplate{section in toc}[sections numbered]
  \tableofcontents%[hideallsubsections]
\end{frame}

%==============================================
\section{Objetivos de hoy}
%==============================================
%\subsection{Charts}
\begin{frame}{\insertsectionhead}
  \framesubtitle{\insertsubsectionhead}
  
\justifying

\begin{itemize}[<+->]
    \item Introducir los conceptos básicos de la lógica proposicional.
    \item Determinar el valor de verdad de proposiciones.
    % \item Demostrar equivalencias lógicas de proposiciones.
\end{itemize}

\end{frame}

%==============================================
\section{Contenidos}
%==============================================
\subsection{Lógica proposicional}
\begin{frame}
  \frametitle{\insertsectionhead}
  \framesubtitle{\insertsubsectionhead}
  
  \justifying
\footnotesize
  
\begin{minipage}[t]{0.5\linewidth}
La lógica es la ciencia formal que estudia los principios de la demostración y nos ayuda a discernir entre \textbf{proposiciones} válidas mediante inferencias o deducciones. \pause
  
  \begin{block}{Definición}
  \justifying
    Las \textbf{proposiciones} son enunciados o frases del lenguaje natural sobre las cuales es posible afirmar su veracidad $(V)$ o falsedad $(F)$. Usualmente se denotan con letras minúsculas $p,q,r,s,t,\ldots$
  \end{block}\pause
  
    \begin{exampleblock}{Ejemplo 1}
    Determine si los siguientes enunciados son proposiciones o no: \pause
    \begin{enumerate}
        \item Dos es mayor que cinco.
        \item Hola.
        \vspace{-0.1cm}
    \end{enumerate}
    \end{exampleblock}
\end{minipage}\pause
\hspace{0.5cm}\begin{minipage}[t]{0.5\linewidth}
\vspace{-1.5cm}
\begin{block}{Definición}
  \justifying
    La veracidad o falsedad de una proposición se denomina \textbf{valor de verdad} de tal proposición.
  \end{block}\pause

  \vspace{-0.2cm}
  \begin{exampleblock}{Ejemplo 2}
    Determine el valor de verdad de las siguentes proposiciones:\pause
    \begin{enumerate}
        \item $2+3=5$.
        \item Hoy es 6 de marzo.
        \vspace{-0.1cm}
    \end{enumerate}
    \end{exampleblock}\pause

    \vspace{-0.2cm}
    \begin{block}{Nota}
    \justifying
    La negación de una proposición asocia a toda proposición $p$ una proposición denotada $\neg p$, verdadera si $p$ es falsa y falsa si $p$ es verdadera. 
  \end{block}\pause

  \vspace{-0.2cm}
  \begin{exampleblock}{Ejemplo 3}
    Escriba la negación de las proposiciones del ejemplo 2.
    \end{exampleblock}
\end{minipage}
  
    
  
\end{frame}

%-------------------------------------------------------------------------

\subsection{Tablas de Verdad}
\begin{frame}
  \frametitle{\insertsectionhead}
  \framesubtitle{\insertsubsectionhead}
  
  \justifying
\footnotesize
  
\begin{minipage}[t]{0.5\linewidth}
  
    \textbf{Pregunta}: Suponga que conoce los valores de verdad de las proposiciones $p$ y $q$. ?`Es posible determinar el valor de verdad de sus proposiciones compuestas asociadas? \pause Para el caso de dos proposiciones $p$ y $q$ existen cuatro formas de combinar sus valores de verdad, los que se han colocado en la siguiente tabla (tabla de verdad): \pause 

    $$\begin{tabular}{|c|c|c|}
    \hline
    $p$& $q$& $p\ast q$ \\
     \hline
     $V$& $V$& $?$\\
     $V$& $F$& $?$\\
     $F$& $V$& $?$\\
     $F$& $F$& $?$\\
     \hline
    \end{tabular}$$
\end{minipage}\pause
\hspace{0.5cm}\begin{minipage}[t]{0.5\linewidth}
\justifying
    Al fijar una sucesión de los cuatro valores $(????)$ correspondientes a $p\ast q$, se obtiene exactamente el significado lógico del \textbf{conector} $\ast$. \pause Cada sucesión de valores se llama una \textbf{evaluación} de $\ast$, así se tiene que: toda evaluación determina un conectivo lógico. \pause 
    
    \vspace{0.2cm}
  
   De esta manera para dos proposiciones $p$ y $q$ existen $2^4 = 16$ conectivos lógicos, los cuales quedarán definidos mediante su evaluación. Algunos de ellos son los más utilizados en el lenguaje natural y en el matemático, y los llamaremos \textbf{conectivos lógicos} usuales.
\end{minipage}

\end{frame}

%-------------------------------------------------------------------------

\subsection{Conectivos Lógicos}
\begin{frame}
\frametitle{\insertsectionhead}
\framesubtitle{\insertsubsectionhead}
  
  \justifying
\footnotesize
  
\begin{minipage}[t]{0.5\linewidth}

    \begin{block}{Definición}
    La \textbf{conjunción} es el conectivo lógico \textbf{y}, notado con el símbolo $\wedge$ y definido por:\pause 
    $$\begin{tabular}{|c|c|c|}
    \hline
     $p$& $q$& $p\wedge q$ \\
     \hline
     $V$& $V$& $V$\\
     $V$& $F$& $F$\\
     $F$& $V$& $F$\\
     $F$& $F$& $F$\\
     \hline
\end{tabular}$$
  \end{block}\pause

Se lee $``p$ y $q"$.
\end{minipage}\pause
\hspace{0.5cm}\begin{minipage}[t]{0.5\linewidth}
\justifying
 \begin{block}{Definición}
   La \textbf{disyunción} es el conectivo lógico \textbf{o}, notado con el símbolo $\vee$ y definido por:\pause 
   $$\begin{tabular}{|c|c|c|}
    \hline
     $p$& $q$& $p\vee q$ \\
     \hline
     $V$& $V$& $V$\\
     $V$& $F$& $V$\\
     $F$& $V$& $V$\\
     $F$& $F$& $F$\\
     \hline
\end{tabular}$$
  \end{block}\pause 

Se lee $``p$ o $q"$.
\end{minipage}

\end{frame}

%-------------------------------------------------------------------------

%\subsection{Conectivos Lógicos}
\begin{frame}
\frametitle{\insertsectionhead}
\framesubtitle{\insertsubsectionhead}
  
  \justifying
\footnotesize
  
\begin{minipage}[t]{0.5\linewidth}

    \begin{block}{Definición}
    \justifying
   La \textbf{implicación} es el conectivo lógico \textbf{si...entonces}, notado con el símbolo $\Rightarrow$ y definido por:\pause
$$\begin{tabular}{|c|c|c|}
    \hline
     $p$& $q$& $p\Rightarrow q$ \\
     \hline
     $V$& $V$& $V$\\
     $V$& $F$& $F$\\
     $F$& $V$& $V$\\
     $F$& $F$& $V$\\
     \hline
\end{tabular}$$
  \end{block}\pause

Se lee $``$si $p$ entonces $q"$ o ``$p$ implica $q"$.
\end{minipage}\pause
\hspace{0.5cm}\begin{minipage}[t]{0.5\linewidth}
\justifying
\begin{block}{Definición}
    \justifying
    La \textbf{doble implicación} es el conectivo lógico \textbf{si y solo si}, notado con el símbolo $\Leftrightarrow$ y definido por:\pause
$$\begin{tabular}{|c|c|c|}
    \hline
     $p$& $q$& $p\Leftrightarrow q$ \\
     \hline
     $V$& $V$& $V$\\
     $V$& $F$& $F$\\
     $F$& $V$& $F$\\
     $F$& $F$& $V$\\
     \hline
\end{tabular}$$
  \end{block}\pause

Se lee $``p$ si y solo si $q"$ o $``p$ es equivalente a $q"$.
\end{minipage}

\end{frame}

%-------------------------------------------------------------------------

%\subsection{Conectivos Lógicos}
\begin{frame}
\frametitle{\insertsectionhead}
\framesubtitle{\insertsubsectionhead}
  
  \justifying
\footnotesize
  
\begin{minipage}[t]{0.5\linewidth}
  
    \begin{exampleblock}{Ejemplo 4}
    \justifying
    Determinar el valor de verdad de cada una de las siguientes proposiciones:
    \begin{enumerate}
        \item Si $3<5$ entonces $-3<-5$.
        \item $\sqrt{16}=4$ o $\sqrt{16}=-4$.
        \item $6+4=10$ y $\sqrt{2}\cdot\sqrt{2}=2$.
        \item $5^2=25~$ o $~2+2=5$.
    \end{enumerate}
    \end{exampleblock}  \pause

    \begin{exampleblock}{Ejemplo 5}
    \justifying
    Pruebe que $(p\Rightarrow q)\Leftrightarrow(\neg p\vee q)$ es una proposición siempre verdadera.
    \end{exampleblock}

\end{minipage} \pause
\hspace{0.5cm}\begin{minipage}[t]{0.5\linewidth}
    \begin{alertblock}{\color{red}{Ejercicio propuesto}}
    \justifying
    Pruebe que $(p\Rightarrow q)\Leftrightarrow(\neg q\Rightarrow\neg p)$ es una proposición siempre verdadera.
    \end{alertblock}
\end{minipage}

\end{frame}

% %-------------------------------------------------------------------------
% \subsection{Equivalencias lógicas}
% \begin{frame}
%   \frametitle{\insertsectionhead}
%   \framesubtitle{\insertsubsectionhead}
  
%   \justifying
  
%     \begin{block}{Definición}\pause
%     Dos proposiciones compuestas $m$ y $n$ son \textbf{equivalentes} si entregan el mismo valor de verdad para todo valor de verdad de $m$ y $n$. Escribimos $m\equiv n$ para indicar esto. 
%     \end{block}  \pause
    
%     De este modo, del ejemplo 5 tenemos que $p\Rightarrow q\equiv \neg q\Rightarrow\neg p.$
    
%     \begin{exampleblock}{Ejemplo 6}\pause
%     Pruebe que $p\Rightarrow q\equiv \neg p\vee q$.
%     \end{exampleblock}

% \end{frame}

% %-------------------------------------------------------------------------

% \begin{frame}
%   \frametitle{\insertsectionhead}
%   \framesubtitle{\insertsubsectionhead}
  
%   \justifying
%     \begin{block}{Definición}\pause
%     Algunas equivalencias lógicas usuales son\pause
%     \begin{enumerate}[<+->]
%     \item $\neg(\neg p)\equiv p$.
%     \item $\neg(p\vee q)\equiv \neg p\wedge \neg q$.
%     \item $\neg(p\wedge q)\equiv \neg p\vee \neg q$.
%     \item $p\Rightarrow q\equiv \neg p\vee q$.
%     \item $p\Rightarrow q\equiv \neg q\Rightarrow \neg p$.
%     \item $p\Leftrightarrow q\equiv (p\Rightarrow q)\wedge(q\Rightarrow p)$.
%     \end{enumerate}
%     \end{block}\pause 
    
%     \begin{exampleblock}{Ejemplo 7}\pause
%     ?`Es cierto que $p\Rightarrow q\equiv q\Rightarrow p$?
%     \end{exampleblock}

% \end{frame}

% %-------------------------------------------------------------------------

% \begin{frame}
%   \frametitle{\insertsectionhead}
%   \framesubtitle{\insertsubsectionhead}
  
%   \justifying
    
%     \begin{alertblock}{\color{red}{Ejercicio propuesto}}\pause
%     Considere el conectivo $\ast$ definido por $$\begin{tabular}{|c|c|c|}
%     \hline
%      $p$& $q$& $p\ast q$ \\
%      \hline
%      $V$& $V$& $F$\\
%      $V$& $F$& $V$\\
%      $F$& $V$& $F$\\
%      $F$& $F$& $F$\\
%      \hline
%      \end{tabular}$$

% Determine una proposición compuesta, lógicamente equivalente a $p\ast q$, que solo contenga los conectivos $\neg$ y $\wedge$. ?`Será verdad que $p\ast q\equiv q\ast p$?
%     \end{alertblock}

% \end{frame}

% %==============================================
% \section{Conclusión}
% %==============================================

% \begin{frame}
%   \frametitle{\insertsectionhead}
%   \framesubtitle{\insertsubsectionhead}
  
%   \justifying
  
%   \begin{itemize}[<+->]
%   \justifying
%       \item Las proposiciones que estudiaremos en este curso son de carácter netamente matemático.
%       \item Los cuatros conectivos lógicos estudiados hoy son los usuales entre dos proposiciones, pero se pueden definir 16 en total.
%       \item En general todos los conectivos se pueden definir en términos de $\neg$, $\wedge$ y $\vee$. Se recomienda escribir $\Rightarrow$ y $\Leftrightarrow$ usándolos.
%       % \item Una equivalencia lógica se puede interpretar como una ``traducción'' de una proposición en otra.
%   \end{itemize}

% \end{frame}

% %==============================================
% \section{Asistencia}
% %==============================================

% \begin{frame}
%   \frametitle{\insertsectionhead}
%   \framesubtitle{\insertsubsectionhead}
  
%   \justifying
  
%   \textbf{Solo se considerará la asistencia hasta 5 minutos después de terminada la clase.}
  
%   \begin{center}
%   \includegraphics[scale=0.23]{QR}
%   \end{center}
  
% \end{frame}


\end{document}