\documentclass[aspectratio=169]{beamer}
\usepackage{graphicx}
\usepackage{ragged2e}
\usefonttheme{professionalfonts}
\usepackage[spanish]{babel}
\usepackage{advdate}
\setbeamercovered{transparent}

% Document metadata
\title{Clase 2}
\subtitle{Equivalencias Lógicas. Álgebra proposicional}
\author{Profesor: Denis Osses}
% \date{\AdvanceDate[+3]\today}
\date{10 de marzo de 2025}

% Image for the title page (use includegraphics option to properly size/place it)
\titlegraphic{\includegraphics[height=\paperheight]{imagen5}}

\usetheme[sectionstyle=style2]{trigon}

% Define logos to use (comment if no logo)
\biglogo{logoFIC} % Used on titlepage only
%\smalllogo{gggg}% Used on top right corner of regular frames

% ------ If you want to change the theme default colors, do it here ------
%\definecolor{tPrim}{HTML}{00843B}   % Green
%\definecolor{tSec}{HTML}{289B38}    % Green light
%\definecolor{tAccent}{HTML}{F07F3C} % Orange


% ------ Packages and definitions used for this demo. Can be removed ------
\usepackage{appendixnumberbeamer} % To use \appendix command
\pdfstringdefDisableCommands{% Fix hyperref translate warning with \appendix
\def\translate#1{#1}%
}
\usepackage{pgf-pie} % For pie charts
\usepackage{caption} % For subfigures
\usepackage{subcaption} % For subfigures
\usepackage{xspace}
\newcommand{\themename}{\textbf{\textsc{Álgebra}}\xspace}
\usepackage[scale=2]{ccicons} % Icons for CC-BY-SA
\usepackage{booktabs} % Better tables


%==============================================================================
%                               BEGIN DOCUMENT
%==============================================================================
\begin{document}

%--------------------------------------
% Create title frame
\titleframe

%--------------------------------------
% Table of contents
\begin{frame}{Temario}
  \setbeamertemplate{section in toc}[sections numbered]
  \tableofcontents%[hideallsubsections]
\end{frame}

%==============================================
\section{Objetivos de hoy}
%==============================================
%\subsection{Charts}
\begin{frame}{\insertsectionhead}
  \framesubtitle{\insertsubsectionhead}
  
\justifying

\begin{itemize}[<+->]
    \item Demostrar equivalencias lógicas de proposiciones.
\end{itemize}

\end{frame}

%==============================================
\section{Contenidos}
%==============================================
\subsection{Equivalencias lógicas}
\begin{frame}
  \frametitle{\insertsectionhead}
  \framesubtitle{\insertsubsectionhead}
  
  \justifying
\footnotesize
  
\begin{minipage}[t]{0.5\linewidth}
\vspace{-2.5cm}
\begin{block}{Definición}
Dos proposiciones compuestas $m$ y $n$ son \textbf{equivalentes} si entregan el mismo valor de verdad para todo valor de verdad de $m$ y $n$. Escribimos $m\equiv n$ para indicar esto. 
\end{block}  \pause
    
De este modo  tenemos que $p\Rightarrow q\equiv \neg p\vee q$. \pause

\begin{exampleblock}{Ejemplo 1}
Pruebe que $p\Rightarrow q\equiv \neg q\Rightarrow\neg p.$
    \end{exampleblock}
\end{minipage}\pause
\hspace{0.5cm}\begin{minipage}{0.5\linewidth}
\justifying
    \begin{block}{Definición}
    Algunas equivalencias lógicas usuales son
    \begin{enumerate}[<+->]
    \item $\neg(\neg p)\equiv p$.
    \item $\neg(p\vee q)\equiv \neg p\wedge \neg q$.
    \item $\neg(p\wedge q)\equiv \neg p\vee \neg q$.
    \item $p\Rightarrow q\equiv \neg p\vee q$.
    \item $p\Rightarrow q\equiv \neg q\Rightarrow \neg p$.
    \item $p\Leftrightarrow q\equiv (p\Rightarrow q)\wedge(q\Rightarrow p)$.
    \end{enumerate}
    \end{block}\pause 
    
    \begin{exampleblock}{Ejemplo 2}
    ?`Es cierto que $p\Rightarrow q\equiv q\Rightarrow p$?
    \end{exampleblock}
\end{minipage}

\end{frame}

%-------------------------------------------------------------------------

\begin{frame}
  \frametitle{\insertsectionhead}
  \framesubtitle{\insertsubsectionhead}
  
  \justifying
    
    \begin{alertblock}{\color{red}{Ejercicio propuesto}}\pause
    Considere el conectivo $\ast$ definido por $$\begin{tabular}{|c|c|c|}
    \hline
     $p$& $q$& $p\ast q$ \\
     \hline
     $V$& $V$& $F$\\
     $V$& $F$& $V$\\
     $F$& $V$& $F$\\
     $F$& $F$& $F$\\
     \hline
     \end{tabular}$$

Determine una proposición compuesta, lógicamente equivalente a $p\ast q$, que solo contenga los conectivos $\neg$ y $\wedge$. ?`Será verdad que $p\ast q\equiv q\ast p$?
    \end{alertblock}

\end{frame}

%-------------------------------------------------------------------------
\subsection{Tautología, contradicción y contingencia}
\begin{frame}
  \frametitle{\insertsectionhead}
  \framesubtitle{\insertsubsectionhead}
  
  \justifying
\footnotesize
  
\begin{minipage}[t]{0.5\linewidth}
    
    \begin{block}{Definición}
    \justifying
    Una proposición cuyo valor de verdad es siempre $V$, cualquiera sea el valor de verdad de las proposiciones simples involucradas, se llama \textbf{tautología}.\pause
    \end{block}
    
    \begin{exampleblock}{Ejemplo 3}
    Probar que $(p\wedge q)\Rightarrow p$ es una tautología.\pause
    \end{exampleblock}
    
    \begin{block}{Definición}
    \justifying
     Una proposición cuyo valor de verdad es siempre $F$, cualquiera sea el valor de verdad de las proposiciones simples involucradas, se llama \textbf{contradicción}.
    \end{block}
\end{minipage}\pause
\hspace{0.5cm}\begin{minipage}[t]{0.5\linewidth}
\vspace{-0.5cm}
 \begin{exampleblock}{Ejemplo 4}
    Probar que $p\wedge\neg p$ es una contradicción.\pause
    \end{exampleblock}
    
    \begin{block}{Definición}
    \justifying
     Una proposición cuyo valor de verdad no es fijo, cualquiera sea el valor de verdad de las proposiciones simples involucradas, se llama \textbf{contingencia}.\pause
    \end{block}
    
    \begin{exampleblock}{Ejemplo 5}
    Determine si la siguiente proposición es una tautología, contradicción o contingencia $$((p \Rightarrow q) \wedge p \wedge \neg q) \Rightarrow (\neg p \vee q)$$
    \end{exampleblock}
\end{minipage}

\end{frame}

%-------------------------------------------------------------------------
%\subsection{Tautología, contradicción y contingencia}
\begin{frame}
  \frametitle{\insertsectionhead}
  \framesubtitle{\insertsubsectionhead}
  
  \justifying
  
    \begin{alertblock}{\color{red}{Ejercicio propuesto}}
    Pruebe las siguientes leyes o equivalencias lógicas
    \begin{enumerate}
        \item $p\vee p\equiv p$ (idempotencia)
        \item $p\wedge p\equiv p$ (idempotencia)
        \item $p\vee(q\vee r)\equiv (p\vee q)\vee r$ (asociatividad)
        \item $p\wedge(q\wedge r)\equiv (p\wedge q)\wedge r$ (asociatividad)
        \item $p\vee(q\wedge r)\equiv (p\vee q)\wedge(p\vee r)$ (distributividad)
        \item $p\wedge(q\vee r)\equiv (p\wedge q)\vee(p\wedge r)$ (distributividad)
        \item $p\wedge(p\vee q)\equiv p$ (absorción)
        \item $p\vee(p\wedge q)\equiv p$ (absorción)
    \end{enumerate}
    \end{alertblock}

\end{frame}


% %==============================================
% \section{Conclusión}
% %==============================================

% \begin{frame}
%   \frametitle{\insertsectionhead}
%   \framesubtitle{\insertsubsectionhead}
  
%   \justifying
  
%   \begin{itemize}[<+->]
%   \justifying
%       \item En general todos los conectivos se pueden definir en términos de $\neg$, $\wedge$ y $\vee$. Se recomienda escribir $\Rightarrow$ y $\Leftrightarrow$ usándolos.
%       \item Una equivalencia lógica se puede interpretar como una ``traducción'' de una proposición en otra.
%   \end{itemize}

% \end{frame}

% %==============================================
% \section{Asistencia}
% %==============================================

% \begin{frame}
%   \frametitle{\insertsectionhead}
%   \framesubtitle{\insertsubsectionhead}
  
%   \justifying
  
%   \textbf{Solo se considerará la asistencia hasta 5 minutos después de terminada la clase.}
  
%   \begin{center}
%   \includegraphics[scale=0.23]{QR}
%   \end{center}
  
% \end{frame}


\end{document}