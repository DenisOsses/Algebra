\documentclass[aspectratio=169]{beamer}
\usepackage{graphicx}
\usepackage{ragged2e}
\usefonttheme{professionalfonts}
\usepackage[spanish]{babel}
\usepackage{advdate}
\setbeamercovered{transparent}

% Document metadata
\title{Clase 4}
\subtitle{Álgebra de funciones proposicionales}
\author{Profesor: Denis Osses}
\date{\AdvanceDate[+1]\today}

% Image for the title page (use includegraphics option to properly size/place it)
\titlegraphic{\includegraphics[height=\paperheight]{imagen5}}

\usetheme[sectionstyle=style2]{trigon}

% Define logos to use (comment if no logo)
\biglogo{logoFIC} % Used on titlepage only
%\smalllogo{gggg}% Used on top right corner of regular frames

% ------ If you want to change the theme default colors, do it here ------
%\definecolor{tPrim}{HTML}{00843B}   % Green
%\definecolor{tSec}{HTML}{289B38}    % Green light
%\definecolor{tAccent}{HTML}{F07F3C} % Orange


% ------ Packages and definitions used for this demo. Can be removed ------
\usepackage{appendixnumberbeamer} % To use \appendix command
\pdfstringdefDisableCommands{% Fix hyperref translate warning with \appendix
\def\translate#1{#1}%
}
\usepackage{pgf-pie} % For pie charts
\usepackage{caption} % For subfigures
\usepackage{subcaption} % For subfigures
\usepackage{xspace}
\newcommand{\themename}{\textbf{\textsc{Álgebra}}\xspace}
\usepackage[scale=2]{ccicons} % Icons for CC-BY-SA
\usepackage{booktabs} % Better tables


%==============================================================================
%                               BEGIN DOCUMENT
%==============================================================================
\begin{document}

%--------------------------------------
% Create title frame
\titleframe

%--------------------------------------
% Table of contents
\begin{frame}{Temario}
  \setbeamertemplate{section in toc}[sections numbered]
  \tableofcontents%[hideallsubsections]
\end{frame}

%==============================================
\section{Objetivos de hoy}
%==============================================
%\subsection{Charts}
\begin{frame}{\insertsectionhead}
  \framesubtitle{\insertsubsectionhead}
  
\justifying

\begin{itemize}[<+->]
    \item Definir un conjunto por comprensión.
    \item Utilizar propiedades de conjuntos en demostraciones.
    % \item Utilizar proposiciones que contienen cuantificadores.
\end{itemize}

\end{frame}

%==============================================
\section{Contenidos}
%==============================================

%---------------------------------------------------------------------
\subsection{Funciones proposicionales}
\begin{frame}
  \frametitle{\insertsectionhead}
  \framesubtitle{\insertsubsectionhead}
  
  \justifying
\footnotesize
  
\begin{minipage}[t]{0.5\linewidth}
\vspace{-0.45cm}
Una expresión que involucra una variable $x$ no es una proposición, como ``$x$ es par", pero al darle valores a la variable $x$ se obtiene una proposición. ¿Dónde toma valores la variable $x$? Si se supone que $x$ toma valores en los números enteros se tiene que al reemplazar el valor de $x$ por 16, la proposición ``16 es par'' es verdadera y al reemplazar por 5 el valor de $x$ la proposición: ``5 es par'' es falsa.\pause

Cada vez que se defina una expresión que contenga una variable debe ser especificado el campo de ésta.\pause

\begin{block}{Definición}
    \justifying 
    Dado un conjunto $E$ (denominado \textbf{referencial} o \textbf{universal}), se llama \textbf{función proposicional} en una variable dentro de $E$, a toda expresión que contiene una variable $x$ y que conduce a una proposición para cada valor $b$ dado a $x$ en $E$. Esta función se denota por $p(x)$.
\end{block}
\end{minipage}\pause
\hspace{0.5cm}\begin{minipage}[t]{0.5\linewidth}
\justifying
Para definir una función proposicional se debe precisar:\pause

\begin{itemize}
\justifying
    \item Un conjunto referencial $E$ (el campo de la variable).\pause
    \item Una expresión que contenga una variable y tome valores en $E$.
\end{itemize} \pause

\begin{exampleblock}{Ejemplo 1}
\justifying
Si se tiene $E = \{1, 2, 3, 4, 5\}$ y $p(x): x$ es par, al darle valores a $x$ en $E$ se obtienen las proposiciones: $p(1) : 1$ es par, $p(2) : 2$ es par, $p(3) : 3$ es par, $p(4) : 4$ es par y $p(5) : 5$ es par. Las proposiciones $p(2)$ y $p(4)$ son verdaderas, así los elementos de $E$ que verifican $p(x)$ son: 2 y 4.
\end{exampleblock}
\end{minipage}

\end{frame}


%---------------------------------------------------------------------
%\subsection{Funciones proposicionales}
\begin{frame}
  \frametitle{\insertsectionhead}
  \framesubtitle{\insertsubsectionhead}
  
  \justifying
\footnotesize
  
\begin{minipage}[t]{0.5\linewidth}
\vspace{-0.5cm}
A toda función proposicional $p(x)$ definida sobre un conjunto $E$ se le asocia un conjunto formado por aquellos elementos de $E$ que verifican $p(x)$, esto constituye el axioma de comprensión.\pause

\begin{block}{Axioma de comprensión}
\justifying
Si $p(x)$ es una función proposicional definida sobre un conjunto $E$, entonces existe un conjunto cuyos elementos son los elementos de $E$ que verifican $p(x)$.
\end{block}\pause

Se designa este subconjunto de $E$ por $E_p$: $$E_p=\{x\in E: p(x)\}$$ y se lee: ``$E$ sub $p$ es el conjunto de los $x$ que pertenecen a $E$, tales que $p(x)$''. Note que, para todo $a\in E_p$, la proposición $p(a)$ es verdadera y para todo a $a\not\in E_p$, la proposición es falsa.
\end{minipage}\pause
\hspace{0.5cm}\begin{minipage}[t]{0.5\linewidth}
\justifying
\vspace{-0.7cm}
\begin{exampleblock}{Ejemplo 2}
\justifying
Sea $E=\mathbb{N}$, asociado a la función proposicional $p(x) : x$ es menor que 6. Así $$E_p=\{1,2,3,4,5\}.$$  Asociado a la función proposicional $q(x) : x$ es múltiplo de 5, tenemos que $$E_p=\{5,10,15,20,25,\ldots,5n,\ldots\}.$$
\end{exampleblock}\pause

\begin{block}{Nota}
\justifying
Un subconjunto $A$ de $E$, asociado a una forma proposicional $p(x)$, se dice definido por comprensión: $A = E_p = \{x\in E: p(x)\}$.
\end{block}

\end{minipage}


\end{frame}



%---------------------------------------------------------------------
\subsection{Álgebra de funciones proposicionales}
\begin{frame}
  \frametitle{\insertsectionhead}
  \framesubtitle{\insertsubsectionhead}

    \justifying
\footnotesize
  
\begin{minipage}[t]{0.5\linewidth}

Sean $p(x)$ y $q(x)$ funciones proposicionales definidas sobre un mismo conjunto referencial $E$, para cada valor $a$ dado a $x$, se obtienen las proposiciones $p(a)$ y $q(a)$ a las cuales se les puede aplicar el álgebra de proposiciones.\pause 

\hspace{0.1cm}

Las proposiciones $\neg p(a), p(a)\vee q(a)$, $p(a)\wedge q(a)$, $p(a)\Rightarrow q(a)$, son verdaderas o falsas dependiendo
del valor que tome $a$ en $E$ y determinan las funciones proposicionales: $\neg p(x), p(x)\vee q(x)$, $p(x)\wedge q(x)$, $p(x)\Rightarrow q(x)$, definidas sobre $E$, que a su vez determinan los subconjuntos de $E$ cuyos elementos verifican
estas funciones proposicionales.
\end{minipage}\pause
\hspace{0.5cm}\begin{minipage}[t]{0.5\linewidth}
\vspace{-1.5cm}
\begin{exampleblock}{Ejemplo 3}
\justifying
Considere las funciones proposicionales: $p(x): x\leq 2$, $q(x): x^2\leq 9$ y $\neg(p(x)\Rightarrow q(x))$, definidas sobre el conjunto referencial $\mathbb{N}$. \pause Observe que: $p(a)$ es verdadera para todo
$a\in \mathbb{N}_p = \{1, 2\}$, $q(a)$ es verdadera para todo $a\in \mathbb{N}_q = \{1,2,3\}$ y $\neg(p(a)\Rightarrow q(a))$ es verdadera para todo $a\not\in \mathbb{N}_q$ y todo $a\in\mathbb{N}_p$. Luego $$\{x\in\mathbb{N}:\neg(p(x)\Rightarrow q(x))\}$$ \pause es un conjunto sin elementos.
\end{exampleblock}\pause

\vspace{-0.2cm}
\begin{block}{Definición}
\justifying
Sea $E$ un conjunto referencial, a la función proposicional $p(x) : x\not\in E$ se le asocia, de acuerdo al axioma de comprensión, un conjunto. Este conjunto no tiene elementos por lo que se denomina \textbf{conjunto vacío}. Se denota por $\emptyset$: 

\vspace{-0.5cm}
$$\emptyset=\{x\in E:x\not\in E\}.$$
\vspace{-0.4cm}
\end{block}
\end{minipage}

\end{frame}

%---------------------------------------------------------------------
%\subsection{Álgebra de funciones proposicionales}
\begin{frame}
  \frametitle{\insertsectionhead}
  \framesubtitle{\insertsubsectionhead}
  
    \justifying
\footnotesize
  
\begin{minipage}[t]{0.5\linewidth}

\begin{block}{Definición}
\justifying
Sea $A$ un subconjunto de un conjunto $E$. El \textbf{complemento} de $A$ en $E$ es el conjunto asociado a la
función proposicional $p(x) : x\not\in A$. Existen varias formas en que se suele denotar el complemento de un conjunto las mas frecuentes son: $A^c, \overline{A}$.
\end{block} \pause 

Evidentemente este conjunto depende tanto de $A$ como del referencial $E$: $$A^c=\{x\in E:x\not\in A\}.$$
\end{minipage}\pause
\hspace{0.5cm}\begin{minipage}[t]{0.5\linewidth}
\vspace{-0.5cm}
\begin{block}{Definición}
\justifying
Si $A$ y $B$ son dos subconjuntos de un referencial $E$, entonces la \textbf{unión} $A\cup B$ es el conjunto asociado a la función proposicional: $(x\in A)\vee (x\in B)$ definida en $E$: \pause  $$A\cup B=\{x\in E:(x\in A)\vee (x\in B)\}.$$
\end{block}\pause 

\begin{block}{Definición}
\justifying
Si $A$ y $B$ son dos subconjuntos de un referencial $E$, entonces la \textbf{intersección} $A\cap B$ es el conjunto asociado a la función proposicional: $(x\in A)\wedge (x\in B)$ definida en $E$: \pause $$A\cap B=\{x\in E:(x\in A)\wedge (x\in B)\}.$$
\end{block}
\end{minipage}

\end{frame}

%---------------------------------------------------------------------
%\subsection{Álgebra de funciones proposicionales}
\begin{frame}
  \frametitle{\insertsectionhead}
  \framesubtitle{\insertsubsectionhead}
    \justifying
\footnotesize
  
\begin{minipage}[t]{0.5\linewidth}
  
 \begin{exampleblock}{Ejemplo 4}
\justifying
Se define la \textbf{diferencia} entre los conjuntos $A$ y $B$ como $$A-B=\{x\in E:x\in A \wedge x\not\in B\}$$ Pruebe que $A-B=A\cap B^c$.
\end{exampleblock}\pause

 \begin{alertblock}{\color{red}{Ejercicio propuesto}}
\justifying
Demuestre que 
\begin{enumerate}
    \item $(A^c)^c=A$.
    \item $(A\cup B)^c=A^c\cap B^c$.
    \item $(A\cap B)^c=A^c\cup B^c$.
\end{enumerate}
\end{alertblock}
\end{minipage}

\end{frame}

% %---------------------------------------------------------------------
% \subsection{Cuantificadores}
% \begin{frame}
%   \frametitle{\insertsectionhead}
%   \framesubtitle{\insertsubsectionhead}
  
%   \justifying
  
% Dada la función proposicional $p(x)$, sobre el conjunto referencial $\mathbb{N}$, definida por $$p(x) : x ~\text{es múltiplo de}~ 5$$\pause observe que las expresiones: ``Todos los $x\in\mathbb{N}$ son múltiplos de 5'' y ``existe un $x\in\mathbb{N}$ tal que $x$ es múltiplo de 5'', ya no son funciones proposicionales sino proposiciones, debido a que se puede determinar con precisión su valor de verdad.\pause

% \hspace{0.1cm}

% Partiendo de una función proposicional $p(x)$, se puede obtener una proposición anteponiendo un \textbf{cuantificador}. Los cuantificadores más usados son el \textbf{universal} y el \textbf{existencial}.

% \end{frame}

% %---------------------------------------------------------------------
% %\subsection{Cuantificadores}
% \begin{frame}
%   \frametitle{\insertsectionhead}
%   \framesubtitle{\insertsubsectionhead}
  
%   \justifying
  
%   \begin{block}{Definición}\pause
%   \justifying
%   Sea $p(x)$ una función proposicional definida sobre un conjunto referencial $E$. Si \underline{para cada valor $a\in E$} se tiene que $p(a)$ es verdadera se escribe $$\forall~x\in E:p(x)$$ y se lee: ``Para todo $x$ en $E$ se verifica $p(x)$'' o ``Cualquiera que sea $x$ en $E$ se verifica $p(x)$''. \pause El símbolo $\forall$, que transforma la función proposicional $p(x)$ en una proposición, se denomina \textbf{cuantificador universal}.
%   \end{block}\pause
  
%   Esta proposición es verdadera si todos los elementos de $E$ hacen que $p(x)$ sea verdadera.
  
  
% \end{frame}
  
%   %---------------------------------------------------------------------
% %\subsection{Cuantificadores}
% \begin{frame}
%   \frametitle{\insertsectionhead}
%   \framesubtitle{\insertsubsectionhead}
  
%   \justifying
  
%   \begin{block}{Definición}\pause
%   \justifying
%   Sea $p(x)$ una función proposicional definida sobre un conjunto referencial $E$. Si \underline{para al menos un valor $a\in E$} se tiene que $p(a)$ es verdadera se escribe $$\exists~x\in E:p(x)$$ y se lee: ``Existe un $x$ en $E$ que verifica $p(x)$'' o ``Para algún $x$ en $E$ que verifica $p(x)$''. \pause El símbolo $\exists$, que transforma la función proposicional $p(x)$ en una proposición, se denomina \textbf{cuantificador existencial}.\pause

%   \end{block}
  
%   Esta proposición es verdadera si existe al menos un elemento de $E$ tal que $p(x)$ sea verdadera.
  
% \end{frame}

%   %---------------------------------------------------------------------
% %\subsection{Cuantificadores}
% \begin{frame}
%   \frametitle{\insertsectionhead}
%   \framesubtitle{\insertsubsectionhead}
  
%   \justifying
  
%   \begin{exampleblock}{Ejemplo 3}
%   \justifying
%     Considere la función proposicional ``$p(x): x$ es positivo''. Analice el valor de verdad de las siguientes proposiciones:
%     \begin{enumerate}
%         \item $\forall~x\in\mathbb{R}:p(x)$.
%         \item $\forall~x\in[1,10]:p(x)$.
%         \item $\exists~x\in\mathbb{R}:p(x)$.
%         \item $\exists~x\in[-10,-1]:p(x)$.
%     \end{enumerate}
%   \end{exampleblock}
  
% \end{frame}


% %==============================================
% \section{Conclusión}
% %==============================================

% \begin{frame}
%   \frametitle{\insertsectionhead}
%   \framesubtitle{\insertsubsectionhead}
  
%   \justifying
  
%   \begin{itemize}[<+->]
%   \justifying
%       \item Existe una clara analogía entre álgebra de funciones proposicionales y la teoría de conjuntos.
%       % \item Los cuantificadores siempre transforman una función proposicional en una proposición.
%   \end{itemize}

% \end{frame}

% %==============================================
% \section{Asistencia}
% %==============================================

% \begin{frame}
%   \frametitle{\insertsectionhead}
%   \framesubtitle{\insertsubsectionhead}
  
%   \justifying
  
%   \textbf{Solo se considerará la asistencia hasta 5 minutos después de terminada la clase.}
  
%   \begin{center}
%   \includegraphics[scale=0.23]{QR}
%   \end{center}
  
% \end{frame}


\end{document}