\documentclass[aspectratio=169]{beamer}
\usepackage{graphicx}
\usepackage{ragged2e}
\usefonttheme{professionalfonts}
\usepackage[spanish]{babel}
\usepackage{advdate}
\setbeamercovered{transparent}

% Document metadata
\title{Clase 6}
\subtitle{Métodos de demostración}
\author{Profesor: Denis Osses}
%\date{\AdvanceDate[+3]\today}
%\date{\today}
\date{19 de marzo de 2025}

% Image for the title page (use includegraphics option to properly size/place it)
\titlegraphic{\includegraphics[height=\paperheight]{imagen5}}

\usetheme[sectionstyle=style2]{trigon}

% Define logos to use (comment if no logo)
\biglogo{logoFIC} % Used on titlepage only
%\smalllogo{gggg}% Used on top right corner of regular frames

% ------ If you want to change the theme default colors, do it here ------
%\definecolor{tPrim}{HTML}{00843B}   % Green
%\definecolor{tSec}{HTML}{289B38}    % Green light
%\definecolor{tAccent}{HTML}{F07F3C} % Orange


% ------ Packages and definitions used for this demo. Can be removed ------
\usepackage{appendixnumberbeamer} % To use \appendix command
\pdfstringdefDisableCommands{% Fix hyperref translate warning with \appendix
\def\translate#1{#1}%
}
\usepackage{pgf-pie} % For pie charts
\usepackage{caption} % For subfigures
\usepackage{subcaption} % For subfigures
\usepackage{xspace}
\newcommand{\themename}{\textbf{\textsc{Álgebra}}\xspace}
\usepackage[scale=2]{ccicons} % Icons for CC-BY-SA
\usepackage{booktabs} % Better tables


%==============================================================================
%                               BEGIN DOCUMENT
%==============================================================================
\begin{document}

%--------------------------------------
% Create title frame
\titleframe

%--------------------------------------
% Table of contents
\begin{frame}{Temario}
  \setbeamertemplate{section in toc}[sections numbered]
  \tableofcontents%[hideallsubsections]
\end{frame}

%==============================================
\section{Objetivos de hoy}
%==============================================
%\subsection{Charts}
\begin{frame}{\insertsectionhead}
  \framesubtitle{\insertsubsectionhead}
  
\justifying

\begin{itemize}[<+->]
    \item Demostrar propiedades de números y conjuntos.
    \item Demostrar propiedades de los números reales.
\end{itemize}

\end{frame}

%==============================================
\section{Contenidos}
%==============================================

%---------------------------------------------------------------------
\subsection{Demostración directa}
\begin{frame}
  \frametitle{\insertsectionhead}
  \framesubtitle{\insertsubsectionhead}
  
  \justifying
\footnotesize
  
\begin{minipage}[t]{0.5\linewidth}
  
Un \textbf{Teorema} es una proposición cuya verdad debe demostrarse. En matemáticas, corresponde a toda proposición que, partiendo de supuestos o \textbf{hipótesis}, produce una afirmación no evidente por sí misma o \textbf{tesis}.\pause

\hspace{0.1cm}

Los teoremas se deben demostrar mediante razonamientos lógicos. El proceso de razonar lógicamente y deducir proposiciones, propiedades o teoremas, se puede sistematizar mediante los denominados \textbf{métodos de demostración}. Estudiaremos 4 de ellos:\pause

 \begin{block}{Demostración directa}
\justifying
Supongamos que una proposición es de la forma $p\Rightarrow q$. Una \textbf{demostración directa} de ella es suponer $p$ verdadera y, a partir de deducciones lógicas, probar $q$. 
\end{block}
\end{minipage}\pause
\hspace{0.4cm}\begin{minipage}[t]{0.5\linewidth}
\justifying
\begin{exampleblock}{Ejemplo 1}
\justifying
Demuestre que $\forall~n\in\mathbb{Z}: n~\textrm{es impar}\Rightarrow n^2~\textrm{es impar}.$
\end{exampleblock}\pause

\begin{block}{Nota}
\justifying
En una proposición de la forma $p\Rightarrow q$, $p$ se denomina \textbf{condición suficiente} (para que ocurra $q$) y $q$ \textbf{condición necesaria} (para que se satisfaga $p$). \pause Por ejemplo si $p:$ llueve y $q:$ la calle se moja, podemos leer $p\Rightarrow q$ como ``si llueve entonces la calle se moja'' que es $V$. Sin embargo, no es necesariamente cierto que si la calle se moja entonces llueve ($q\Rightarrow p$).

% \pause

% Es suficiente que llueva para que la calle se moje, por ejemplo.

\end{block}
\end{minipage}

\end{frame}

%---------------------------------------------------------------------
\subsection{Demostración por contrarrecíproco y reducción al absurdo}
\begin{frame}
  \frametitle{\insertsectionhead}
  \framesubtitle{\insertsubsectionhead}
  
  \justifying
\footnotesize
  
\begin{minipage}[t]{0.5\linewidth}

 \begin{block}{Demostración por contrarrecíproco}
\justifying
Supongamos que una proposición es de la forma $p\Rightarrow q$. Una \textbf{demostración por contrarrecíproco} de ella se basa en la equivalencia lógica $$p\Rightarrow q\equiv\neg q\Rightarrow\neg p.$$\pause Es decir, probar que $p\Rightarrow q$ es verdadera equivale a demostrar que $\neg q\Rightarrow\neg p$ es también verdadera (lo que se puede hacer de manera directa).
\end{block}\pause

\begin{exampleblock}{Ejemplo 2}
\justifying
Demuestre que $\forall~n\in\mathbb{Z}: n^2~\textrm{es par}\Rightarrow n~\textrm{es par}.$
\end{exampleblock}
\end{minipage}\pause
\hspace{0.4cm}\begin{minipage}[t]{0.5\linewidth}
\justifying
 \begin{block}{Demostración por reducción al absurdo}
\justifying
Supongamos que queremos demostrar la proposición $p$. \pause En lugar de demostrarla directamente, comenzamos suponiendo que $\neg p$ es verdadera y, a partir de ella, deducimos una contradicción (una proposición siempre falsa). \pause Esta contradicción proviene de suponer que $\neg p\equiv V$, lo que es incorrecto; así que, $\neg p\equiv F$. \pause Esto implica que $p\equiv V$.
\end{block}\pause

\begin{exampleblock}{Ejemplo 3}
\justifying
Probar que $\sqrt{2}$ no es racional.
\end{exampleblock}
\end{minipage}

\end{frame}


%---------------------------------------------------------------------
\subsection{Demostración por casos}
\begin{frame}
  \frametitle{\insertsectionhead}
  \framesubtitle{\insertsubsectionhead}
  
  \justifying
\footnotesize
  
\begin{minipage}[t]{0.5\linewidth}
 \begin{block}{Demostración por casos}
\justifying
La \textbf{demostración por casos} es un método en el cual la proposición a ser probada se divide en un número finito de casos y cada uno de ellos es demostrado por separado.
\end{block}\pause

\begin{exampleblock}{Ejemplo 4}
\justifying
Probar que $\forall~x\in\mathbb{R}: x^2\geq0$.
\end{exampleblock}
\end{minipage}\pause
\hspace{0.4cm}\begin{minipage}[t]{0.51\linewidth}
\justifying
\begin{alertblock}{\color{red}{Ejercicio propuesto}}
\justifying
Probar que $$\forall~a,b,c\in\mathbb{R^+}: ab(a+b)+bc(b+c)+ca(a+c)\geq6abc$$
\end{alertblock}
\end{minipage}

\end{frame}

% %==============================================
% \section{Conclusión}
% %==============================================

% \begin{frame}
%   \frametitle{\insertsectionhead}
%   \framesubtitle{\insertsubsectionhead}
  
%   \justifying
  
%   \begin{itemize}[<+->]
%   \justifying
%       \item Las demostraciones son fundamentales para establecer las verdades matemáticas.
%       \item En cada demostración que realice, es importante identificar el método, hipótesis, tesis, y condiciones necesarias y suficientes (si aplica).
%   \end{itemize}

% \end{frame}

% %==============================================
% \section{Asistencia}
% %==============================================

% \begin{frame}
%   \frametitle{\insertsectionhead}
%   \framesubtitle{\insertsubsectionhead}
  
%   \justifying
  
%   \textbf{Solo se considerará la asistencia hasta 5 minutos después de terminada la clase.}
  
%   \begin{center}
%   \includegraphics[scale=0.23]{QR}
%   \end{center}
  
% \end{frame}


\end{document}