\documentclass[aspectratio=169]{beamer}
\usepackage{graphicx}
\usepackage{ragged2e}
\usefonttheme{professionalfonts}
\usepackage[spanish]{babel}
\usepackage{advdate}
\setbeamercovered{transparent}

% Document metadata
\title{Clases 7-8}
\subtitle{Sucesiones. Principio de inducción. Ejemplos}
\author{Profesor: Denis Osses}
%\date{\AdvanceDate[+1]\today}
\date{24 y 26 de marzo de 2025}

% Image for the title page (use includegraphics option to properly size/place it)
\titlegraphic{\includegraphics[height=\paperheight]{imagen5}}

\usetheme[sectionstyle=style2]{trigon}

% Define logos to use (comment if no logo)
\biglogo{logoFIC} % Used on titlepage only
%\smalllogo{gggg}% Used on top right corner of regular frames

% ------ If you want to change the theme default colors, do it here ------
%\definecolor{tPrim}{HTML}{00843B}   % Green
%\definecolor{tSec}{HTML}{289B38}    % Green light
%\definecolor{tAccent}{HTML}{F07F3C} % Orange


% ------ Packages and definitions used for this demo. Can be removed ------
\usepackage{appendixnumberbeamer} % To use \appendix command
\pdfstringdefDisableCommands{% Fix hyperref translate warning with \appendix
\def\translate#1{#1}%
}
\usepackage{pgf-pie} % For pie charts
\usepackage{caption} % For subfigures
\usepackage{subcaption} % For subfigures
\usepackage{xspace}
\newcommand{\themename}{\textbf{\textsc{Álgebra}}\xspace}
\usepackage[scale=2]{ccicons} % Icons for CC-BY-SA
\usepackage{booktabs} % Better tables


%==============================================================================
%                               BEGIN DOCUMENT
%==============================================================================
\begin{document}

%--------------------------------------
% Create title frame
\titleframe

%--------------------------------------
% Table of contents
\begin{frame}{Temario}
  \setbeamertemplate{section in toc}[sections numbered]
  \tableofcontents%[hideallsubsections]
\end{frame}

%==============================================
\section{Objetivos de hoy}
%==============================================
%\subsection{Charts}
\begin{frame}{\insertsectionhead}
  \framesubtitle{\insertsubsectionhead}
  
\justifying

\begin{itemize}[<+->]
    \item Reconocer diferentes tipos de sucesiones. 
    \item Aplicar el principio de inducción.
\end{itemize}

\end{frame}

%==============================================
\section{Contenidos}
%==============================================

%---------------------------------------------------------------------
\subsection{Sucesiones}
\begin{frame}
  \frametitle{\insertsectionhead}
  \framesubtitle{\insertsubsectionhead}
  
  \justifying
\footnotesize
  
\begin{minipage}[t]{0.5\linewidth}

\vspace{-0.4cm}
 \begin{block}{Definición}
\justifying
 Sea $A$ un conjunto dado. Una \textbf{sucesión} es una función  de la forma $f: D \subseteq \mathbb{N} \to A$. \pause Así, para cada $n \in D$, $f(n)$ es un elemento de A. \pause Usualmente escribimos $f(n):=f_n$ o $f(n):=a_n$ para referirnos al término general de la sucesión y $\{f_n\}_{n\in D}$ para la sucesión.
\end{block}\pause 

 \begin{exampleblock}{Ejemplo 1}
\justifying
Si $A=\mathbb{R}$ entonces diremos que $\{f_n\}_{n\in D}$ es una sucesión real.\pause

\begin{enumerate}[<+->]
 \justifying
     \item $f:\mathbb{N}\to\mathbb{R}$ dada por $f(n)=\dfrac1{n}$ o $\left\{\dfrac1{n}\right\}_{n\in\mathbb{N}}$.
     \item $f:\{1,2,3,4,5\}\to\mathbb{R}$ dada por $f(n)=(-1)^n$ o $\{-1,1\}$.
     %\item $A=\left\{y\in\mathbb{R}:y=f(n)=\dfrac{n-1}{n+1}, n\in\mathbb{N}\right\}$.
 \end{enumerate}
\end{exampleblock}
\end{minipage}\pause
\hspace{0.4cm}\begin{minipage}[t]{0.5\linewidth}
\justifying
 \begin{exampleblock}{Ejemplo 2}
\justifying
Si $A=\{p: p~\text{es una proposición}\}$, entonces $\{f_n\}$ es una sucesión de proposiciones, es decir, una función proposicional cuyo dominio es un subconjunto de $\mathbb{N}$.\pause

 \begin{enumerate}[<+->]
 \justifying
     \item $f(n): n^2+n$ es par.
     \item $\displaystyle \phi(n): 1+2+3+\cdots+n=\frac{n(n+1)}{2}$.
     \item $q(n): \left(1+\dfrac1{n}\right)^{n}>2$.
 \end{enumerate}
\end{exampleblock}
\end{minipage}

\end{frame}

%---------------------------------------------------------------------
\subsection{Sucesiones importantes}
\begin{frame}
  \frametitle{\insertsectionhead}
  \framesubtitle{\insertsubsectionhead}
  
  \justifying
\footnotesize
  
\begin{minipage}[t]{0.5\linewidth}
\vspace{-0.4cm}
\begin{block}{Definición}
\justifying
Sean $d\in\mathbb{R}$ y $a\in\mathbb{R}$. Una \textbf{progresión aritmética} (P.A.) de primer término $a$ y diferencia $d$ es una sucesión $\{a_n\}_{n\in\mathbb{N}}$ tal que \begin{eqnarray*}a_1&=&a\\ a_{n+1}&=&a_n+d~,~n\in\mathbb{N} \end{eqnarray*}\pause Lo que caracteriza a una P.A. es que la diferencia de dos términos consecutivos de la sucesión es siempre constante igual a $d$.
\end{block}\pause

\begin{exampleblock}{Ejemplo}
\justifying
La sucesión $1,3,5,7,9,11,\ldots$ es una P.A. de primer término $a=1$ y diferencia $d=2$. 

\end{exampleblock}
\end{minipage}\pause
\hspace{0.4cm}\begin{minipage}[t]{0.5\linewidth}
\justifying
\vspace{-0.4cm}
\begin{block}{Definición}
\justifying
Sean $r\in\mathbb{R}-\{0\}$ y $a\in\mathbb{R}$. Una \textbf{progresión geométrica} (P.G.) de primer término $a$ y razón $r$ es una sucesión $\{a_n\}_{n\in\mathbb{N}}$ tal que \begin{eqnarray*}a_1&=&a\\ a_{n+1}&=&a_n\cdot r~,~n\in\mathbb{N} \end{eqnarray*} \pause Lo que caracteriza a una P.G. es que la razón de dos términos consecutivos de la sucesión es siempre constante igual a $r$.
\end{block}\pause

\begin{exampleblock}{Ejemplo}
\justifying
La sucesión $1,\frac{1}{2},\frac{1}{4},\frac{1}{8},\frac{1}{16},\ldots$ es una P.G. de primer término $a=1$ y razón $r=\frac{1}{2}$.
\end{exampleblock}
\end{minipage}

\end{frame}

% %---------------------------------------------------------------------
% %\subsection{Sucesiones importantes}
% \begin{frame}
%   \frametitle{\insertsectionhead}
%   \framesubtitle{\insertsubsectionhead}
  
%   \justifying

% \begin{block}{Definición}\pause
% \justifying
% Una sucesión $\{a_n\}_{n\in\mathbb{N}}$ es una \textbf{progresión armónica} (P.H.) de primer término $a$ si existe $d\in\mathbb{R}$ tal que \begin{eqnarray*}a_1&=&a\\ a_{n+1}&=&\frac{a_n}{1+a_n\cdot d}~,~n\in\mathbb{N} \end{eqnarray*}
% \end{block}\pause

% \begin{exampleblock}{Ejemplo}
% \justifying
% La sucesión $1,\frac{1}{2},\frac{1}{3},\frac{1}{4},\frac{1}{5},\ldots$ es una P.H. de primer término $a=1$, con $d=1$.
% \end{exampleblock}

% \end{frame}


%---------------------------------------------------------------------
\subsection{Primer principio de inducción}
\begin{frame}
  \frametitle{\insertsectionhead}
  \framesubtitle{\insertsubsectionhead}
  
  \justifying
\footnotesize
  
\begin{minipage}[t]{0.5\linewidth}

 \begin{block}{Definición}
\textbf{Primer Principio de Inducción}
\vspace{0.1cm}
\justifying
Sea $\phi(n)$ una función proposicional. Entonces $$\big[\phi(1)\wedge\forall~n\in\mathbb{N}: \big(\phi(n)\Rightarrow\phi(n+1)\big)\big]\Rightarrow\forall~n\in\mathbb{N}:\phi(n)$$\pause
Este principio nos dice que para probar una proposición $\phi(n)$ para todo $n\in\mathbb{N}$, basta con demostrar dos cosas:\pause

\begin{enumerate}[<+->]
\justifying
    \item $\phi(1)$.
    \item $\forall~n\in\mathbb{N}: \phi(n)\Rightarrow\phi(n+1)$.
\end{enumerate}

\end{block}
\end{minipage}\pause
\hspace{0.4cm}\begin{minipage}[t]{0.5\linewidth}
\justifying
\begin{exampleblock}{Ejemplo 3}
\justifying
Demuestre las siguientes proposiciones:\pause
\begin{enumerate}[<+->]
    \item $\forall~n \in \mathbb{N}: n(n+1)$ es par.
    \item $\forall~n \in \mathbb{N}: n^3-n$ es divisible por 6.
    \item $\forall~n\in\mathbb{N}: 1+2+3+\ldots+n=\dfrac{n(n+1)}{2}.$
    \item $\forall~n \in \mathbb{N}: 4^n\geq 3n+1$.
\end{enumerate}

\end{exampleblock}

\end{minipage}


\end{frame}

%---------------------------------------------------------------------
%\subsection{Principio de inducción}
\begin{frame}
  \frametitle{\insertsectionhead}
  \framesubtitle{\insertsubsectionhead}
  
  \justifying
\footnotesize
  
\begin{minipage}[t]{0.5\linewidth}
  
\begin{exampleblock}{Ejemplo 4}
\justifying
Demuestre que $$\forall~n\in\mathbb{N}:1+8+27+\cdots+n^3=\frac{n^2(n+1)^2}{4}.$$
\end{exampleblock}\pause

\begin{exampleblock}{Ejemplo 5}
\justifying
Demuestre que si $a_1=1$ y $a_n=2a_{n-1}+2^{n-1}$ para $n\geq2$, entonces para todo $n\in\mathbb{N}$ se tiene que $a_n=n2^{n-1}$.
\end{exampleblock}
\end{minipage}

\end{frame}


% %==============================================
% \section{Conclusión}
% %==============================================

% \begin{frame}
%   \frametitle{\insertsectionhead}
%   \framesubtitle{\insertsubsectionhead}
  
%   \justifying
  
%   \begin{itemize}[<+->]
%   \justifying
%       \item El primer principio de inducción se utiliza solamente para proposiciones sobre $\mathbb{N}$, no $\mathbb{R}$.
%       \item Utilizamos este principio principalmente para demostrar propiedades de divisibilidad, sumas, desigualdades (que involucren algún término natural) y fórmulas recursivas. 
%   \end{itemize}

% \end{frame}

% %==============================================
% \section{Asistencia}
% %==============================================

% \begin{frame}
%   \frametitle{\insertsectionhead}
%   \framesubtitle{\insertsubsectionhead}
  
%   \justifying
  
%   \textbf{Solo se considerará la asistencia hasta 5 minutos después de terminada la clase.}
  
%   \begin{center}
%   \includegraphics[scale=0.23]{QR}
%   \end{center}
  
% \end{frame}


\end{document}