\documentclass[aspectratio=169]{beamer}
\usepackage{graphicx}
\usepackage{ragged2e}
\usefonttheme{professionalfonts}
\usepackage[spanish]{babel}
\usepackage{advdate}
\setbeamercovered{transparent}

% Document metadata
\title{Clase 9}
\subtitle{Sumatorias}
\author{Profesor: Denis Osses}
% \date{5 y \AdvanceDate[+3]\today}
%\date{\today}
\date{31 de marzo de 2025}

% Image for the title page (use includegraphics option to properly size/place it)
\titlegraphic{\includegraphics[height=\paperheight]{imagen5}}

\usetheme[sectionstyle=style2]{trigon}

% Define logos to use (comment if no logo)
\biglogo{logoFIC} % Used on titlepage only
%\smalllogo{gggg}% Used on top right corner of regular frames

% ------ If you want to change the theme default colors, do it here ------
%\definecolor{tPrim}{HTML}{00843B}   % Green
%\definecolor{tSec}{HTML}{289B38}    % Green light
%\definecolor{tAccent}{HTML}{F07F3C} % Orange


% ------ Packages and definitions used for this demo. Can be removed ------
\usepackage{appendixnumberbeamer} % To use \appendix command
\pdfstringdefDisableCommands{% Fix hyperref translate warning with \appendix
\def\translate#1{#1}%
}
\usepackage{pgf-pie} % For pie charts
\usepackage{caption} % For subfigures
\usepackage{subcaption} % For subfigures
\usepackage{xspace}
\newcommand{\themename}{\textbf{\textsc{Álgebra}}\xspace}
\usepackage[scale=2]{ccicons} % Icons for CC-BY-SA
\usepackage{booktabs} % Better tables


%==============================================================================
%                               BEGIN DOCUMENT
%==============================================================================
\begin{document}

%--------------------------------------
% Create title frame
\titleframe

%--------------------------------------
% Table of contents
\begin{frame}{Temario}
  \setbeamertemplate{section in toc}[sections numbered]
  \tableofcontents%[hideallsubsections]
\end{frame}

%==============================================
\section{Objetivos de hoy}
%==============================================
%\subsection{Charts}
\begin{frame}{\insertsectionhead}
  \framesubtitle{\insertsubsectionhead}
  
\justifying

\begin{itemize}[<+->]
    \item Definir y calcular sumas aplicando sus propiedades.
\end{itemize}

\end{frame}

%==============================================
\section{Contenidos}
%==============================================

%---------------------------------------------------------------------
\subsection{Sumatorias. Propiedades}
\begin{frame}
  \frametitle{\insertsectionhead}
  \framesubtitle{\insertsubsectionhead}
  
  \justifying
\footnotesize
  
\begin{minipage}[t]{0.5\linewidth}
  
\begin{block}{Definición}
\justifying
Considere la sucesión $\{f(k)\}_{k\in\mathbb{N}}=\{a_k\}_{k\in\mathbb{N}}$. La suma de los $n$ primeros términos de esta sucesión se denomina \textbf{sumatoria} de tales términos. \pause Anotamos $$\sum_{k=1}^n a_k=a_1+a_2+\cdots+a_n.$$ \pause Notamos que $$\sum_{k=1}^1 a_k=a_1~~,~~\sum_{k=1}^{n+1} a_k=\left(\sum_{k=1}^{n}a_k\right)+a_{n+1}.$$
\end{block}
\end{minipage}\pause
\hspace{0.4cm}\begin{minipage}[t]{0.5\linewidth}
  \justifying
  \vspace{-0.5cm}
\begin{block}{Propiedades iniciales}
\justifying
\begin{enumerate}[<+->]
\justifying
    \item $\displaystyle\sum_{k=1}^{n}c=\underbrace{c+c+c+\cdots+c}_{n~\text{veces}}=nc,$ donde $c$ es constante.
    \item $\displaystyle\sum_{k=1}^{n}k=1+2+3+\cdots+n=\frac{n(n+1)}{2}$.
    \item $\displaystyle\sum_{k=1}^{n}k^2=1^2+2^2+3^2+\cdots+n^2=\frac{n(n+1)(2n+1)}{6}$.
    \item $\displaystyle\sum_{k=1}^{n}k^3=1^3+2^3+3^3+\cdots+n^3=\frac{n^2(n+1)^2}{4}$.
\end{enumerate}
\end{block}
\end{minipage}

\end{frame}

%---------------------------------------------------------------------
\subsection{Propiedades}
\begin{frame}
  \frametitle{\insertsectionhead}
  \framesubtitle{\insertsubsectionhead}
  
  \justifying
\footnotesize
  
\begin{minipage}[t]{0.5\linewidth}

\begin{block}{Sumas importantes}
\justifying
\begin{enumerate}[<+->]
\justifying
    \item \textbf{Aritmética}: $\displaystyle\sum_{k=1}^{n}\left(a+kd\right)=an+d\cdot\frac{n(n+1)}{2}$, donde $a,d$ son constantes.
    \item \textbf{Geométrica}: $\displaystyle\sum_{k=1}^{n}r^{k}=r\cdot\dfrac{1-r^n}{1-r}$, si $r\neq1$.
    \item \textbf{Telescópica}:  $\displaystyle\sum_{k=m}^{n}(a_{k}-a_{k+1})=a_{m}-a_{n+1},$ con $ 1 \leq m\leq n$
\end{enumerate}
\end{block}
\end{minipage}\pause
\hspace{0.4cm}\begin{minipage}[t]{0.5\linewidth}
\justifying
\vspace{-0.8cm}
\begin{block}{Propiedades}
\justifying
\begin{enumerate}[<+->]
\item \textbf{Linealidad}: $\displaystyle\sum_{k=1}^{n}(a_{k}+b_{k})=\displaystyle\sum_{k=1}^{n}a_{k}+\displaystyle\sum_{k=1}^{n}b_{k}$
\item \textbf{Asociatividad}: $\displaystyle\sum_{k=1}^{n}a_{k}=\displaystyle\sum_{k=1}^{m}a_{k}+\displaystyle\sum_{k=m+1}^{n}a_{k} \hspace{0,1 cm}, \ \mbox{con} \ 1 \leq m < n$.
\item \textbf{Factorización}: $\displaystyle\sum_{k=1}^{n}ca_{k}=c\displaystyle\sum_{k=1}^{n}a_{k}, \ \mbox{con} \ c \ \mbox{constante real}.$
\item \textbf{Cambio de índice}: $\displaystyle\sum_{k=m}^{n}a_{k}=\displaystyle\sum_{k=m-l}^{n-l}a_{k+l}=\displaystyle\sum_{k=m+l}^{n+l}a_{k-l}, \ \mbox{con} \ l \leq m<n$.  

\end{enumerate}
\vspace{-0.2cm}
\end{block}
\end{minipage}

\end{frame}

%---------------------------------------------------------------------
\subsection{Ejemplos}
\begin{frame}
  \frametitle{\insertsectionhead}
  \framesubtitle{\insertsubsectionhead}
  
  \justifying
\footnotesize
  
\begin{minipage}[t]{0.5\linewidth}
\vspace{-0.4cm}
\begin{exampleblock}{Ejemplo 1}
\justifying
Calcule: $\displaystyle\sum_{k=11}^{20}(2-k)$
\end{exampleblock}\pause

\begin{exampleblock}{Ejemplo 2}
\justifying
Pruebe que: $\displaystyle\sum_{k=3}^{20}\ln\left(\frac{k}{k+1}\right)=\ln(3)-\ln(21)$.
\end{exampleblock}\pause

\begin{exampleblock}{Ejemplo 3}
\justifying
Demuestre la fórmula de suma geométrica generalizada: $\displaystyle\sum_{k=m}^{n}r^k=\frac{r^m-r^{k+1}}{1-r}$, $r\neq1$.
\end{exampleblock}
\end{minipage}\pause
\hspace{0.4cm}\begin{minipage}[t]{0.5\linewidth}
\justifying
\vspace{-0.7cm}
\begin{exampleblock}{Ejemplo 4}
\justifying
Calcular $\displaystyle\sum_{k=1}^{100}a_{k}$, donde $\{a_{k}\}_{k\in\mathbb{N}}$ es la sucesi\'on definida por:
$$a_{k} = \; \left\{\begin{array}{ccl}
\displaystyle\frac{2^{k}+(-2)^{k}}{3^{k+1}} & \mbox{si}& 1\leq k\leq 50\\
\\
\sqrt{k+2}-\sqrt{k} & \mbox{si}& k>50 \\
\end{array}\right.$$
\end{exampleblock}\pause

\begin{exampleblock}{Ejemplo 5}
\justifying
Considere la sucesión $\{a_n\}_{n \in \mathbb{N}}$, definida por $a_n=n^2+n$.  Escriba la suma  $a_{2}+a_{4}+a_{6}+\cdots+a_{200}$ \ usando el símbolo de sumatoria $\sum$ y calcule el valor de dicha suma.
\end{exampleblock}
\end{minipage}

\end{frame}

%---------------------------------------------------------------------
%\subsection{Sumatorias}
\begin{frame}
  \frametitle{\insertsectionhead}
  \framesubtitle{\insertsubsectionhead}
  
  \justifying
\footnotesize
  
\begin{minipage}[t]{0.5\linewidth}
\begin{exampleblock}{Ejemplo 6}
\justifying
A una paciente se le administran 10 unidades de cierta medicina diariamente. El organismo, después de 24 horas, elimina el $20\%$ de la
medicina que se tiene en el cuerpo cada día. La droga es peligrosa para el organismo si se acumulan muchas unidades en el cuerpo; de hecho, es mortal si en el cuerpo hay 60 unidades de medicina o más. Determine si es posible administar indefinidamente la medicina a la paciente
sin correr riesgo de muerte
\end{exampleblock}
\end{minipage}

\end{frame}

% %==============================================
% \section{Conclusión}
% %==============================================

% \begin{frame}
%   \frametitle{\insertsectionhead}
%   \framesubtitle{\insertsubsectionhead}
  
%   \justifying
  
%   \begin{itemize}[<+->]
%   \justifying
%       \item Las sumatorias son adiciones de elementos de una sucesión generada por una patrón o fórmula $f(k)=a_k$. Su símbolo $\sum$ es una forma abreviada de escribir dicha suma.
%   \end{itemize}

% \end{frame}

% % %==============================================
% \section{Asistencia}
% %==============================================

% \begin{frame}
%   \frametitle{\insertsectionhead}
%   \framesubtitle{\insertsubsectionhead}
  
%   \justifying
  
%   \textbf{Solo se considerará la asistencia hasta 5 minutos después de terminada la clase.}
  
%   \begin{center}
%   \includegraphics[scale=0.23]{QR}
%   \end{center}
  
% \end{frame}


\end{document}