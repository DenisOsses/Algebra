\documentclass[handout,aspectratio=169]{beamer}
\usepackage{graphicx}
\usepackage{ragged2e}
\usefonttheme{professionalfonts}
\usepackage[spanish]{babel}
\usepackage{advdate}
\setbeamercovered{transparent}

% Document metadata
\title{Clase 9}
\subtitle{Principio de inducción. Ejemplos}
\author{Profesor: Denis Osses}
%\date{\AdvanceDate[+1]\today}
\date{\AdvanceDate[+1]\today}

% Image for the title page (use includegraphics option to properly size/place it)
\titlegraphic{\includegraphics[height=\paperheight]{imagen5}}

\usetheme[sectionstyle=style2]{trigon}

% Define logos to use (comment if no logo)
\biglogo{logoFIC} % Used on titlepage only
%\smalllogo{gggg}% Used on top right corner of regular frames

% ------ If you want to change the theme default colors, do it here ------
%\definecolor{tPrim}{HTML}{00843B}   % Green
%\definecolor{tSec}{HTML}{289B38}    % Green light
%\definecolor{tAccent}{HTML}{F07F3C} % Orange


% ------ Packages and definitions used for this demo. Can be removed ------
\usepackage{appendixnumberbeamer} % To use \appendix command
\pdfstringdefDisableCommands{% Fix hyperref translate warning with \appendix
\def\translate#1{#1}%
}
\usepackage{pgf-pie} % For pie charts
\usepackage{caption} % For subfigures
\usepackage{subcaption} % For subfigures
\usepackage{xspace}
\newcommand{\themename}{\textbf{\textsc{Álgebra}}\xspace}
\usepackage[scale=2]{ccicons} % Icons for CC-BY-SA
\usepackage{booktabs} % Better tables


%==============================================================================
%                               BEGIN DOCUMENT
%==============================================================================
\begin{document}

%--------------------------------------
% Create title frame
\titleframe

%--------------------------------------
% Table of contents
\begin{frame}{Temario}
  \setbeamertemplate{section in toc}[sections numbered]
  \tableofcontents%[hideallsubsections]
\end{frame}

%==============================================
\section{Objetivos de hoy}
%==============================================
%\subsection{Charts}
\begin{frame}{\insertsectionhead}
  \framesubtitle{\insertsubsectionhead}
  
\justifying

\begin{itemize}[<+->]
    \item Aplicar el principio de inducción.
\end{itemize}

\end{frame}

%==============================================
\section{Contenidos}
%==============================================

%---------------------------------------------------------------------
\subsection{Principio de inducción}
\begin{frame}
  \frametitle{\insertsectionhead}
  \framesubtitle{\insertsubsectionhead}
  
  \justifying

 \begin{block}{Definición}
\textbf{Principio de Inducción}\\\pause
\vspace{0.1cm}
\justifying
Sea $\phi(n)$ una función proposicional. Entonces $$\big[\phi(1)\wedge\forall~n\in\mathbb{N}: \big(\phi(n)\Rightarrow\phi(n+1)\big)\big]\Rightarrow\forall~n\in\mathbb{N}:\phi(n).$$\pause
Este principio nos dice que para probar una proposición $\phi(n)$ para todo $n\in\mathbb{N}$, basta con demostrar dos cosas:\pause

\begin{enumerate}[<+->]
\justifying
    \item $\phi(1)$.
    \item $\forall~n\in\mathbb{N}: \phi(n)\Rightarrow\phi(n+1)$.
\end{enumerate}

\end{block}

\end{frame}

%---------------------------------------------------------------------
%\subsection{Principio de inducción}
\begin{frame}
  \frametitle{\insertsectionhead}
  \framesubtitle{\insertsubsectionhead}
  
  \justifying

\begin{exampleblock}{Ejemplo 1}
\justifying
Demuestre las siguientes proposiciones:\pause
\begin{enumerate}[<+->]
    \item $\forall~n \in \mathbb{N}: n(n+1)$ es par.
    \item $\forall~n \in \mathbb{N}: n^3-n$ es divisible por 6.
    \item $\forall~n\in\mathbb{N}: 1+2+3+\ldots+n=\dfrac{n(n+1)}{2}.$
    \item $\forall~n \in \mathbb{N}: 4^n\geq 3n+1$.
\end{enumerate}
\end{exampleblock}

\end{frame}

%---------------------------------------------------------------------
%\subsection{Principio de inducción}
\begin{frame}
  \frametitle{\insertsectionhead}
  \framesubtitle{\insertsubsectionhead}
  
  \justifying
  
\begin{exampleblock}{Ejemplo 2}
\justifying
Demuestre que $$\forall~n\in\mathbb{N}:1+8+27+\cdots+n^3=\frac{n^2(n+1)^2}{4}.$$
\end{exampleblock}\pause

\begin{exampleblock}{Ejemplo 3}
\justifying
Demuestre que si $a_1=1$ y $a_n=2a_{n-1}+2^{n-1}$ para $n\geq2$, entonces para todo $n\in\mathbb{N}$ se tiene que $a_n=n2^{n-1}$.
\end{exampleblock}

\end{frame}


%==============================================
\section{Conclusión}
%==============================================

\begin{frame}
  \frametitle{\insertsectionhead}
  \framesubtitle{\insertsubsectionhead}
  
  \justifying
  
  \begin{itemize}[<+->]
  \justifying
      \item El principio de inducción se utiliza solamente para proposiciones sobre $\mathbb{N}$, no $\mathbb{R}$.
      \item Utilizamos este principio principalmente para demostrar propiedades de divisibilidad, sumas, desigualdades (que involucren algún término natural) y fórmulas recursivas. 
  \end{itemize}

\end{frame}

%==============================================
\section{Asistencia}
%==============================================

\begin{frame}
  \frametitle{\insertsectionhead}
  \framesubtitle{\insertsubsectionhead}
  
  \justifying
  
  \textbf{Solo se considerará la asistencia hasta 5 minutos después de terminada la clase.}
  
  \begin{center}
  \includegraphics[scale=0.23]{QR}
  \end{center}
  
\end{frame}


\end{document}