\documentclass[aspectratio=169]{beamer}
\usepackage{graphicx}
\usepackage{ragged2e}
\usefonttheme{professionalfonts}
\usepackage[spanish]{babel}
\usepackage{advdate}
\setbeamercovered{transparent}

% Document metadata
\title{Clase 10}
\subtitle{Sumas Dobles y Productorias}
\author{Profesor: Denis Osses}
%\date{\AdvanceDate[+4]\today}
%\date{\today}
\date{9 de abril de 2025}

% Image for the title page (use includegraphics option to properly size/place it)
\titlegraphic{\includegraphics[height=\paperheight]{imagen5}}

\usetheme[sectionstyle=style2]{trigon}

% Define logos to use (comment if no logo)
\biglogo{logoFIC} % Used on titlepage only
%\smalllogo{gggg}% Used on top right corner of regular frames

% ------ If you want to change the theme default colors, do it here ------
%\definecolor{tPrim}{HTML}{00843B}   % Green
%\definecolor{tSec}{HTML}{289B38}    % Green light
%\definecolor{tAccent}{HTML}{F07F3C} % Orange


% ------ Packages and definitions used for this demo. Can be removed ------
\usepackage{appendixnumberbeamer} % To use \appendix command
\pdfstringdefDisableCommands{% Fix hyperref translate warning with \appendix
\def\translate#1{#1}%
}
\usepackage{pgf-pie} % For pie charts
\usepackage{caption} % For subfigures
\usepackage{subcaption} % For subfigures
\usepackage{xspace}
\newcommand{\themename}{\textbf{\textsc{Álgebra}}\xspace}
\usepackage[scale=2]{ccicons} % Icons for CC-BY-SA
\usepackage{booktabs} % Better tables


%==============================================================================
%                               BEGIN DOCUMENT
%==============================================================================
\begin{document}

%--------------------------------------
% Create title frame
\titleframe

%--------------------------------------
% Table of contents
\begin{frame}{Temario}
  \setbeamertemplate{section in toc}[sections numbered]
  \tableofcontents%[hideallsubsections]
\end{frame}

%==============================================
\section{Objetivos de hoy}
%==============================================
%\subsection{Charts}
\begin{frame}{\insertsectionhead}
  \framesubtitle{\insertsubsectionhead}
  
\justifying

\begin{itemize}[<+->]
    \item Definir las sumatorias dobles y calcular algunas de ellas.
    \item Calcular productos aplicando sus propiedades.
\end{itemize}

\end{frame}

%==============================================
\section{Contenidos}
%==============================================

%-------------------------------------------------------------------------
\subsection{Sumatorias Dobles}
\begin{frame}
  \frametitle{\insertsectionhead}
 \framesubtitle{\insertsubsectionhead}
  
  \justifying

\begin{minipage}[t]{0.5\linewidth}
\justifying
%\vspace{0.1cm}
\footnotesize
\vspace{-0.8cm}
\begin{exampleblock}{Ejemplo}
    \justifying
    Considere el siguiente arreglo rectangular de números: \pause 

\begin{center}
\vspace{-0.2cm}
$\begin{array}{cccccccc}
3 & 4 & 5& 6 & 7 & \cdots & 101 & 102\\
6 & 8 & 10 & 12 & 14 & \cdots & 202 & 204\\  
9 & 12 & 15 & 18 & 21 & \cdots & 303 & 306\\  
12 & 16 & 20 & 24 & 28 & \cdots & 404 & 408\\  
 \vdots & \vdots & \vdots & \vdots & \vdots &  & \vdots & \vdots\\  
 300 & 400  & 500 & 600 & 700 & \cdots & 10100 & 10200\\  
303 & 404 & 505 & 606 & 707 & \cdots & 10201 & 10302\\ 
\end{array}
$
\end{center} \pause 

\vspace{-0.3cm}
\begin{enumerate}[{(a)}]
\justifying
\item Escriba la suma de todos los n\'umeros de la cuarta fila del arreglo rectangular  anterior utilizando el s\'imbolo de sumatoria y calcule esta suma.

\item Escriba la suma de todos los n\'umeros que aparecen en el arreglo anterior usando sumatorias y calcule dicha suma.
\vspace{-0.1cm}
\end{enumerate}

\end{exampleblock}

\end{minipage}\pause
\hspace{0.4cm}
\begin{minipage}[t]{0.5\linewidth}
\justifying
\vspace{-1.3cm}
\footnotesize
\begin{block}{Definición}
    \justifying
    Considere la sucesión $\{F(i)\}_{i\in\mathbb{N}}$. Sabemos que 
    \vspace{-0.1cm}
    $$\sum_{i=1}^m F(i)=F(1)+F(2)+\cdots+F(m).$$ \pause Si cada término de la sucesión $\{F(i)\}_{i\in\mathbb{N}}$ es, a su vez, una sumatoria, es decir 
     \vspace{-0.2cm}
    $$F(i)=\sum_{j=1}^{g(i)}f(i,j)
     \vspace{-0.2cm}$$ \pause 
     \vspace{-0.2cm}
    Entonces 
    \vspace{-0.2cm}
    $$\displaystyle\sum_{i=1}^m \sum_{j=1}^{g(i)}f(i,j)=\sum_{i=1}^m F(i) \vspace{-0.2cm}$$ corresponde a la \textbf{suma doble} de los términos $f(i,j)$. \pause 
\end{block}

\textbf{Nota}: Estudiaremos los siguientes casos más simples: cuando $g(i)=n$ (constante) y $g(i)=i
$.

\end{minipage}


\end{frame}

%-------------------------------------------------------------------------
%\subsection{Sumatorias Dobles}
\begin{frame}
  \frametitle{\insertsectionhead}
 \framesubtitle{\insertsubsectionhead}
  
  \justifying

\begin{minipage}[t]{0.5\linewidth}
\justifying
\vspace{-0.6cm}
\footnotesize
\begin{block}{Suma Rectangular}
    \justifying
    \underline{Si $g(i)=n$}: En este caso tenemos la sumatoria doble $$\displaystyle\sum_{i=1}^m \sum_{j=1}^{n}f(i,j)$$ \pause y podemos interpretarla como la suma de los términos de un arreglo rectangular de $m$ filas y $n$ columnas: \pause 

\begin{center}
$\begin{array}{ccccc}
f(1,1) & f(1,2) & f(1,3)&  \cdots & f(1,n) \\
f(2,1) & f(2,2) & f(2,3)&  \cdots & f(2,n)\\  
\vdots & \vdots & \vdots &  & \vdots \\  
f(m,1) & f(m,2) & f(m,3)&  \cdots & f(m,n)
\end{array}$
\end{center}
donde $f(i,j)$ corresponde al término en la fila $i$ y columna $j$ ($i=1,2,\ldots,m$ y $j=1,2,\ldots,n$).

\end{block}

\end{minipage}\pause
\hspace{0.4cm}
\begin{minipage}[t]{0.5\linewidth}
\justifying
\vspace{-0.3cm}
\footnotesize
\begin{block}{Suma Triangular}
    \justifying
\underline{Si $g(i)=i$}: En este caso tenemos la sumatoria doble $$\displaystyle\sum_{i=1}^m \sum_{j=1}^{i}f(i,j)$$ \pause y podemos interpretarla como la suma de los términos de un arreglo triangular de $m$ filas y $m$ columnas: \pause 

\begin{center}
$\begin{array}{ccccc}
f(1,1) &  & &   &  \\
f(2,1) & f(2,2) & &   &\\  
\vdots & \vdots & \ddots &  & \\  
f(m,1) & f(m,2) & f(m,3)&  \cdots & f(m,m)
\end{array}$
\end{center}

\end{block}

\end{minipage}


\end{frame}

%-------------------------------------------------------------------------
%\subsection{Sumatorias Dobles}
\begin{frame}
  \frametitle{\insertsectionhead}
 \framesubtitle{\insertsubsectionhead}
  
  \justifying

\begin{minipage}[t]{0.5\linewidth}
\justifying
\vspace{-0.1cm}
\footnotesize
\begin{exampleblock}{Ejemplo}
    \justifying
    Usando sumas dobles exprese y calcule la suma de todos los números del siguiente arreglo triangular:

\begin{center}
$\begin{array}{cccccccc}
1 &  & & &  &  &  & \\
1 & 2 &  &  & &  &&\\  
1& 2 & 3 & &  &  &  & \\  
1 & 2& 3 & 4 &  &  & & \\  
 \vdots & \vdots & \vdots & \vdots & \ddots &  &  & \\  
 1 & 2  & 3 & 4 & \cdots & n-1 &  & \\  
1 & 2 & 3 & 4 & 5 & \cdots & n & \\ 
\end{array}
$
\end{center}
\end{exampleblock}

\end{minipage}\pause
\hspace{0.4cm}
\begin{minipage}[t]{0.5\linewidth}
\justifying
\vspace{-1cm}
\footnotesize
\begin{exampleblock}{\color{red}{Propuesto}}
\justifying
Considere el siguiente arreglo cuadrado

\begin{center}
$\begin{array}{ccccc}
a_{11} & a_{12} & a_{13 }&  \cdots & a_{1n} \\
a_{21} & a_{22} & a_{23} & \cdots & a_{2n}\\  
\vdots & \vdots & \vdots &  & \vdots \\  
a_{n1} & a_{n2}  & a_{n3} &  \cdots & a_{nn}
\end{array}
$
\end{center}

Diremos que $a_{ij}$ está en el triángulo superior del arreglo si y solo si $i<j$. Análogamente, diremos que $a_{ij}$ está en el triángulo inferior del arreglo si y solo si $i>j$. 

\begin{enumerate}[{(a)}]
\justifying
\item Usando sumas dobles exprese la suma de todos los términos del arreglo cuadrado que están en el triángulo superior del arreglo.
\item Usando sumas dobles exprese la suma de todos los términos del arreglo cuadrado que están en el triángulo inferior del arreglo.
\end{enumerate} 
\vspace{-0.2cm}
\end{exampleblock}

\end{minipage}

\end{frame}

%-------------------------------------------------------------------------
\subsection{Productorias}
\begin{frame}
  \frametitle{\insertsectionhead}
 \framesubtitle{\insertsubsectionhead}
  
  \justifying

\begin{minipage}[t]{0.5\linewidth}
\justifying
\vspace{-0.1cm}
\footnotesize
\begin{block}{Definición}\pause
\justifying
Considere la sucesión $\{f(k)\}_{k\in\mathbb{N}}=\{a_k\}_{k\in\mathbb{N}}$. El producto de los $n$ primeros términos de esta sucesión se denomina \textbf{productoria} de tales términos. \pause Anotamos $$\prod_{k=1}^n a_k=a_1\cdot a_2\cdots a_n.$$ \pause Notamos que $$\prod_{k=1}^1 a_k=a_1 \pause ~~,~~\prod_{k=1}^{n+1} a_k=\left(\prod_{k=1}^{n}a_k\right)\cdot a_{n+1}.$$ 

\end{block}

\end{minipage}\pause
\hspace{0.4cm}
\begin{minipage}[t]{0.5\linewidth}
\justifying
\vspace{-0.4cm}
\footnotesize
\begin{block}{Propiedades}\pause
Si $\{a_{k}\}_{k\in\mathbb{N}}$ y $\{b_{k}\}_{k\in\mathbb{N}}$ son dos sucesiones reales, entonces: \pause
\justifying
\begin{enumerate}[<+->]
    \item \textbf{Productoria del producto}: $\displaystyle\prod_{k=1}^{n}(a_{k}\cdot b_{k})=\left(\displaystyle\prod_{k=1}^{n}a_{k}\right)\left(\displaystyle\prod_{k=1}^{n}b_{k}\right)$
    \item \textbf{Asociatividad}: $\displaystyle\prod_{k=1}^{n}a_{k}=\left(\displaystyle\prod_{k=1}^{m}a_{k}\right)\displaystyle\left(\prod_{k=m+1}^{n}a_{k}\right)$\hspace{0,1 cm},~con $1\leq m < n$.
    \item \textbf{Factorización}: $\displaystyle\prod_{k=1}^{n}ca_{k}=c^{n}\displaystyle\prod_{k=1}^{n}a_{k}~~$,~con $c\in\mathbb{R}$.
\end{enumerate}
\end{block}

\end{minipage}

\end{frame}

%-------------------------------------------------------------------------
%\subsection{Productorias}
\begin{frame}
  \frametitle{\insertsectionhead}
 \framesubtitle{\insertsubsectionhead}
  
  \justifying

\begin{minipage}[t]{0.5\linewidth}
\justifying
\vspace{-0.6cm}
\footnotesize
\begin{block}{Propiedades}\pause
\justifying
\begin{enumerate}[<+->]
    \item[4.] \textbf{Cambio de índice}: $\displaystyle\prod_{k=m-l}^{n-l}a_{k+l}=\displaystyle\prod_{k=m}^{n}a_{k}=\displaystyle\prod_{k=m+l}^{n+l}a_{k-l}~~$\hspace{0,1 cm},  con $l\leq m<n$.
    \item[5.] \textbf{Cancelación}: $ \displaystyle\prod_{k=m}^{n}\frac{a_{k}}{a_{k+1}}=\frac{a_{m}}{a_{n+1}}~~$, con  $1\leq m\leq n$.
\end{enumerate}
\end{block}\pause 

\begin{block}{Nota}
\justifying
Definimos el \textbf{factorial} de $n$ como: $~~~~n!=1\cdot 2\cdot3\cdots n=\displaystyle\prod_{k=1}^n k~~,~~0!=1$
\end{block}

\end{minipage}\pause
\hspace{0.4cm}
\begin{minipage}[t]{0.5\linewidth}
\justifying
\vspace{-0.4cm}
\footnotesize
\begin{exampleblock}{Ejemplo}
\justifying
 Calcular $\displaystyle \prod_{k=1}^{n}\frac{3k2^{k}}{5^{2k}(2k+2)}$
\end{exampleblock}\pause 

\begin{exampleblock}{Ejemplo}
\justifying
Calcule $\displaystyle\prod_{k=1}^{n}ka^{b+k}$ en términos de $a,b$ y $n$.
\end{exampleblock}\pause

\begin{exampleblock}{\color{red}{Propuesto}}
\justifying
Verifique que  $\displaystyle \prod_{n=1}^{10}\sum_{i=1}^{n}(1+i)=\dfrac{10!\cdot13!}{2^{10}\cdot3!}$
\end{exampleblock}


\end{minipage}

\end{frame}


% %---------------------------------------------------------------------
% %\subsection{Propiedades}
% \begin{frame}
%   \frametitle{\insertsectionhead}
%   \framesubtitle{\insertsubsectionhead}
  
%   \justifying

% \begin{exampleblock}{Ejemplo 4}
% \justifying
%  Calcular $\displaystyle \sum_{i=0}^{100}\prod_{j=0}^i 3\left(\frac{2^j}{2^{j+1}}\right)$
% \end{exampleblock}

% \end{frame}

% %==============================================
% \section{Conclusión}
% %==============================================

% \begin{frame}
%   \frametitle{\insertsectionhead}
%   \framesubtitle{\insertsubsectionhead}
  
%   \justifying
  
%   \begin{itemize}[<+->]
%   \justifying
%       \item Las productorias son multiplicaciones de elementos de una sucesión generada por una patrón o fórmula $f(k)=a_k$. Su símbolo $\prod$ es una forma abreviada de escribir dicho producto. No debe confundir con la sumatoria ni sus propiedades.
%   \end{itemize}

% \end{frame}

% %==============================================
% \section{Asistencia}
% %==============================================

% \begin{frame}
%   \frametitle{\insertsectionhead}
%   \framesubtitle{\insertsubsectionhead}
  
%   \justifying
  
%   \textbf{Solo se considerará la asistencia hasta 5 minutos después de terminada la clase.}
  
%   \begin{center}
%   \includegraphics[scale=0.23]{QR}
%   \end{center}
  
% \end{frame}


\end{document}