\documentclass[aspectratio=169]{beamer}
\usepackage{graphicx}
\usepackage{ragged2e}
\usefonttheme{professionalfonts}
\usepackage[spanish]{babel}
\usepackage{advdate}
\setbeamercovered{transparent}

% Document metadata
\title{Clase 11}
\subtitle{Teorema del binomio}
\author{Profesor: Denis Osses}
%\date{12 y \AdvanceDate[+3]\today}
%\date{\today}
\date{16 de abril de 2025}

% Image for the title page (use includegraphics option to properly size/place it)
\titlegraphic{\includegraphics[height=\paperheight]{imagen5}}

\usetheme[sectionstyle=style2]{trigon}

% Define logos to use (comment if no logo)
\biglogo{logoFIC} % Used on titlepage only
%\smalllogo{gggg}% Used on top right corner of regular frames

% ------ If you want to change the theme default colors, do it here ------
%\definecolor{tPrim}{HTML}{00843B}   % Green
%\definecolor{tSec}{HTML}{289B38}    % Green light
%\definecolor{tAccent}{HTML}{F07F3C} % Orange


% ------ Packages and definitions used for this demo. Can be removed ------
\usepackage{appendixnumberbeamer} % To use \appendix command
\pdfstringdefDisableCommands{% Fix hyperref translate warning with \appendix
\def\translate#1{#1}%
}
\usepackage{pgf-pie} % For pie charts
\usepackage{caption} % For subfigures
\usepackage{subcaption} % For subfigures
\usepackage{xspace}
\newcommand{\themename}{\textbf{\textsc{Álgebra}}\xspace}
\usepackage[scale=2]{ccicons} % Icons for CC-BY-SA
\usepackage{booktabs} % Better tables


%==============================================================================
%                               BEGIN DOCUMENT
%==============================================================================
\begin{document}

%--------------------------------------
% Create title frame
\titleframe

%--------------------------------------
% Table of contents
\begin{frame}{Temario}
  \setbeamertemplate{section in toc}[sections numbered]
  \tableofcontents%[hideallsubsections]
\end{frame}

%==============================================
\section{Objetivos de hoy}
%==============================================
%\subsection{Charts}
\begin{frame}{\insertsectionhead}
  \framesubtitle{\insertsubsectionhead}
  
\justifying

\begin{itemize}[<+->]
    \item Utilizar y demostrar propiedades de los coeficientes binomiales.
    \item Aplicar el Teorema del binomio.
\end{itemize}

\end{frame}

%==============================================
\section{Contenidos}
%==============================================

%---------------------------------------------------------------------
\subsection{Coeficientes binomiales}
\begin{frame}
  \frametitle{\insertsectionhead}
  \framesubtitle{\insertsubsectionhead}
  
  \justifying
  
Al desarrollar o expandir las potencias de binomio $(a+b)^n$, con $n\in\mathbb{N}_0$, ¿es posible obtener algún patrón numérico? \pause

\begin{eqnarray*}(a+b)^0 &=& 1\cdot a^0b^0 \pause\\ (a+b)^1 &=& 1\cdot a^1b^0+1\cdot a^0b^1 \pause \\ (a+b)^2 &=& 1\cdot a^2b^0+2\cdot a^1b^1+1\cdot a^0b^2 \pause \\ (a+b)^3 &=& 1\cdot a^3b^0+3\cdot a^2b^1+3\cdot a^1b^2+1\cdot a^0b^3 \pause \\ (a+b)^4 &=& 1\cdot a^4b^0+4\cdot a^3b^1+6\cdot a^2b^2+4\cdot a^1b^3+1\cdot a^0b^4\\  &\vdots&\end{eqnarray*}

\end{frame}

%---------------------------------------------------------------------
%\subsection{Coeficientes binomiales}
\begin{frame}
  \frametitle{\insertsectionhead}
  \framesubtitle{\insertsubsectionhead}
  
  \justifying

Los coeficientes numéricos de los términos $a^kb^{n-k}$, donde $k=0,1,\ldots,n$, corresponden a los números del \textbf{triángulo de Pascal} \pause

\begin{center}
    \includegraphics[scale=1]{pascal}
\end{center}

\end{frame}

%---------------------------------------------------------------------
%\subsection{Coeficientes binomiales}
\begin{frame}
  \frametitle{\insertsectionhead}
  \framesubtitle{\insertsubsectionhead}
  
  \justifying

Los coeficientes numéricos que aparecen en la expansión de las potencias del binomio \pause 

$$(a+b)^n=\underbrace{(a+b)(a+b)(a+b)\cdots(a+b)}_{n~\textrm{veces}}$$ \pause

consisten en todos los productos posibles que pueden ser formados tomando una letra ($a$ o $b$) de cada factor $(a+b)$. \pause El número de maneras en que se puede formar el producto $a^kb^{n-k}$ es exactamente igual al número de maneras de elegir $k$ factores (de los $n$ posibles) que contribuyen una $a$, ya que el resto de los factores contribuye una $b$. \pause El número de maneras de elegir $k$ factores entre $n$ es $C(n,k)=\displaystyle {n\choose k}=\frac{n!}{k!\cdot(n-k)!}$. \pause Así

\end{frame}

%---------------------------------------------------------------------
%\subsection{Coeficientes binomiales}
\begin{frame}
  \frametitle{\insertsectionhead}
  \framesubtitle{\insertsubsectionhead}
  
  \justifying
\begin{eqnarray*}(a+b)^0 &=& {0\choose 0}a^0b^0  \pause\\ (a+b)^1 &=& {1\choose 1} a^1b^0+{1\choose 0}a^0b^1 \pause \\ (a+b)^2 &=& {2\choose 2} a^2b^0+{2\choose 1} a^1b^1+{2\choose 0} a^0b^2 \pause \\ (a+b)^3 &=& {3\choose 3}a^3b^0+{3\choose 2} a^2b^1+{3\choose 1}a^1b^2+{3\choose 0} a^0b^3 \pause \\ (a+b)^4 &=& {4\choose 4} a^4b^0+{4\choose 3}a^3b^1+{4\choose 2} a^2b^2+{4\choose 1} a^1b^3+{4\choose 0} a^0b^4\\ &\vdots&\end{eqnarray*}

\end{frame}

%---------------------------------------------------------------------
\subsection{Teorema del binomio}
\begin{frame}
  \frametitle{\insertsectionhead}
  \framesubtitle{\insertsubsectionhead}
  
  \justifying
\footnotesize

\begin{minipage}[t]{0.5\linewidth}
\justifying

\begin{block}{Teorema del binomio}
\justifying
$\forall~n\in\mathbb{N}_0$ se cumple que $$(a+b)^n=\sum_{k=0}^n{n\choose k}a^kb^{n-k}.$$ 
\end{block}\pause

\begin{block}{Nota}
\justifying
Se tiene que para todo $k=0,1,2,\ldots, n$: $${n\choose n-k}={n\choose k}.$$
\end{block}
\end{minipage}\pause
\hspace{0.4cm}\begin{minipage}[t]{0.5\linewidth}
\justifying
Así, aprovechando esta propiedad simétrica de los coeficientes binomiales, podemos reescribir el Teorema del binomio como: \pause $$(a+b)^n=\sum_{k=0}^n{n\choose k}a^kb^{n-k}=\sum_{k=0}^n{n\choose k}a^{n-k}b^{k}.$$ \pause

\vspace{-0.3cm}
\begin{block}{Nota}
\justifying
Una propiedad importante de los coeficientes binomiales es que $${n+1\choose k+1}={n\choose k}+{n\choose k+1}~~,~~k=0,1,2,\ldots,n-1.$$ \pause Esta propiedad nos da el resultado de la suma de dos coeficientes consecutivos en el triángulo de Pascal.
\end{block}
\end{minipage}

\end{frame}

%---------------------------------------------------------------------
%\subsection{Teorema del binomio}
\begin{frame}
  \frametitle{\insertsectionhead}
  \framesubtitle{\insertsubsectionhead}
  
  \justifying
  \footnotesize

\begin{minipage}[t]{0.5\linewidth}
\justifying

\begin{exampleblock}{Ejemplo 1}
\justifying
Determine el coeficiente de $x^{10}$ en la expansión del binomio $(x-2)^{10}$.
\end{exampleblock}\pause 

\begin{exampleblock}{Ejemplo 2}
\justifying
Encuentre, si existe, el coeficiente que acompa\~{n}a a $x^{50}$ en el desarrollo de $\left(x^{2}-\displaystyle\frac{1}{x^{2}}\right)^{100}$.
\end{exampleblock}

\end{minipage}\pause
\hspace{0.4cm}\begin{minipage}[t]{0.5\linewidth}
\justifying

\begin{exampleblock}{Ejemplo 3}
\justifying
Calcular el valor de 
$$\displaystyle \sum_{k=1}^{701}\left(\begin{array}{c}
701 \\
k\end{array}\right)\frac{1}{7^{k-1}}.$$
\end{exampleblock}
\end{minipage}

\end{frame}

% %==============================================
% \section{Conclusión}
% %==============================================

% \begin{frame}
%   \frametitle{\insertsectionhead}
%   \framesubtitle{\insertsubsectionhead}
  
%   \justifying
  
%   \begin{itemize}[<+->]
%   \justifying
%       \item El teorema del binomio es la generalización del ``cuadrado de binomio".
%       \item El coeficiente binomial ${n\choose k}$ se puede entender como el número de subconjuntos con $k$ elementos escogidos de un conjunto con $n$ elementos.
%   \end{itemize}

% \end{frame}

% %==============================================
% \section{Asistencia}
% %==============================================

% \begin{frame}
%   \frametitle{\insertsectionhead}
%   \framesubtitle{\insertsubsectionhead}
  
%   \justifying
  
%   \textbf{Solo se considerará la asistencia hasta 5 minutos después de terminada la clase.}
  
%   \begin{center}
%   \includegraphics[scale=0.23]{QR}
%   \end{center}
  
% \end{frame}


\end{document}