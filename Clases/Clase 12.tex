\documentclass[aspectratio=169]{beamer}
\usepackage{graphicx}
\usepackage{ragged2e}
\usefonttheme{professionalfonts}
\usepackage[spanish]{babel}
\usepackage{advdate}
\setbeamercovered{transparent}
\usepackage{array} % for "\newcolumntype" macro
\newcolumntype{C}{>{$\displaystyle}c<{$}}
\newcommand\myfrac[2]{\frac{#1}{#2\mathstrut}}

% Document metadata
\title{Clase 12}
\subtitle{Razones trigonométricas en triángulos rectángulos}
\author{Profesor: Denis Osses}
% \date{\AdvanceDate[+2]\today}
%\date{\today}
\date{21 de abril de 2025}

% Image for the title page (use includegraphics option to properly size/place it)
\titlegraphic{\includegraphics[height=\paperheight]{imagen5}}

\usetheme[sectionstyle=style2]{trigon}

% Define logos to use (comment if no logo)
\biglogo{logoFIC} % Used on titlepage only
%\smalllogo{gggg}% Used on top right corner of regular frames

% ------ If you want to change the theme default colors, do it here ------
%\definecolor{tPrim}{HTML}{00843B}   % Green
%\definecolor{tSec}{HTML}{289B38}    % Green light
%\definecolor{tAccent}{HTML}{F07F3C} % Orange


% ------ Packages and definitions used for this demo. Can be removed ------
\usepackage{appendixnumberbeamer} % To use \appendix command
\pdfstringdefDisableCommands{% Fix hyperref translate warning with \appendix
\def\translate#1{#1}%
}
\usepackage{pgf-pie} % For pie charts
\usepackage{caption} % For subfigures
\usepackage{subcaption} % For subfigures
\usepackage{xspace}
\newcommand{\themename}{\textbf{\textsc{Álgebra}}\xspace}
\usepackage[scale=2]{ccicons} % Icons for CC-BY-SA
\usepackage{booktabs} % Better tables


%==============================================================================
%                               BEGIN DOCUMENT
%==============================================================================
\begin{document}

%--------------------------------------
% Create title frame
\titleframe

%--------------------------------------
% Table of contents
\begin{frame}{Temario}
  \setbeamertemplate{section in toc}[sections numbered]
  \tableofcontents%[hideallsubsections]
\end{frame}

%==============================================
\section{Objetivos de hoy}
%==============================================
%\subsection{Charts}
\begin{frame}{\insertsectionhead}
  \framesubtitle{\insertsubsectionhead}
  
\justifying

\begin{itemize}[<+->]
    \item Definir y calcular razones trigonométricas en triángulos rectángulos.
\end{itemize}

\end{frame}

%==============================================
\section{Contenidos}
%==============================================

%---------------------------------------------------------------------
\subsection{Razones trigonométricas}
\begin{frame}
  \frametitle{\insertsectionhead}
  \framesubtitle{\insertsubsectionhead}
  
  \justifying
  
  La idea fundamental de las razones trigonométricas es notar que en todo triángulo rectángulo ``extendido'', las proporciones entre sus lados se mantienen invariantes (constantes) independiente de la longitud de ellos (debido al Teorema de Thales) 
  
  \begin{columns}
        \begin{column}{0.5\textwidth}
            \includegraphics[scale=0.17]{trigon1}
        \end{column}\pause
        \begin{column}{0.5\textwidth}
            \begin{center}
                \begin{eqnarray*}
                {\scriptstyle
                \frac{\overline{AB}}{\overline{AC}}}&=& {\scriptstyle\frac{\overline{AB_1}}{\overline{AC_1}}= \frac{\overline{AB_2}}{\overline{AC_2}}= \frac{\overline{AB_3}}{\overline{AC_3}}=\cdots=\frac{\text{adyacente a}~\theta}{\text{hipotenusa}}}\\ \pause
                {\scriptstyle
                \frac{\overline{BC}}{\overline{AC}}}&=&
                {\scriptstyle\frac{\overline{B_1C_1}}{\overline{AC_1}}= \frac{\overline{B_2C_2}}{\overline{AC_2}}= \frac{\overline{B_3C_3}}{\overline{AC_3}}=\cdots=\frac{\text{opuesto a}~\theta}{\text{hipotenusa}}}
                \end{eqnarray*}
            \end{center}
    \end{column}
    \end{columns}

\end{frame}

%---------------------------------------------------------------------
%\subsection{Razones trigonométricas}
\begin{frame}
  \frametitle{\insertsectionhead}
  \framesubtitle{\insertsubsectionhead}
  
  \justifying
  
  Estas razones invariantes son conocidas como \textbf{trigonométricas} y se pueden definir 6: \pause
  
  \begin{block}{Definición}\pause
  \justifying
  Las \textbf{razones trigonométricas} asociadas al ángulo $\theta$ son el \textbf{seno}, \textbf{coseno}, \textbf{tangente}, \textbf{cosecante}, \textbf{secante} y \textbf{cotangente}, respectivamente: \pause
  
  \begin{center}
  \begin{tabular}{ccc}
      $\sen(\theta)=\frac{\text{opuesto a}~\theta}{\text{hipotenusa}}\hspace{1.5cm}$ \pause & $\cos(\theta)=\frac{\text{adyacente a}~\theta}{\text{hipotenusa}}\hspace{1.5cm}$ \pause & $\tan(\theta)=\frac{\text{opuesto a}~\theta}{\text{adyacente a}~\theta}$ \pause \\
      &&\\
      $\csc(\theta)=\frac{\text{hipotenusa}}{\text{opuesto a}~\theta}\hspace{1.5cm}$ \pause & $\sec(\theta)=\frac{\text{hipotenusa}}{\text{adyacente a}~\theta}\hspace{1.5cm}$ \pause & $\cot(\theta)=\frac{\text{adyacente a}~\theta}{\text{opuesto a}~\theta}$ 
  \end{tabular}
  \end{center}
  
  \end{block}
  
\end{frame}

%---------------------------------------------------------------------
%\subsection{Razones trigonométricas}
\begin{frame}
  \frametitle{\insertsectionhead}
  \framesubtitle{\insertsubsectionhead}
  
  \justifying

  \begin{block}{Nota}\pause
  \justifying
    Tenemos las siguientes identidades \pause $$\tan(\theta)=\frac{\sen(\theta)}{\cos( \theta)}~~ \pause ,~~\csc(\theta)=\frac{1}{\sen(\theta)}~~ \pause ,~~\sec(\theta)=\frac{1}{\cos(\theta)}~~ \pause ,~~\cot(\theta)=\frac{1}{\tan(\theta)}$$ \pause
    Además, si observamos el $\triangle ABC$ (rectángulo en $B$), entonces el $\measuredangle ACB=90^\circ-\theta$. \pause Así \pause 
    
    \begin{center}
    \begin{tabular}{ccc}
       $\sen(90^\circ-\theta)=\cos(\theta)$ \pause & , & $\cos(90^\circ-\theta)=\sen(\theta)$ \pause \\
       $\sec(90^\circ-\theta)=\csc(\theta)$  \pause & , & $\csc(90^\circ-\theta)=\sec(\theta)$ \pause \\
       $\tan(90^\circ-\theta)=\cot(\theta)$  \pause & , & $\cot(90^\circ-\theta)=\tan(\theta)$ \pause 
    \end{tabular}
    \end{center}
    
    de ahí se deriva el nombre \textbf{coseno}: ``\textbf{co}mplemento del \textbf{seno}''. \pause Análogamente con el resto.
    
  \end{block}
  
\end{frame}

%---------------------------------------------------------------------
\subsection{Ángulos notables}
\begin{frame}
  \frametitle{\insertsectionhead}
  \framesubtitle{\insertsubsectionhead}
  
  \justifying
  
  Observemos ahora los siguientes triángulos, el primero isóceles y el segundo equilátero:  \pause
 
  \vspace{0.2cm}
  
  \begin{columns}
        \begin{column}{0.2\textwidth}
            \includegraphics[scale=0.75]{trigon2}
        \end{column}
        \begin{column}{0.3\textwidth}
            \includegraphics[scale=0.5]{trigon3}
        \end{column}
    \end{columns} \pause
    
\vspace{0.2cm}
    
A partir de ellos podemos calcular las razones trigonométricas de los ángulos $30^\circ$, $45^\circ$ y $60^\circ$. \pause Además agregamos los ángulos ``extremos'' $0^\circ$ y $90^\circ$, para construir la siguiente tabla:

\end{frame}

%---------------------------------------------------------------------
%\subsection{Ángulos notables}
\begin{frame}
  \frametitle{\insertsectionhead}
  \framesubtitle{\insertsubsectionhead}
  
  \justifying
  
  \begin{center}
  \renewcommand\arraystretch{1.5}
      \begin{tabular}{|c||c|c|c|c|c|c|}
      \hline
          $\theta$ (en grados) & $\sen(\theta)$ & $\cos(\theta)$ & $\tan(\theta)$ & $\csc(\theta)$ & $\sec(\theta)$ & $\cot(\theta)$ \pause\\
          \hline 
          $0$ & $0$ & $1$ & $0$ & no existe & $1$ & no existe \pause\\
          \hline 
          $30$ & $\myfrac{1}{2}$ & $\myfrac{\sqrt{3}}{2}$ & $\myfrac{\sqrt{3}}{3}$ & $2$ & $\myfrac{2\sqrt{3}}{3}$ & $\sqrt{3}$ \pause\\
          \hline 
          $45$ & $\myfrac{\sqrt{2}}{2}$ & $\myfrac{\sqrt{2}}{2}$ & $1$ & $\sqrt{2}$ & $\sqrt{2}$ & $1$ \pause\\
          \hline 
          $60$ & $\myfrac{\sqrt{3}}{2}$ & $\myfrac{1}{2}$ & $\sqrt{3}$ & $\myfrac{2\sqrt{3}}{3}$ & $2$ & $\myfrac{\sqrt{3}}{3}$ \pause\\
          \hline
          $90$ & $1$ & $0$ & no existe & $1$ & no existe & $0$\\
          \hline
      \end{tabular}
  \end{center}

\end{frame}

%---------------------------------------------------------------------
\subsection{Teorema fundamental}
\begin{frame}
  \frametitle{\insertsectionhead}
  \framesubtitle{\insertsubsectionhead}

\justifying

\begin{minipage}[t]{0.5\linewidth}
\justifying
\vspace{-0.6cm}
\footnotesize
\begin{block}{Teorema}\pause 
  \justifying
  Consideremos un $\triangle ABC$ rectángulo. \pause Si $\theta$ es un ángulo interior del triángulo, entonces se cumple el \textbf{teorema fundamental de la trigonometría} \pause $$\sen^2(\theta)+\cos^2(\theta)=1$$
  \end{block}\pause
  
  \begin{block}{Nota}\pause 
  \justifying
    Este teorema es equivalente al teorema de Pitágoras. \pause Por otro lado, a partir de esta identidad podemos obtener nuevas identidades trigonométricas: \pause
    \vspace{0.3cm}
    \begin{center}
        \begin{tabular}{c}
             $\tan^2(\theta)+1=\sec^2(\theta)$ \pause \\
             $1+\cot^2(\theta)=\csc^2(\theta)$ 
        \end{tabular}
    \end{center}
  \end{block}
\end{minipage}\pause
\hspace{0.4cm}
\begin{minipage}[t]{0.5\linewidth}
\justifying
\vspace{-0.3cm}
\footnotesize
\begin{exampleblock}{Ejemplo 1}
  \justifying
  El ángulo de elevación de la azotea de una torre es $30^\circ$. Acercándose $30$ metros, el ángulo de elevación es $60^\circ$. Determine la altura de la torre.
  \end{exampleblock}\pause
  
  \begin{exampleblock}{Ejemplo 2}
  \justifying
  Resuelva la ecuación: $$\sen^2(\theta)-2\sen(\theta)\cos(\theta)+\cos^2(\theta)=0.$$
  \end{exampleblock}\pause
  
  \begin{alertblock}{\color{red}{Ejercicio propuesto}}
  \justifying
  Si $\tan(\theta)=\frac{4}{3}$, calcule $\sen(\theta)$, $\cos(\theta)$ y $\sec(\theta)$.
  \end{alertblock}
\end{minipage}

\end{frame}

% %==============================================
% \section{Conclusión}
% %==============================================

% \begin{frame}
%   \frametitle{\insertsectionhead}
%   \framesubtitle{\insertsubsectionhead}
  
%   \justifying
  
%   \begin{itemize}[<+->]
%   \justifying
%       \item Las razones trigonométricas fundamentales son el seno y el coseno, ya que todas las demás se pueden escribir en función de ellas.
%   \end{itemize}

% \end{frame}

% %==============================================
% \section{Asistencia}
% %==============================================

% \begin{frame}
%   \frametitle{\insertsectionhead}
%   \framesubtitle{\insertsubsectionhead}
  
%   \justifying
  
%   \textbf{Solo se considerará la asistencia hasta 5 minutos después de terminada la clase.}
  
%   \begin{center}
%   \includegraphics[scale=0.23]{QR}
%   \end{center}
  
% \end{frame}


\end{document}