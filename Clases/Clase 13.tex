\documentclass[handout,aspectratio=169]{beamer}
\usepackage{graphicx}
\usepackage{ragged2e}
\usefonttheme{professionalfonts}
\usepackage[spanish]{babel}
\usepackage{advdate}
\setbeamercovered{transparent}
\usepackage{array} % for "\newcolumntype" macro
\newcolumntype{C}{>{$\displaystyle}c<{$}}
\newcommand\myfrac[2]{\frac{#1}{#2\mathstrut}}

% Document metadata
\title{Clase 13}
\subtitle{Funciones trigonométricas}
\author{Profesor: Denis Osses}
%\date{\AdvanceDate[+2]\today}
%\date{\today}
\date{23 de abril de 2025}

% Image for the title page (use includegraphics option to properly size/place it)
\titlegraphic{\includegraphics[height=\paperheight]{imagen5}}

\usetheme[sectionstyle=style2]{trigon}

% Define logos to use (comment if no logo)
\biglogo{logoFIC} % Used on titlepage only
%\smalllogo{gggg}% Used on top right corner of regular frames

% ------ If you want to change the theme default colors, do it here ------
%\definecolor{tPrim}{HTML}{00843B}   % Green
%\definecolor{tSec}{HTML}{289B38}    % Green light
%\definecolor{tAccent}{HTML}{F07F3C} % Orange


% ------ Packages and definitions used for this demo. Can be removed ------
\usepackage{appendixnumberbeamer} % To use \appendix command
\pdfstringdefDisableCommands{% Fix hyperref translate warning with \appendix
\def\translate#1{#1}%
}
\usepackage{pgf-pie} % For pie charts
\usepackage{caption} % For subfigures
\usepackage{subcaption} % For subfigures
\usepackage{xspace}
\newcommand{\themename}{\textbf{\textsc{Álgebra}}\xspace}
\usepackage[scale=2]{ccicons} % Icons for CC-BY-SA
\usepackage{booktabs} % Better tables


%==============================================================================
%                               BEGIN DOCUMENT
%==============================================================================
\begin{document}

%--------------------------------------
% Create title frame
\titleframe

%--------------------------------------
% Table of contents
\begin{frame}{Temario}
  \setbeamertemplate{section in toc}[sections numbered]
  \tableofcontents%[hideallsubsections]
\end{frame}

%==============================================
\section{Objetivos de hoy}
%==============================================
%\subsection{Charts}
\begin{frame}{\insertsectionhead}
  \framesubtitle{\insertsubsectionhead}
  
\justifying

\begin{itemize}[<+->]
    \item Definir y estudiar las propiedades de las funciones trigonométricas.
\end{itemize}

\end{frame}

%==============================================
\section{Contenidos}
%==============================================

%---------------------------------------------------------------------
\subsection{Circunferencia goniométrica}
\begin{frame}
  \frametitle{\insertsectionhead}
  \framesubtitle{\insertsubsectionhead}
  
\justifying
Consideremos una circunferencia de radio 1, con centro en $(0,0)$ cuyo punto inicial es el $(1,0)$ 

\begin{center}
    \includegraphics[scale=0.6]{goniometrica.png}
\end{center}

\end{frame}

%---------------------------------------------------------------------
\subsection{Funciones trigonométricas}
\begin{frame}
  \frametitle{\insertsectionhead}
  \framesubtitle{\insertsubsectionhead}
  
\justifying
\footnotesize

\begin{minipage}[t]{0.5\linewidth}
\justifying
\vspace{-0.2cm}

Ir desde $(1,0)$ al punto $P$ es girar en sentido positivo. \pause El ángulo del sector circular formado por $(1,0)$, el origen y $P$ es $x$. \pause El punto $P$ tiene coordenadas $P(a_x,b_x)$ (dependen de $x$). \pause

\begin{block}{Definición}
\justifying
\begin{enumerate}[<+->]
\justifying
    \item  La función \textbf{seno} de $x$ se define como: $f(x)=\sen(x)=b_x$.
    \item  La función \textbf{coseno} de $x$ se define como: $f(x)=\cos(x)=a_x$.
    \item La función \textbf{tangente} de $x$ se define como: $f(x)=\tan(x)=\dfrac{b_x}{a_x}.$
\end{enumerate}
\end{block}
\end{minipage}\pause
\hspace{0.5cm}\begin{minipage}[t]{0.5\linewidth}
\justifying
\vspace{-0.2cm}
\begin{block}{Nota}
\justifying
Mediremos el ángulo $x$ en \textbf{radianes}. \pause La equivalencia entre radianes y grados (sexagesimales) está dada por la idea de que un giro completo a la circunferencia de radio 1 equivale a rotar en $360^\circ$ y la distancia recorrida en dicho giro es $2\pi$. \pause Así, tenemos $$2\pi~(\textrm{rad})=360^\circ~~\Leftrightarrow~~\pi~ (\textrm{rad})=180^\circ.$$
\end{block}\pause

\begin{exampleblock}{Ejemplo 1}
Determine en radianes el ángulo $330^\circ$.
\end{exampleblock}
\end{minipage}

\end{frame}


%---------------------------------------------------------------------
%\subsection{Funciones trigonométricas}
\begin{frame}
  \frametitle{\insertsectionhead}
  \framesubtitle{\insertsubsectionhead}
  
\justifying

  \begin{center}
  \renewcommand\arraystretch{1.5}
      \begin{tabular}{|c||c|c|c|c|c|c|}
      \hline
          $x$ (en radianes) & $\sen(x)$ & $\cos(x)$ & $\tan(x)$ & $\csc(x)$ & $\sec(x)$ & $\cot(x)$ \pause\\
          \hline 
          $0$ & $0$ & $1$ & $0$ & no existe & $1$ & no existe \pause\\
          \hline 
          $\myfrac{\pi}{6}$ & $\myfrac{1}{2}$ & $\myfrac{\sqrt{3}}{2}$ & $\myfrac{\sqrt{3}}{3}$ & $2$ & $\myfrac{2\sqrt{3}}{3}$ & $\sqrt{3}$ \pause\\
          \hline 
          $\myfrac{\pi}{4}$ & $\myfrac{\sqrt{2}}{2}$ & $\myfrac{\sqrt{2}}{2}$ & $1$ & $\sqrt{2}$ & $\sqrt{2}$ & $1$ \pause\\
          \hline 
          $\myfrac{\pi}{3}$ & $\myfrac{\sqrt{3}}{2}$ & $\myfrac{1}{2}$ & $\sqrt{3}$ & $\myfrac{2\sqrt{3}}{3}$ & $2$ & $\myfrac{\sqrt{3}}{3}$ \pause\\
          \hline
          $\myfrac{\pi}{2}$ & $1$ & $0$ & no existe & $1$ & no existe & $0$\\
          \hline
      \end{tabular}
  \end{center}\pause
  
\begin{exampleblock}{Ejemplo 2}
Determine el valor de $\cos\left(\dfrac{2\pi}{3}\right)$, $\cos\left(\pi\right)$, $\sen\left(\dfrac{5\pi}{4}\right)$, $\cos\left(-\dfrac{\pi}{3}\right)$, 
\end{exampleblock}

\end{frame}

%---------------------------------------------------------------------
\subsection{Estudio de las funciones trigonométricas}
\begin{frame}
  \frametitle{\insertsectionhead}
  \framesubtitle{\insertsubsectionhead}
  
\justifying
\footnotesize

\begin{minipage}[t]{0.5\linewidth}
\justifying
\vspace{-0.2cm}

\begin{block}{Función seno}
\justifying
Como $x$ puede tomar cualquier valor entonces $$dom(\sen(x))=\mathbb{R}.$$ \pause La segunda coordenada $b_x$ de $P$ solamente puede tomar valores entre $-1$ y $1$, ya que el punto $P$ está sobre la circunferencia. Luego $$rec(\sen(x))=[-1,1].$$ \pause
Además, el punto $P(a_x,b_x)$ permanece invariante si hacemos rotaciones en ángulos de valor $\pm2\pi$; es decir  $$P(a_x,b_x)=P(a_{x\pm 2\pi},b_{x\pm 2\pi}).$$
\end{block}
\end{minipage}\pause
\hspace{0.5cm}
\begin{minipage}[t]{0.5\linewidth}
\justifying
\vspace{-0.2cm}
\begin{block}
\justifying
A partir de esto, decimos que $\sen(x)$ es \textbf{periódica} (con periodo $2\pi$); es decir $$\sen(x\pm 2\pi)=\sen(x).$$ Con la periodicidad basta estudiar $\sen(x)$ para $x\in[0,2\pi]$. \pause

\vspace{0.2cm}

Otra propiedad importante es que $\sen(x)$ es \textbf{impar}, es decir $\sen(-x)=-b_x=-\sen(x)$. Con esto, basta con estudiar $\sen(x)$ para $x\in[0,\pi]$\pause

\vspace{0.2cm}

Para construir su gráfica, vemos la siguiente aplicación de \href{https://www.geogebra.org/m/tu3PFgUf}{Geogebra}
\end{block}

\end{minipage}

\end{frame}

%---------------------------------------------------------------------
%\subsection{Estudio de las funciones trigonométricas}
\begin{frame}
  \frametitle{\insertsectionhead}
  \framesubtitle{\insertsubsectionhead}
  
\justifying

\begin{block}
\justifying
Su gráfica es 
\begin{center}
    \includegraphics[scale=1]{seno}
\end{center}

\end{block}

\end{frame}

%---------------------------------------------------------------------
%\subsection{Estudio de las funciones trigonométricas}
\begin{frame}
  \frametitle{\insertsectionhead}
  \framesubtitle{\insertsubsectionhead}
  
\justifying
\footnotesize

\begin{minipage}[t]{0.5\linewidth}
\justifying
\vspace{-0.2cm}

\begin{block}{Función coseno}
\justifying
Como $x$ puede tomar cualquier valor entonces $$dom(\cos(x))=\mathbb{R}.$$ \pause La primera coordenada $a_x$ de $P$ solamente puede tomar valores entre $-1$ y $1$, ya que el punto $P$ está sobre la circunferencia. Luego $$rec(\cos(x))=[-1,1].$$ \pause
Además, el punto $P(a_x,b_x)$ permanece invariante si hacemos rotaciones en ángulos de valor $\pm2\pi$; es decir  $$P(a_x,b_x)=P(a_{x\pm 2\pi},b_{x\pm 2\pi}).$$
\end{block}
\end{minipage}\pause
\hspace{0.5cm}\begin{minipage}[t]{0.5\linewidth}
\justifying
\vspace{-0.2cm}
\begin{block}
\justifying
A partir de esto, decimos que $\cos(x)$ es \textbf{periódica} (con periodo $2\pi$); es decir $$\cos(x\pm 2\pi)=\cos(x).$$ Con la periodicidad basta estudiar $\cos(x)$ para $x\in[0,2\pi]$. \pause

\vspace{0.2cm}

Otra propiedad importante es que $\cos(x)$ es \textbf{par}, es decir $\cos(-x)=a_x=\cos(x)$. Con esto, basta con estudiar $\cos(x)$ para $x\in\left[-\frac{\pi}{2},\frac{\pi}{2}\right]$\pause

\vspace{0.2cm}

Para construir su gráfica, vemos la siguiente aplicación de \href{https://www.geogebra.org/m/tu3PFgUf}{Geogebra}
\end{block}

\end{minipage}

\end{frame}

%---------------------------------------------------------------------
%\subsection{Estudio de las funciones trigonométricas}
\begin{frame}
  \frametitle{\insertsectionhead}
  \framesubtitle{\insertsubsectionhead}
  
\justifying

\begin{block}
\justifying
Su gráfica es 
\begin{center}
    \includegraphics[scale=1]{coseno}
\end{center}

\end{block}

\end{frame}

%---------------------------------------------------------------------
%\subsection{Estudio de las funciones trigonométricas}
\begin{frame}
  \frametitle{\insertsectionhead}
  \framesubtitle{\insertsubsectionhead}
  
\justifying
\footnotesize

\begin{minipage}[t]{0.5\linewidth}
\justifying
\vspace{-0.2cm}

\begin{block}{Función tangente}
\justifying
Como $\tan(x)=\frac{\sen(x)}{\cos(x)}=\frac{b_x}{a_x}$ entonces $\cos(x)\neq0$, lo cual ocurre en los múltiplos impares de $\frac{\pi}{2}$, es decir  $$dom(\tan(x))=\mathbb{R}-\left\{(2k+1)\cdot\frac{\pi}{2}, k\in\mathbb{Z}\right\}.$$ \pause Ya que $-1\leq a_x,b_x\leq1$, el cuociente $\frac{b_x}{a_x}$ cubre todo el rango real. Luego $$rec(\tan(x))=\mathbb{R}.$$ \pause
Además, el cuociente $\frac{b_x}{a_x}$ permanece invariante en el primer y tercer cuadrantes, o en el cuarto y segundo cuadrantes de la circunferencia goniométrica, es decir, bajo rotaciones en ángulos de valor $\pm\pi$. 
\end{block}
\end{minipage}\pause
\hspace{0.5cm}\begin{minipage}[t]{0.5\linewidth}
\justifying
\vspace{-0.2cm}
\begin{block}
\justifying
A partir de esto, decimos que $\tan(x)$ es \textbf{periódica} (con periodo $\pi$); es decir $$\tan(x\pm \pi)=\tan(x).$$ Con la periodicidad basta estudiar $\tan(x)$ para $x\in\left]-\frac{\pi}{2},\frac{\pi}{2}\right[$. \pause

\vspace{0.2cm}

Otra propiedad importante es que $\tan(x)$ es \textbf{impar}, es decir $\tan(-x)=-\tan(x)$. \pause

\vspace{0.2cm}

Para construir su gráfica, vemos la siguiente aplicación de \href{https://www.geogebra.org/m/tu3PFgUf}{Geogebra}
\end{block}

\end{minipage}

\end{frame}

%---------------------------------------------------------------------
%\subsection{Estudio de las funciones trigonométricas}
\begin{frame}
  \frametitle{\insertsectionhead}
  \framesubtitle{\insertsubsectionhead}
  
\justifying

\begin{block}
\justifying
Su gráfica es 
\begin{center}
    \includegraphics[scale=0.5]{tangente}
\end{center}

\end{block}

\end{frame}

% %==============================================
% \section{Conclusión}
% %==============================================

% \begin{frame}
%   \frametitle{\insertsectionhead}
%   \framesubtitle{\insertsubsectionhead}
  
%   \justifying
  
%   \begin{itemize}[<+->]
%   \justifying
%       \item Las otras funciones trigonométricas: cosecante, secante y cotangente, se definen de modo análogo al de la clase anterior.
%       \item Todas las definiciones e identidades vistas en la clase anterior, son válidas para las funciones trigonométricas en general.
%   \end{itemize}

% \end{frame}

% %==============================================
% \section{Asistencia}
% %==============================================

% \begin{frame}
%   \frametitle{\insertsectionhead}
%   \framesubtitle{\insertsubsectionhead}
  
%   \justifying
  
%   \textbf{Solo se considerará la asistencia hasta 5 minutos después de terminada la clase.}
  
%   \begin{center}
%   \includegraphics[scale=0.23]{QR}
%   \end{center}
  
% \end{frame}


\end{document}