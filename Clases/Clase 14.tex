\documentclass[aspectratio=169]{beamer}
\usepackage{graphicx}
\usepackage{ragged2e}
\usefonttheme{professionalfonts}
\usepackage[spanish]{babel}
\usepackage{advdate}
\setbeamercovered{transparent}
\usepackage{array} % for "\newcolumntype" macro
\newcolumntype{C}{>{$\displaystyle}c<{$}}
\newcommand\myfrac[2]{\frac{#1}{#2\mathstrut}}

% Document metadata
\title{Clase 14}
\subtitle{Identidades trigonométricas}
\author{Profesor: Denis Osses}
%\date{\AdvanceDate[+1]\today}
%\date{\today}
\date{28 de abril de 2025}

% Image for the title page (use includegraphics option to properly size/place it)
\titlegraphic{\includegraphics[height=\paperheight]{imagen5}}

\usetheme[sectionstyle=style2]{trigon}

% Define logos to use (comment if no logo)
\biglogo{logoFIC} % Used on titlepage only
%\smalllogo{gggg}% Used on top right corner of regular frames

% ------ If you want to change the theme default colors, do it here ------
%\definecolor{tPrim}{HTML}{00843B}   % Green
%\definecolor{tSec}{HTML}{289B38}    % Green light
%\definecolor{tAccent}{HTML}{F07F3C} % Orange


% ------ Packages and definitions used for this demo. Can be removed ------
\usepackage{appendixnumberbeamer} % To use \appendix command
\pdfstringdefDisableCommands{% Fix hyperref translate warning with \appendix
\def\translate#1{#1}%
}
\usepackage{pgf-pie} % For pie charts
\usepackage{caption} % For subfigures
\usepackage{subcaption} % For subfigures
\usepackage{xspace}
\newcommand{\themename}{\textbf{\textsc{Álgebra}}\xspace}
\usepackage[scale=2]{ccicons} % Icons for CC-BY-SA
\usepackage{booktabs} % Better tables


%==============================================================================
%                               BEGIN DOCUMENT
%==============================================================================
\begin{document}

%--------------------------------------
% Create title frame
\titleframe

%--------------------------------------
% Table of contents
\begin{frame}{Temario}
  \setbeamertemplate{section in toc}[sections numbered]
  \tableofcontents%[hideallsubsections]
\end{frame}

%==============================================
\section{Objetivos de hoy}
%==============================================
%\subsection{Charts}
\begin{frame}{\insertsectionhead}
  \framesubtitle{\insertsubsectionhead}
  
\justifying

\begin{itemize}[<+->]
    \item Demostrar identidades trigonométricas
\end{itemize}

\end{frame}

%==============================================
\section{Contenidos}
%==============================================

%---------------------------------------------------------------------
\subsection{Suma y resta de ángulos. Ángulo Doble}
\begin{frame}
  \frametitle{\insertsectionhead}
  \framesubtitle{\insertsubsectionhead}
  
\justifying
\footnotesize

\begin{minipage}[t]{0.5\linewidth}
\justifying
\vspace{-0.2cm}

\begin{block}{Teorema}
\justifying
Si $\alpha$ es un ángulo cualquiera de un triángulo, entonces \pause $$\sen^2(\alpha)+\cos^2(\alpha)=1$$
\end{block}

\begin{block}{Teorema}
\justifying
Si $\alpha$ y $\beta$ son dos ángulos cualesquiera de un triángulo, entonces \pause
\vspace{0.2cm}
\begin{center}
    \begin{tabular}{ccc}
         $\sen(\alpha\pm\beta)$ & $=$ & $\sen(\alpha)\cos(\beta)\pm\sen(\beta)\cos(\alpha)$ \pause\\
         $\cos(\alpha\pm\beta)$ & $=$ & $\cos(\alpha)\cos(\beta)\mp\sen(\alpha)\sen(\beta)$ \pause\\
         $\tan(\alpha\pm\beta)$ & $=$ & $\dfrac{\tan(\alpha)\pm\tan(\beta)}{1\mp\tan(\alpha)\tan(\beta)}$
    \end{tabular}
\end{center}
\end{block}
\end{minipage}\pause
\hspace{0.4cm}\begin{minipage}[t]{0.5\linewidth}
\justifying
\vspace{-0.2cm}
\begin{block}{Teorema}
\justifying
Si $\alpha$ es un ángulo cualquiera de un triángulo, entonces \pause
\vspace{0.2cm}
\begin{center}
    \begin{tabular}{ccc}
         $\sen(2\alpha)$ & $=$ & $2\sen(\alpha)\cos(\alpha)$ \pause\\
         $\cos(2\alpha)$ & $=$ & $\cos^2(\alpha)-\sen^2(\alpha)$ \pause\\
         $\cos(2\alpha)$ & $=$ & $1-2\sen^2(\alpha)$ \pause\\
         $\cos(2\alpha)$ & $=$ & $2\cos^2(\alpha)-1$ \pause\\
         $\tan(2\alpha)$ & $=$ & $\dfrac{2\tan(\alpha)}{1-\tan^2(\alpha)}$
    \end{tabular}
\end{center}
\end{block}
\end{minipage}

\end{frame}

%---------------------------------------------------------------------
\subsection{Ángulo medio. Suma en producto}
\begin{frame}
  \frametitle{\insertsectionhead}
  \framesubtitle{\insertsubsectionhead}
  
\justifying
\footnotesize

\begin{minipage}[t]{0.45\linewidth}
\justifying
\vspace{-0.2cm}

\begin{block}{Teorema}
\justifying
Si $\alpha$ es un ángulo cualquiera de un triángulo, entonces \pause
\vspace{0.2cm}
\begin{center}
    \begin{tabular}{ccc}
         $\sen\left(\dfrac{\alpha}{2}\right)$ & $=$ & $\pm\sqrt{\dfrac{1-\cos(\alpha)}{2}}$ \pause\\
         $\cos\left(\dfrac{\alpha}{2}\right)$ & $=$ & $\pm\sqrt{\dfrac{1+\cos(\alpha)}{2}}$ \pause\\
         $\tan\left(\dfrac{\alpha}{2}\right)$ & $=$ & $\pm\sqrt{\dfrac{1-\cos(\alpha)}{1+\cos(\alpha)}}$ 
    \end{tabular}
\end{center}
\end{block}
\end{minipage}\pause
\hspace{0.4cm}\begin{minipage}[t]{0.55\linewidth}
\justifying
\vspace{-0.2cm}
\begin{block}{Teorema}
\justifying
Si $\alpha$ y $\beta$ son dos ángulos cualesquiera de un triángulo, entonces \pause
\begin{center}
    \begin{tabular}{ccc}
         $\sen(\alpha)+\sen(\beta)$ & $=$ & $2\sen\left(\dfrac{\alpha+\beta}{2}\right)\cos\left(\dfrac{\alpha-\beta}{2}\right)$ \pause\\
         $\sen(\alpha)-\sen(\beta)$ & $=$ & $2\cos\left(\dfrac{\alpha+\beta}{2}\right)\sen\left(\dfrac{\alpha-\beta}{2}\right)$ \pause\\
         $\cos(\alpha)+\cos(\beta)$ & $=$ & $2\cos\left(\dfrac{\alpha+\beta}{2}\right)\cos\left(\dfrac{\alpha-\beta}{2}\right)$ \pause\\
         $\cos(\alpha)-\cos(\beta)$ & $=$ & $-2\sen\left(\dfrac{\alpha+\beta}{2}\right)\sen\left(\dfrac{\alpha-\beta}{2}\right)$
    \end{tabular}
\end{center}
\end{block}
\end{minipage}

\end{frame}

%---------------------------------------------------------------------
\subsection{Producto en suma}
\begin{frame}
  \frametitle{\insertsectionhead}
  \framesubtitle{\insertsubsectionhead}
  
\justifying
\footnotesize

\begin{minipage}[t]{0.5\linewidth}
\justifying
\vspace{-0.2cm}

\begin{block}{Teorema}
\justifying
Si $\alpha$ y $\beta$ son dos ángulos cualesquiera de un triángulo, entonces \pause
\vspace{0.2cm}
\begin{center}
    \begin{tabular}{ccc}
         $\sen(\alpha)\cos(\beta)$ & $=$ & $\frac{1}{2}\big[\sen(\alpha+\beta)+\sen(\alpha-\beta)\big]$ \pause\\
         $\cos(\alpha)\cos(\beta)$ & $=$ & $\frac{1}{2}\big[\cos(\alpha+\beta)+\cos(\alpha-\beta)\big]$ \pause\\
         $\sen(\alpha)\sen(\beta)$ & $=$ & $\frac{1}{2}\big[\cos(\alpha-\beta)-\cos(\alpha+\beta)\big]$
    \end{tabular}
\end{center}
\end{block}
\end{minipage}\pause
\hspace{0.35cm}\begin{minipage}[t]{0.51\linewidth}
\justifying
\vspace{-0.2cm}

\begin{exampleblock}{Ejemplo}
\justifying
Probar las siguientes identidades: \pause
\vspace{0.2cm}
\begin{enumerate}[<+->]
    \item $2\cot^2(\alpha)+1=\csc^4(\alpha)-\cot^4(\alpha)$.
    \item $\left(\cos\left(\dfrac{\beta}{2}\right)-\sen\left(\dfrac{\beta}{2}\right)\right)^2=1-\sen(\beta)$
    \item Si $\alpha+\beta+\gamma=\pi$ entonces $$\hspace{-0.8cm}\sen(\alpha)+\sen(\beta)+\sen(\gamma)=4\cos\left(\dfrac{\alpha}{2}\right)\cos\left(\dfrac{\beta}{2}\right)\cos\left(\dfrac{\gamma}{2}\right)$$
\end{enumerate}
\end{exampleblock}
\end{minipage}

\end{frame}

% %==============================================
% \section{Conclusión}
% %==============================================

% \begin{frame}
%   \frametitle{\insertsectionhead}
%   \framesubtitle{\insertsubsectionhead}
  
%   \justifying
  
%   \begin{itemize}[<+->]
%   \justifying
%       \item Todas las definiciones e identidades vistas en clases anteriores, son válidas para las identidades trigonométricas generales.
%   \end{itemize}

% \end{frame}

% %==============================================
% \section{Asistencia}
% %==============================================

% \begin{frame}
%   \frametitle{\insertsectionhead}
%   \framesubtitle{\insertsubsectionhead}
  
%   \justifying
  
%   \textbf{Solo se considerará la asistencia hasta 5 minutos después de terminada la clase.}
  
%   \begin{center}
%   \includegraphics[scale=0.23]{QR}
%   \end{center}
  
% \end{frame}


\end{document}