\documentclass[aspectratio=169]{beamer}
\usepackage{graphicx}
\usepackage{ragged2e}
\usefonttheme{professionalfonts}
\usepackage[spanish]{babel}
\usepackage{advdate}
\setbeamercovered{transparent}
\usepackage{array} % for "\newcolumntype" macro
\newcolumntype{C}{>{$\displaystyle}c<{$}}
\newcommand\myfrac[2]{\frac{#1}{#2\mathstrut}}

% Document metadata
\title{Clase 16}
\subtitle{Teoremas del seno y del coseno}
\author{Profesor: Denis Osses}
%\date{\AdvanceDate[+2]\today}
%\date{\today}
\date{12 de mayo de 2025}

% Image for the title page (use includegraphics option to properly size/place it)
\titlegraphic{\includegraphics[height=\paperheight]{imagen5}}

\usetheme[sectionstyle=style2]{trigon}

% Define logos to use (comment if no logo)
\biglogo{logoFIC} % Used on titlepage only
%\smalllogo{gggg}% Used on top right corner of regular frames

% ------ If you want to change the theme default colors, do it here ------
%\definecolor{tPrim}{HTML}{00843B}   % Green
%\definecolor{tSec}{HTML}{289B38}    % Green light
%\definecolor{tAccent}{HTML}{F07F3C} % Orange


% ------ Packages and definitions used for this demo. Can be removed ------
\usepackage{appendixnumberbeamer} % To use \appendix command
\pdfstringdefDisableCommands{% Fix hyperref translate warning with \appendix
\def\translate#1{#1}%
}
\usepackage{pgf-pie} % For pie charts
\usepackage{caption} % For subfigures
\usepackage{subcaption} % For subfigures
\usepackage{xspace}
\newcommand{\themename}{\textbf{\textsc{Álgebra}}\xspace}
\usepackage[scale=2]{ccicons} % Icons for CC-BY-SA
\usepackage{booktabs} % Better tables


%==============================================================================
%                               BEGIN DOCUMENT
%==============================================================================
\begin{document}

%--------------------------------------
% Create title frame
\titleframe

%--------------------------------------
% Table of contents
\begin{frame}{Temario}
  \setbeamertemplate{section in toc}[sections numbered]
  \tableofcontents%[hideallsubsections]
\end{frame}

%==============================================
\section{Objetivos de hoy}
%==============================================
%\subsection{Charts}
\begin{frame}{\insertsectionhead}
  \framesubtitle{\insertsubsectionhead}
  
\justifying

\begin{itemize}[<+->]
    \item Utilizar los teoremas del seno y el coseno en la resolución de problemas.
\end{itemize}

\end{frame}

%==============================================
\section{Contenidos}
%==============================================

%---------------------------------------------------------------------
\subsection{Teoremas del seno y del coseno}
\begin{frame}
  \frametitle{\insertsectionhead}
  \framesubtitle{\insertsubsectionhead}
  
\justifying
\footnotesize

\begin{minipage}[t]{0.5\linewidth}
\justifying
\vspace{-0.5cm}
\begin{block}{Teorema del seno}
\justifying
Si en un $\triangle ABC$, las medidas de los lados opuestos a los ángulos $\alpha$, $\beta$, $\gamma$ son respectivamente $a$, $b$, $c$, entonces: $$\frac{a}{\sen(\alpha)}=\frac{b}{\sen(\beta)}=\frac{c}{\sen(\gamma)}$$
\end{block}\pause

\begin{center}
    \visible<2-4>{\includegraphics[scale=0.25]{TeoSen}}
\end{center}
\end{minipage}\pause
\hspace{0.4cm}\begin{minipage}[t]{0.5\linewidth}
\justifying
\vspace{-0.5cm}
\begin{block}{Teorema del coseno}
\justifying
Si en un $\triangle ABC$, las medidas de los lados opuestos a los ángulos $\alpha$, $\beta$, $\gamma$ son respectivamente $a$, $b$, $c$, entonces: 

\begin{center}
    \begin{tabular}{c}
        $a^2=b^2+c^2-2bc\cos(\alpha)$ \\
        $b^2=a^2+c^2-2ac\cos(\beta)$ \\
        $c^2=a^2+b^2-2ab\cos(\gamma)$
    \end{tabular}
\end{center}
\end{block}\pause 

\begin{center}
    \visible<4>{\includegraphics[scale=0.6]{LeyCoseno}}
\end{center}

\end{minipage}

\end{frame}

%---------------------------------------------------------------------
\subsection{Ejemplos}
\begin{frame}
  \frametitle{\insertsectionhead}
  \framesubtitle{\insertsubsectionhead}
  
\justifying
\footnotesize

\begin{minipage}[t]{0.5\linewidth}
\justifying
\vspace{-0.5cm}

\begin{exampleblock}{Ejemplo 1}
\justifying
Dos excursionistas salen de su campamento al mismo tiempo, con rumbos $42^\circ$ NO y $20^\circ$ SE, respectivamente. Si cada uno de ellos camina a un promedio de 5 [km/h], ¿a qué distancia están después de 1 hora?
\end{exampleblock}\pause

\begin{exampleblock}{Ejemplo 2}
\justifying
Calcular el ángulo agudo que forman entre sí las diagonales de un rectángulo, si las longitudes de sus lados son 2.2 [m] y 1.4 [m].
\end{exampleblock}

\end{minipage}\pause
\hspace{0.4cm}\begin{minipage}[t]{0.5\linewidth}
\justifying
\vspace{-0.5cm}
\begin{exampleblock}{Ejemplo 3}
\justifying
El ángulo de elevación de una torre CD desde un lugar A al Sur de ella es $30^\circ$ y desde un lugar B hacia el Oeste de A su ángulo de elevación es de $15^\circ$. Si $AB = a$ probar que la altura de la torre es $$\dfrac{a(2-\sqrt{3})}{\sqrt{12\sqrt{3}-20}}$$
\end{exampleblock}
\end{minipage}

\end{frame}


% %==============================================
% \section{Conclusión}
% %==============================================

% \begin{frame}
%   \frametitle{\insertsectionhead}
%   \framesubtitle{\insertsubsectionhead}
  
%   \justifying
  
%   \begin{itemize}[<+->]
%   \justifying
%       \item El teorema del seno es utilizado para resolver problemas en los que se conocen dos ángulos del triángulo y un lado opuesto a uno de ellos. También se usa cuando conocemos dos lados del triángulo y un ángulo opuesto a uno de ellos.
%       \item El teorema el coseno es una generalización del teorema de Pitágoras.
%   \end{itemize}

% \end{frame}

% %==============================================
% \section{Asistencia}
% %==============================================

% \begin{frame}
%   \frametitle{\insertsectionhead}
%   \framesubtitle{\insertsubsectionhead}
  
%   \justifying
  
%   \textbf{Solo se considerará la asistencia hasta 5 minutos después de terminada la clase.}
  
%   \begin{center}
%   \includegraphics[scale=0.23]{QR}
%   \end{center}
  
% \end{frame}


\end{document}