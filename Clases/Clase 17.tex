\documentclass[aspectratio=169]{beamer}
\usepackage{graphicx}
\usepackage{ragged2e}
\usefonttheme{professionalfonts}
\usepackage[spanish]{babel}
\usepackage{advdate}
\usepackage{multicol}
\setbeamercovered{transparent}
\usepackage{array} % for "\newcolumntype" macro
\newcolumntype{C}{>{$\displaystyle}c<{$}}
\newcommand\myfrac[2]{\frac{#1}{#2\mathstrut}}

% Document metadata
\title{Clase 17}
\subtitle{Vectores}
\author{Profesor: Denis Osses}
%\date{\AdvanceDate[+6]\today}
%\date{\today}
\date{14 de mayo de 2025}

% Image for the title page (use includegraphics option to properly size/place it)
\titlegraphic{\includegraphics[height=\paperheight]{imagen5}}

\usetheme[sectionstyle=style2]{trigon}

% Define logos to use (comment if no logo)
\biglogo{logoFIC} % Used on titlepage only
%\smalllogo{gggg}% Used on top right corner of regular frames

% ------ If you want to change the theme default colors, do it here ------
%\definecolor{tPrim}{HTML}{00843B}   % Green
%\definecolor{tSec}{HTML}{289B38}    % Green light
%\definecolor{tAccent}{HTML}{F07F3C} % Orange


% ------ Packages and definitions used for this demo. Can be removed ------
\usepackage{appendixnumberbeamer} % To use \appendix command
\pdfstringdefDisableCommands{% Fix hyperref translate warning with \appendix
\def\translate#1{#1}%
}
\usepackage{pgf-pie} % For pie charts
\usepackage{caption} % For subfigures
\usepackage{subcaption} % For subfigures
\usepackage{xspace}
\newcommand{\themename}{\textbf{\textsc{Álgebra}}\xspace}
\usepackage[scale=2]{ccicons} % Icons for CC-BY-SA
\usepackage{booktabs} % Better tables


%==============================================================================
%                               BEGIN DOCUMENT
%==============================================================================
\begin{document}

%--------------------------------------
% Create title frame
\titleframe

%--------------------------------------
% Table of contents
\begin{frame}{Temario}
  \setbeamertemplate{section in toc}[sections numbered]
  \tableofcontents%[hideallsubsections]
\end{frame}

%==============================================
\section{Objetivos de hoy}
%==============================================
%\subsection{Charts}
\begin{frame}{\insertsectionhead}
  \framesubtitle{\insertsubsectionhead}
  
\justifying

\begin{itemize}[<+->]
    \item Representar geométricamente vectores y sus operaciones. 
    \item Aplicar las propiedades fundamentales de las operaciones de vectores para la resolución de problemas.
\end{itemize}

\end{frame}

%==============================================
\section{Contenidos}
%==============================================

%---------------------------------------------------------------------
\subsection{Segmento dirigido}
\begin{frame}
  \frametitle{\insertsectionhead}
  \framesubtitle{\insertsubsectionhead}
  
\justifying

\footnotesize

\begin{minipage}[t]{0.5\linewidth}
    \begin{center}
    \includegraphics[scale=0.5]{vector1}
\end{center}\pause

\begin{block}{Definición}\pause
\justifying
Dados dos puntos $A$ y $B$ en el plano o en el espacio denotamos por $(A,B)$ al \textbf{segmento dirigido} con punto inicial $A$ y punto final $B$.
\end{block}
\end{minipage}\pause 
\hspace{0.4cm}\begin{minipage}[t]{0.5\linewidth}
\vspace{-2.5cm}
El segmento dirigido anterior tiene una \textbf{dirección}, \pause que está dada por la recta que determinan los puntos $A$ y $B$, \pause tiene un \textbf{sentido} dado, \pause que identifica en ellos un punto inicial y un punto final. \pause Podemos observar que $(A,B)\neq (B,A)$ (puesto que tienen sentidos contrarios). \pause Los segmentos dirigidos tienen también asociado un valor numérico o \textbf{magnitud}, \pause dado por la distancia entre los puntos $A$ y $B$. \pause 

\vspace{0.1cm}
El concepto de vector es algo un poco más general que el de segmento dirigido, \pause en relación a que ``\textbf{dos segmentos dirigidos que coincidan en dirección, sentido y magnitud, representan el mismo vector}". \pause Es fundamental comprender esta idea en el análisis vectorial.
\end{minipage}

\end{frame}


%---------------------------------------------------------------------
\subsection{Vectores}
\begin{frame}
  \frametitle{\insertsectionhead}
  \framesubtitle{\insertsubsectionhead}
  
\justifying
\footnotesize

\begin{minipage}[t]{0.5\linewidth}
    Lo anterior se formaliza matemáticamente considerando el conjunto de \underline{todos} los segmentos dirigidos del plano o el espacio y estableciendo la siguiente relación $\uparrow$ entre ellos: \pause $$(A,B)\uparrow (C,D)$$ si los segmentos dirigidos $(A,B)$ y $(C,D)$ tienen igual magnitud, dirección y sentido. \pause

\vspace{0.1cm}

\begin{block}{Definición}\pause
\justifying El \textbf{vector} $\vec{p}$ cuyo punto inicial es $O$ (origen) y final es $P$, corresponde al ``\textbf{representante}" de todos los segmentos dirigidos $(A,B)$ que tienen la misma dirección, sentido y magnitud que el segmento dirigido $(O,P)$, \pause o sea: $$\vec{p}=[(O,P)]=\{[A,B]:(A,B)\uparrow(O,P)\}.$$
\end{block}
\end{minipage}\pause 
\hspace{0.3cm}\begin{minipage}[t]{0.5\linewidth}
Todos los vectores del siguiente esquema son idénticos ya que tienen igual magnitud, dirección y sentido. El vector $\vec{p}$ es el representante de todos ellos.

\begin{center}
\visible<7>{\includegraphics[scale=0.45]{vector2}}
\end{center}
\end{minipage}

\end{frame}


%---------------------------------------------------------------------
\subsection{Suma de vectores}
\begin{frame}
  \frametitle{\insertsectionhead}
  \framesubtitle{\insertsubsectionhead}
  
\justifying

\footnotesize

\begin{minipage}[t]{0.5\linewidth}
\begin{block}{Nota 1}\pause
\justifying
El \textbf{vector nulo} $\vec{0}$ corresponde al representante de todos los segmentos dirigidos que comienzan en el punto $A$ y terminan en $A$.
\end{block}\pause

\begin{block}{Nota 2}\pause
\justifying
El segmento dirigido que comienza en el origen $O$ y termina en el punto $P$, es decir el vector $\vec{p}$ suele escribirse como $$\vec{p}=\overrightarrow{OP}$$ y se denomina \textbf{vector posición} del punto $P$.
\end{block}
\end{minipage}\pause 
\hspace{0.5cm}\begin{minipage}[t]{0.5\linewidth}
\begin{block}{Definición}
\justifying
Consideremos dos vectores $\vec{a}$ y $\vec{b}$. El \textbf{vector suma o adición geométrica}, $\vec{a}+\vec{b}$, es aquel que resulta al unir el origen con el vértice del paralelógramo formado por $\vec{a}$ y $\vec{b}$.
\end{block}

\begin{center}
  \visible<6>{\includegraphics[scale=0.5]{sumavector}} 
\end{center}
\end{minipage}

\end{frame}

%---------------------------------------------------------------------
\subsection{Ponderación de vectores y propiedades}
\begin{frame}
  \frametitle{\insertsectionhead}
  \framesubtitle{\insertsubsectionhead}
  
\justifying

\footnotesize

\begin{minipage}[t]{0.5\linewidth}
    \begin{block}{Definición}
\justifying
La \textbf{ponderación} del vector $\vec{a}$ por el escalar (número real) $\alpha$ es definida como el nuevo vector $\alpha\vec{a}$ que tiene las siguientes características: si $\alpha>0$ entonces $\alpha\vec{a}$ mantiene el sentido del vector $\vec{a}$, y si $\alpha<0$ entonces $\alpha\vec{a}$ tiene sentido opuesto al vector $\vec{a}$. Si $\alpha=0$ entonces $0\vec{a}=\vec{0}$. 
\end{block}\pause 

\begin{center}
  \visible<2-14>{\includegraphics[scale=0.42]{pondvector} } 
\end{center}
\end{minipage}\pause 
\hspace{0.5cm}\begin{minipage}[t]{0.5\linewidth}
\vspace{-0.5cm}
\begin{block}{Propiedades}\pause
\justifying
El conjunto $V$ de todos los vectores geométricos tiene las siguientes propiedades respecto a la adición y ponderación antes definidas: $\forall~\vec{a}, \vec{b}, \vec{c}\in V$ y $\alpha,\beta\in\mathbb{R}$. \pause

\vspace{0.3cm}
\tiny{
\begin{multicols}{2}
\begin{enumerate}[<+->]
\justifying
    \item $\vec{a}+\vec{b}\in V$. (Clausura de la adición)
    \item $(\vec{a}+\vec{b})+\vec{c}=\vec{a}+(\vec{b}+\vec{c})$. (Asociatividad)
    \item $\vec{a}+\vec{O}=\vec{a}$. (Elemento neutro aditivo)
    \item $\vec{a}+(-\vec{a})=\vec{0}$. (Inverso aditivo)
    \item $\vec{a}+\vec{b}=\vec{b}+\vec{a}$. (Conmutatividad)
    \item $\alpha\vec{a}\in V$. (Clausura de la ponderación)
    \item $1\vec{a}=\vec{a}$. (Neutro de la ponderación)
    \item $\alpha(\beta\vec{a})=(\alpha\beta)\vec{a}$. (Asociatividad)
    \item $(\alpha+\beta)\vec{a}=\alpha\vec{a}+\beta\vec{a}.$ (Distributividad)
    \item $\alpha(\vec{a}+\vec{b})=\alpha\vec{a}+\alpha\vec{b}$. (Distributividad)
\end{enumerate}
\end{multicols}}

\end{block}
\end{minipage}



\end{frame}


%---------------------------------------------------------------------
\subsection{Resta de vectores}
\begin{frame}
  \frametitle{\insertsectionhead}
  \framesubtitle{\insertsubsectionhead}
  
\justifying

\footnotesize

\begin{minipage}[t]{0.5\linewidth}
\vspace{-0.5cm}
\begin{exampleblock}{Ejemplo 1}
\justifying
A partir de la cuadrícula dada, determine los vectores $2\vec{u}+\vec{v}$ y $\vec{v}-\dfrac{1}{3}\vec{u}$. 
\begin{center}
    \includegraphics[scale=0.4]{EJ1}
\end{center}
\end{exampleblock}
\end{minipage}\pause 
\hspace{0.5cm}\begin{minipage}[t]{0.5\linewidth}
\begin{block}{Definición}
\justifying
La \textbf{resta} de los vectores $\vec{a}$ y $\vec{b}$ se define como $$\vec{a}-\vec{b}=\vec{a}+(-\vec{b}).$$
\begin{center}
\visible<2>{\includegraphics[scale=0.5]{resta}}
\end{center}
\end{block}
\end{minipage}

\end{frame}


%---------------------------------------------------------------------
%\subsection{Resta de vectores}
\begin{frame}
  \frametitle{\insertsectionhead}
  \framesubtitle{\insertsubsectionhead}
  
\justifying

\footnotesize

\begin{minipage}[t]{0.5\linewidth}
\vspace{-0.7cm}
\begin{block}{Definición}\pause
\justifying
Dados los puntos $A$ y $B$ del plano o el espacio, tenemos que el vector que comienza en $A$ y termina en $B$ se puede escribir como una resta de los vectores posición de $A$ y $B$, \pause es decir $$\overrightarrow{AB}=\overrightarrow{OB}-\overrightarrow{OA}=\vec{b}-\vec{a}.$$ \pause 

\vspace{-0.4cm}
\begin{center}
\visible<4-5>{\includegraphics[scale=0.06]{resta2}}
\end{center}
\end{block}
\end{minipage}\pause 
\hspace{0.5cm}\begin{minipage}[t]{0.5\linewidth}
\begin{exampleblock}{Ejemplo 2}
\justifying
Sean $A,B,C,D,E$ y $F$, en ese orden, los vértices de un hexágono regular. Demostrar que $$\overrightarrow{AB}+\overrightarrow{AC}+\overrightarrow{AD}+\overrightarrow{AE}+\overrightarrow{AF}=3\overrightarrow{AD}$$
\end{exampleblock}
\end{minipage}

\end{frame}


% %==============================================
% \section{Conclusión}
% %==============================================

% \begin{frame}
%   \frametitle{\insertsectionhead}
%   \framesubtitle{\insertsubsectionhead}
  
%   \justifying
  
%   \begin{itemize}[<+->]
%   \justifying
%       \item Es fundamental comprender que un vector representa a infinitos objetos que tienen las mismas características: sentido, dirección y magnitud.
%   \end{itemize}

% \end{frame}


% %==============================================
% \section{Asistencia}
% %==============================================

% \begin{frame}
%   \frametitle{\insertsectionhead}
%   \framesubtitle{\insertsubsectionhead}
  
%   \justifying
  
%   \textbf{Solo se considerará la asistencia hasta 5 minutos después de terminada la clase.}
  
%   \begin{center}
%   \includegraphics[scale=0.23]{QR}
%   \end{center}
  
% \end{frame}


\end{document}