\documentclass[handout,aspectratio=169]{beamer}
\usepackage{graphicx}
\usepackage{ragged2e}
\usefonttheme{professionalfonts}
\usepackage[spanish]{babel}
\usepackage{advdate}
\usepackage{multicol}
\setbeamercovered{transparent}
\usepackage{array} % for "\newcolumntype" macro
\newcolumntype{C}{>{$\displaystyle}c<{$}}
\newcommand\myfrac[2]{\frac{#1}{#2\mathstrut}}

% Document metadata
\title{Clase 18}
\subtitle{Vectores en coordenadas. Norma}
\author{Profesor: Denis Osses}
%\date{\AdvanceDate[+1]\today}
%\date{\today}
\date{16 de mayo de 2025}

% Image for the title page (use includegraphics option to properly size/place it)
\titlegraphic{\includegraphics[height=\paperheight]{imagen5}}

\usetheme[sectionstyle=style2]{trigon}

% Define logos to use (comment if no logo)
\biglogo{logoFIC} % Used on titlepage only
%\smalllogo{gggg}% Used on top right corner of regular frames

% ------ If you want to change the theme default colors, do it here ------
%\definecolor{tPrim}{HTML}{00843B}   % Green
%\definecolor{tSec}{HTML}{289B38}    % Green light
%\definecolor{tAccent}{HTML}{F07F3C} % Orange


% ------ Packages and definitions used for this demo. Can be removed ------
\usepackage{appendixnumberbeamer} % To use \appendix command
\pdfstringdefDisableCommands{% Fix hyperref translate warning with \appendix
\def\translate#1{#1}%
}
\usepackage{pgf-pie} % For pie charts
\usepackage{caption} % For subfigures
\usepackage{subcaption} % For subfigures
\usepackage{xspace}
\newcommand{\themename}{\textbf{\textsc{Álgebra}}\xspace}
\usepackage[scale=2]{ccicons} % Icons for CC-BY-SA
\usepackage{booktabs} % Better tables


%==============================================================================
%                               BEGIN DOCUMENT
%==============================================================================
\begin{document}

%--------------------------------------
% Create title frame
\titleframe

%--------------------------------------
% Table of contents
\begin{frame}{Temario}
  \setbeamertemplate{section in toc}[sections numbered]
  \tableofcontents%[hideallsubsections]
\end{frame}

%==============================================
\section{Objetivos de hoy}
%==============================================
%\subsection{Charts}
\begin{frame}{\insertsectionhead}
  \framesubtitle{\insertsubsectionhead}
  
\justifying

\begin{itemize}[<+->]
\justifying
    \item Representar algebraicamente vectores y sus operaciones.
    \item Aplicar las propiedades fundamentales de las operaciones de vectores para la resolución de problemas.
\end{itemize}

\end{frame}

%==============================================
\section{Contenidos}
%==============================================

%---------------------------------------------------------------------
\subsection{Vectores en coordenadas}
\begin{frame}
  \frametitle{\insertsectionhead}
  \framesubtitle{\insertsubsectionhead}
  
\justifying

\begin{minipage}[t]{0.5\linewidth}
\begin{block}{Definición. Puntos en el plano}
\justifying
Cualquier punto $P$ en el plano puede localizarce por un único par ordenado $(a,b)$ \pause 
\begin{center}
    \visible<2-4>{\includegraphics[scale=0.06]{GeomVect1.png}}
\end{center}
\end{block}
\end{minipage}\pause
\hspace{0.5cm}\begin{minipage}[t]{0.5\linewidth}
\vspace{-0.4cm}
\begin{block}{Definición}
\justifying
La \textbf{distancia entre los puntos} $P_1(x_1,y_1)$ y $P_2(x_2,y_2)$ es $$d(P_2,P_1)=\sqrt{(x_2-x_1)^2+(y_2-y_1)^2}.$$ \pause 
\vspace{-0.3cm}
\begin{center}
    \visible<4>{ \includegraphics[scale=0.5]{GeomVect2.png}}
\end{center}
\end{block}
\end{minipage}

\end{frame}



%---------------------------------------------------------------------
%\subsection{Vectores en Coordenadas}
\begin{frame}
  \frametitle{\insertsectionhead}
  \framesubtitle{\insertsubsectionhead}
  
\justifying
\footnotesize

\begin{minipage}[t]{0.5\linewidth}
\vspace{-0.3cm}
\begin{block}{Definición. Puntos en el espacio}
\justifying
Cualquier punto $P$ en el espacio puede localizarce por un único trío de números $(a,b,c)$ \pause 
\begin{center}
   \visible<2-6>{\includegraphics[scale=0.45]{GeomVect3.png}}
\end{center} \pause 
\end{block}
\begin{exampleblock}{Ejemplo 1}
Localice los puntos $(2,4,7)$ y $(-4,3,-5)$.
\end{exampleblock}
\end{minipage}\pause
\hspace{0.5cm}\begin{minipage}[t]{0.5\linewidth}
\vspace{-1cm}
\begin{center}
   \visible<4-6>{\includegraphics[scale=0.28]{GeomVect4.png}}
\end{center} \pause 
\vspace{-0.5cm}
\begin{block}{Definición}
\justifying
La \textbf{distancia entre los puntos del espacio} $P(x_1,y_1,z_1)$ y $Q(x_2,y_2,z_2)$ es $$d(Q,P)=\sqrt{(x_2-x_1)^2+(y_2-y_1)^2+(z_2-z_1)^2}.$$ \pause 
\vspace{-0.8cm}
\begin{center}
    \visible<6>{\includegraphics[scale=0.3]{GeomVect5.png}}
\end{center}
\vspace{-0.3cm}
\end{block}
\end{minipage}

\end{frame}

%---------------------------------------------------------------------
\subsection{Adición, ponderación y magnitud en coordenadas}
\begin{frame}
  \frametitle{\insertsectionhead}
  \framesubtitle{\insertsubsectionhead}
  
\justifying
\footnotesize

\begin{minipage}[t]{0.5\linewidth}
\vspace{-0.5cm}
\begin{block}{Definición}
\justifying
Todos los puntos $P(a,b)$ del plano y $P(a,b,c)$ del espacio tienen asociado un vector posición $$\vec{p}=(a,b)~~\text{o}~~\vec{p}=(a,b,c)$$ \pause que corresponde a las \textbf{coordenadas} de dicho vector. \pause Esto es así porque podemos mirar todos los puntos $P$ desde el origen $O$ y asociarles su vector posición $\overrightarrow{OP}=\vec{p}$. 
\end{block} \pause 

\begin{block}{Definición}
\justifying
Sean $\vec{p_1}=(a_1,b_1,c_1)$ o $\vec{p_2}=(a_2,b_2,c_2)$ vectores del espacio. La \textbf{adición} de $\vec{p_1}$ y $\vec{p_2}$ en coordenadas es $$\vec{p_1}+\vec{p_2}=(a_1+a_2,b_1+b_2,c_1+c_2).$$ 
\end{block}
\end{minipage}\pause
\hspace{0.5cm}\begin{minipage}[t]{0.5\linewidth}
\begin{block}{Definición}
\justifying
La \textbf{ponderación} del vector $\vec{p}=(a,b,c)$ por el escalar $\alpha\in\mathbb{R}$ en coordenadas es $$\alpha\vec{p}=\alpha(a,b,c)=(\alpha a,\alpha b,\alpha c).$$
\end{block} \pause 

\begin{block}{Definición}
\justifying
El \textbf{módulo} o \textbf{norma} del vector $\vec{p}=(a,b,c)$ es $$||\vec{p}||=\sqrt{a^2+b^2+c^2}.$$
\end{block}
\end{minipage}



\end{frame}


%---------------------------------------------------------------------
\subsection{Distancia}
\begin{frame}
  \frametitle{\insertsectionhead}
  \framesubtitle{\insertsubsectionhead}
  
\justifying
\footnotesize

\begin{minipage}[t]{0.5\linewidth}
\begin{block}{Propiedad}
\justifying
Si $\vec{u}$ es un vector en el plano o el espacio y $\alpha\in\mathbb{R}$, entonces: $$||\alpha\vec{u}||=|\alpha|\cdot||\vec{u}||.$$ 
\end{block}\pause

\begin{block}{Definición}
\justifying
Si $A$ y $B$ son puntos del plano o el espacio, y $\vec{a}$, $\vec{b}$ son sus vectores posición asociados entonces: $$d(A,B)=||\overrightarrow{AB}||=||\vec{b}-\vec{a}||$$ \pause  es la \textbf{distancia} entre $A$ y $B$, o la norma del vector $\overrightarrow{AB}$. 
\end{block}
\end{minipage}\pause 
\hspace{0.5cm}\begin{minipage}[t]{0.5\linewidth}
\vspace{1.5cm}
\begin{exampleblock}{Ejemplo 2}
\justifying
Obtener las coordenadas del vector $\overrightarrow{AB}$ determinado por un punto inicial $A(3,2)$ y un punto final $B(-5,1)$. ¿Cuáles son las coordenadas del punto $D$ si $\overrightarrow{CD}=\overrightarrow{AB}$ con $C(-4,7)$?
\end{exampleblock}
\end{minipage}


\end{frame}




% %==============================================
% \section{Conclusión}
% %==============================================

% \begin{frame}
%   \frametitle{\insertsectionhead}
%   \framesubtitle{\insertsubsectionhead}
  
%   \justifying
  
%   \begin{itemize}[<+->]
%   \justifying
%       \item La definiciones para vectores en $\mathbb{R}^3$ y $\mathbb{R}^2$ son análogas. En $\mathbb{R}^3$, basta con anular la tercera coordenada en las expresiones anteriores para obtener las definiciones en $\mathbb{R}^2$.
%   \end{itemize}

% \end{frame}


% %==============================================
% \section{Asistencia}
% %==============================================

% \begin{frame}
%   \frametitle{\insertsectionhead}
%   \framesubtitle{\insertsubsectionhead}
  
%   \justifying
  
%   \textbf{Solo se considerará la asistencia hasta 5 minutos después de terminada la clase.}
  
%   \begin{center}
%   \includegraphics[scale=0.23]{QR}
%   \end{center}
  
% \end{frame}


\end{document}