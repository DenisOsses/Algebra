\documentclass[handout,aspectratio=169]{beamer}
\usepackage{graphicx}
\usepackage{ragged2e}
\usefonttheme{professionalfonts}
\usepackage[spanish]{babel}
\usepackage{advdate}
\usepackage{multicol}
\setbeamercovered{transparent}
\usepackage{array} % for "\newcolumntype" macro
\newcolumntype{C}{>{$\displaystyle}c<{$}}
\newcommand\myfrac[2]{\frac{#1}{#2\mathstrut}}

% Document metadata
\title{Clase 19}
\subtitle{Combinación lineal. Vectores paralelos. Producto Punto}
\author{Profesor: Denis Osses}
%\date{\AdvanceDate[+1]\today}
%\date{\today}
\date{23 de mayo de 2025}

% Image for the title page (use includegraphics option to properly size/place it)
\titlegraphic{\includegraphics[height=\paperheight]{imagen5}}

\usetheme[sectionstyle=style2]{trigon}

% Define logos to use (comment if no logo)
\biglogo{logoFIC} % Used on titlepage only
%\smalllogo{gggg}% Used on top right corner of regular frames

% ------ If you want to change the theme default colors, do it here ------
%\definecolor{tPrim}{HTML}{00843B}   % Green
%\definecolor{tSec}{HTML}{289B38}    % Green light
%\definecolor{tAccent}{HTML}{F07F3C} % Orange


% ------ Packages and definitions used for this demo. Can be removed ------
\usepackage{appendixnumberbeamer} % To use \appendix command
\pdfstringdefDisableCommands{% Fix hyperref translate warning with \appendix
\def\translate#1{#1}%
}
\usepackage{pgf-pie} % For pie charts
\usepackage{caption} % For subfigures
\usepackage{subcaption} % For subfigures
\usepackage{xspace}
\newcommand{\themename}{\textbf{\textsc{Álgebra}}\xspace}
\usepackage[scale=2]{ccicons} % Icons for CC-BY-SA
\usepackage{booktabs} % Better tables


%==============================================================================
%                               BEGIN DOCUMENT
%==============================================================================
\begin{document}

%--------------------------------------
% Create title frame
\titleframe

%--------------------------------------
% Table of contents
\begin{frame}{Temario}
  \setbeamertemplate{section in toc}[sections numbered]
  \tableofcontents%[hideallsubsections]
\end{frame}

%==============================================
\section{Objetivos de hoy}
%==============================================
%\subsection{Charts}
\begin{frame}{\insertsectionhead}
  \framesubtitle{\insertsubsectionhead}
  
\justifying

\begin{itemize}[<+->]
\justifying
    \item Aplicar las propiedades fundamentales de las operaciones de vectores para la resolución de problemas.
    \item Definir los conceptos de paralelismo entre vectores y producto punto.
    % \item Reconocer los elementos de las diferentes representaciones de rectas y determinar sus ecuaciones. Usar las distintas representaciones de rectas y sus propiedades para resolver problemas geométricos.
\end{itemize}

\end{frame}

%==============================================
\section{Contenidos}
%==============================================

%---------------------------------------------------------------------
\subsection{Combinación lineal}
\begin{frame}
  \frametitle{\insertsectionhead}
  \framesubtitle{\insertsubsectionhead}
  
\justifying

\begin{minipage}[t]{0.5\linewidth}
\begin{block}{Definición}
\justifying
Un vector $\vec{v}$ es \textbf{combinación lineal} de los vectores $\vec{u_1}$ y $\vec{u_2}$ si existen contantes $\alpha,\beta\in\mathbb{R}$ tales que $$\vec{v}=\alpha\vec{u_1}+\beta\vec{u_2}.$$
\end{block}\pause

\begin{exampleblock}{Ejemplo 1}
\justifying
Sean $\vec{u}=(3,1)$ y $\vec{v}=(1,2)$. Determine si el vector $\vec{w}=(-1,3)$ es combinación lineal de $\vec{u}$ y $\vec{v}$.
\end{exampleblock}
\end{minipage}\pause
\hspace{0.2cm}\begin{minipage}[t]{0.5\linewidth}
\begin{center}
   \visible<3>{\includegraphics[scale=0.8]{Vect1.png}}
\end{center}
\end{minipage}

\end{frame}

%---------------------------------------------------------------------
\subsection{Vectores paralelos}
\begin{frame}
  \frametitle{\insertsectionhead}
  \framesubtitle{\insertsubsectionhead}
  
\justifying
\footnotesize

\begin{minipage}[t]{0.5\linewidth}
\begin{block}{Definición}
\justifying
Decimos que los vectores $\vec{u}$ y $\vec{v}$ son \textbf{paralelos} si tienen la misma dirección. \pause Matemáticamente esto significa que un vector es un ponderado o múltiplo del otro, es decir, existe un número $\lambda\in\mathbb{R}$, $\lambda\neq0$, tal que $$\vec{u}=\lambda\vec{v}.$$
\end{block}\pause

\begin{exampleblock}{Ejemplo 2}
Decida si los vectores $\vec{u}=(-1,2)$ y $\vec{v}=(4,-8)$ son paralelos o no.
\end{exampleblock}
\end{minipage}\pause
\hspace{0.3cm}\begin{minipage}[t]{0.5\linewidth}
% \vspace{0.2cm}
% Observemos la recta $L$ (del plano o espacio) en la figura siguiente: \pause

% \begin{center}
%     \visible<5>{\includegraphics[scale=0.14]{EcVecRecta}}
% \end{center}
\end{minipage}

\end{frame}

%---------------------------------------------------------------------
\subsection{Producto punto}
\begin{frame}
  \frametitle{\insertsectionhead}
  \framesubtitle{\insertsubsectionhead}
  
\justifying
\footnotesize

\begin{minipage}[t]{0.5\linewidth}
\vspace{-0.7cm}
\begin{block}{Definición}
\justifying
Sean $\vec{a}=(a_1,a_2,a_3)$ y $\vec{b}=(b_1,b_2,b_3)$ vectores en el espacio. \pause El \textbf{producto punto} o \textbf{producto escalar} de $\vec{a}$ con $\vec{b}$ es el número real $\vec{a}\cdot\vec{b}$ definido por \pause $$\vec{a}\cdot\vec{b}=a_1b_1+a_2b_2+a_3b_3=\sum_{i=1}^3a_ib_i.$$
\end{block}\pause 
Así, para encontrar el producto punto de $\vec{a}$ con $\vec{b}$ se multiplican las coordenadas correspondientes y se suman. \pause El resultado no es un vector, es un número real. \pause 
\begin{exampleblock}{Ejemplo 3}
\justifying
Calcule el producto punto entre $\vec{a}=(1,2,3)$ y $\vec{b}=(-1,0,1)$.
\end{exampleblock}
\end{minipage}\pause 
\hspace{0.5cm}\begin{minipage}[t]{0.5\linewidth}
\begin{block}{Propiedades}\pause
\justifying Si $\vec{a}$, $\vec{b}$ y $\vec{c}$ son vectores en el plano o en el espacio y $\alpha\in\mathbb{R}$, entonces \pause
\begin{enumerate}[<+->][{(a)}]
\justifying
    \item $\vec{a}\cdot\vec{a}=||\vec{a}||^2$.
    \item $\vec{a}\cdot \vec{b}=\vec{b} \cdot \vec{a}$.
    \item $\vec{a}\cdot (\vec{b}+\vec{c})=\vec{a}\cdot \vec{b}+\vec{a}\cdot \vec{c}$.
    \item $\alpha(\vec{a}\cdot \vec{b})=(\alpha \vec{a})\cdot \vec{b}=\vec{a}\cdot (\alpha \vec{b})$.
    \item $\vec{0}\cdot\vec{a}=0$.
    \item $\vec{a}\cdot \vec{b}=\frac{1}{2}( ||\vec{a}||^2+||\vec{b}||^2-||\vec{b}-\vec{a}||^2)$
\end{enumerate}
\end{block}

\end{minipage}

\end{frame}

% %---------------------------------------------------------------------
% \subsection{Ángulo entre vectores}
% \begin{frame}
%   \frametitle{\insertsectionhead}
%   \framesubtitle{\insertsubsectionhead}
  
% \justifying
% \footnotesize

% \begin{minipage}[t]{0.5\linewidth}
% \vspace{-0.5cm}
% \begin{center}
%     \includegraphics[scale=0.7]{AnguloVect.png}
% \end{center} \pause
% El producto punto $\vec{a}\cdot \vec{b}$ se puede interpretar geométricamente en términos del ángulo $\theta$ entre $\vec{a}$ y $\vec{b}$, que se define como el ángulo entre los vectores posición asociados a ellos. 
% \end{minipage}\pause 
% \hspace{0.5cm}\begin{minipage}[t]{0.5\linewidth}
% \vspace{-0.5cm}
% \begin{block}{Teorema}
% \justifying
% Si $\theta$ es el ángulo entre $\vec{a}$ y $\vec{b}$ entonces $$\color{red}\boxed{\color{black}\vec{a}\cdot\vec{b}=||\vec{a}||\cdot||\vec{b}||\cos(\theta).}$$ \pause
% Además, $$\color{red}\boxed{\color{black}\cos(\theta)=\frac{\vec{a}\cdot\vec{b}}{||\vec{a}||\cdot||\vec{b}||}.}$$
% \end{block}\pause
% \begin{block}{Nota}
% \justifying
% Si $\theta=0$ o $\theta=\pi$ entonces los vectores son paralelos y $\vec{a}\cdot\vec{b}=\pm||\vec{a}||\cdot||\vec{b}||$, respectivamente.
% \end{block}
% \end{minipage}

% \end{frame}

% %---------------------------------------------------------------------
% \subsection{Ecuación vectorial de la recta}
% \begin{frame}
%   \frametitle{\insertsectionhead}
%   \framesubtitle{\insertsubsectionhead}
  
% \justifying
% \footnotesize

% \begin{minipage}[t]{0.5\linewidth}
% $P_0$ y $P_1$ son puntos fijos de la recta $L$ y $P$ es un punto variable (móvil sobre la recta). \pause Notamos que $$\vec{r}=\vec{p_0}+\overrightarrow{P_0P}.$$ \pause Por otro lado, el vector $\overrightarrow{P_0P}$ es paralelo con $\vec{v}$, es decir $\overrightarrow{P_0P}=\lambda\vec{v}$ para algún $\lambda\in\mathbb{R}.$ \pause Luego $${\color{red}\boxed{\color{black}\vec{r}=\vec{p_0}+\lambda\vec{v}}}$$ \pause es conocida como la \textbf{ecuación vectorial de la recta} en el plano o el espacio, \pause que pasa por el punto $P_0$ y sigue la dirección del vector $\vec{v}$ (\textbf{vector director} de la recta).
% \end{minipage} \pause
% \hspace{0.3cm}\begin{minipage}[t]{0.5\linewidth}
% \begin{block}{Nota 1}
% \justifying
% El vector director $\vec{v}$ puede ser cualquier vector que siga o indique la dirección de la recta $L$; así que no es necesario que esté explícitamente sobre dicha recta, basta con un vector paralelo a ella.
% \end{block}\pause 

% \begin{block}{Nota 2}
% \justifying
% Podemos visualizar la ecuación de la recta en el plano con punto móvil $P$ y vector dirección $\vec{v}$ en el siguiente enlace de \href{https://www.geogebra.org/m/Tn8n4Q5a}{\textbf{Geogebra}} y el siguiente \href{https://www.geogebra.org/m/R28FMcuF}{\textbf{enlace}} para la recta en el espacio.
% \end{block}
% \end{minipage}

% \end{frame}

% %---------------------------------------------------------------------
% \subsection{Ecuación paramétrica y cartesiana de la recta}
% \begin{frame}
%   \frametitle{\insertsectionhead}
%   \framesubtitle{\insertsubsectionhead}
  
% \justifying
% \footnotesize

% \begin{minipage}[t]{0.5\linewidth}
% Si escribimos los vectores anteriores en coordenadas: \pause $\vec{r}=(x,y,z)$, $\vec{p_0}=(x_0,y_0,z_0)$, \newline $\vec{v}=(a,b,c)$, \pause entonces podemos deducir la ecuación \textbf{paramétrica} de la recta \begin{eqnarray*}x&=&x_0+a\lambda\\ y&=&y_0+b\lambda~~~~~~~~,~~~~\lambda\in\mathbb{R}\\ z&=&z_0+c\lambda\end{eqnarray*} \pause y la ecuación \textbf{cartesiana} de la recta \pause $$\frac{x-x_0}{a}=\frac{y-y_0}{b}=\frac{z-z_0}{c}\quad,\quad a,b,c\neq0$$
% \end{minipage} \pause
% \hspace{0.3cm}\begin{minipage}[t]{0.5\linewidth}
% \vspace{-0.3cm}
% \begin{exampleblock}{Ejemplo 3}
% \justifying
% Determinar en cada caso un vector director para la recta indicada y una ecuación vectorial de esa recta que: \pause

% \begin{enumerate}[<+->]
% \justifying
%     \item Contiene a los puntos $A(3,9)$ y $B(-5,1)$.
%     %\item Recta que contiene al punto $C(12,1)$ y es paralela a la recta $(x,y)=(1,1)+\lambda(4,-5)$, $\lambda \in \mathbb{R}$.
%     %\item Recta que contiene al punto $C(0,4)$ y es perpendicular a la recta $(x,y)=(7,-1)+\lambda(3,2)$, $\lambda \in\mathbb{R}$.
%     \item Contiene a los dos puntos $D(1,-2,1)$ y $E(1,-1,1)$.
%     \item Contiene al punto $G(0,1,-2)$ y es paralela a la recta $(x,y,z)=(2,0,1)+\lambda(3,1,0)$, $\lambda \in\mathbb{R}$.
%     \item Pasa por $P(1,0,1)$ y que corta a la recta de ecuación vectorial $\vec{r}=(1, 2,-1)+\lambda(2, 1,-2)$, en los puntos situados a dos unidades de distancia de $P_0(1, 2,-1)$.
% \end{enumerate}

% \end{exampleblock}
% \end{minipage}

% \end{frame}


% %==============================================
% \section{Conclusión}
% %==============================================

% \begin{frame}
%   \frametitle{\insertsectionhead}
%   \framesubtitle{\insertsubsectionhead}
  
%   \justifying
  
%   \begin{itemize}[<+->]
%   \justifying
%       \item La definiciones para vectores en $\mathbb{R}^3$ y $\mathbb{R}^2$ son análogas. En $\mathbb{R}^3$, basta con anular la tercera coordenada en las expresiones anteriores para obtener las definiciones en $\mathbb{R}^2$.
%       \item Un buen ejercicio es probar que la ecuación vectorial de la recta en el plano es equivalente a la ecuación lineal en el plano.
%   \end{itemize}

% \end{frame}


% %==============================================
% \section{Asistencia}
% %==============================================

% \begin{frame}
%   \frametitle{\insertsectionhead}
%   \framesubtitle{\insertsubsectionhead}
  
%   \justifying
  
%   \textbf{Solo se considerará la asistencia hasta 5 minutos después de terminada la clase.}
  
%   \begin{center}
%   \includegraphics[scale=0.23]{QR}
%   \end{center}
  
% \end{frame}


\end{document}