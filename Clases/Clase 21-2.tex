\documentclass[handout,aspectratio=169]{beamer}
\usepackage{graphicx}
\usepackage{ragged2e}
\usefonttheme{professionalfonts}
\usepackage[spanish]{babel}
\usepackage{advdate}
\usepackage{multicol}
\setbeamercovered{transparent}
\usepackage{array} % for "\newcolumntype" macro
\newcolumntype{C}{>{$\displaystyle}c<{$}}
\newcommand\myfrac[2]{\frac{#1}{#2\mathstrut}}

% Document metadata
\title{Clase 23}
\subtitle{Posiciones relativas de rectas. Vector proyección}
\author{Profesor: Denis Osses}
%\date{\AdvanceDate[+1]\today}
%\date{\today}
\date{7 de junio de 2024}

% Image for the title page (use includegraphics option to properly size/place it)
\titlegraphic{\includegraphics[height=\paperheight]{imagen5}}

\usetheme[sectionstyle=style2]{trigon}

% Define logos to use (comment if no logo)
\biglogo{logoFIC} % Used on titlepage only
%\smalllogo{gggg}% Used on top right corner of regular frames

% ------ If you want to change the theme default colors, do it here ------
%\definecolor{tPrim}{HTML}{00843B}   % Green
%\definecolor{tSec}{HTML}{289B38}    % Green light
%\definecolor{tAccent}{HTML}{F07F3C} % Orange


% ------ Packages and definitions used for this demo. Can be removed ------
\usepackage{appendixnumberbeamer} % To use \appendix command
\pdfstringdefDisableCommands{% Fix hyperref translate warning with \appendix
\def\translate#1{#1}%
}
\usepackage{pgf-pie} % For pie charts
\usepackage{caption} % For subfigures
\usepackage{subcaption} % For subfigures
\usepackage{xspace}
\newcommand{\themename}{\textbf{\textsc{Álgebra}}\xspace}
\usepackage[scale=2]{ccicons} % Icons for CC-BY-SA
\usepackage{booktabs} % Better tables


%==============================================================================
%                               BEGIN DOCUMENT
%==============================================================================
\begin{document}

%--------------------------------------
% Create title frame
\titleframe

%--------------------------------------
% Table of contents
\begin{frame}{Temario}
  \setbeamertemplate{section in toc}[sections numbered]
  \tableofcontents%[hideallsubsections]
\end{frame}

%==============================================
\section{Objetivos de hoy}
%==============================================
%\subsection{Charts}
\begin{frame}{\insertsectionhead}
  \framesubtitle{\insertsubsectionhead}
  
\justifying

\begin{itemize}[<+->]
\justifying
    \item Caracterizar la perpendicularidad de rectas en el plano y en el espacio.
    \item Aplicar las propiedades fundamentales de las operaciones de vectores para la resolución de problemas.
\end{itemize}

\end{frame}

%==============================================
\section{Contenidos}
%==============================================

% %---------------------------------------------------------------------
% \subsection{Paralelismo de rectas}
% \begin{frame}
%   \frametitle{\insertsectionhead}
%   \framesubtitle{\insertsubsectionhead}
  
% \justifying

% \begin{block}{Definición}
% \justifying
% Dos rectas en el espacio (o en el plano) son paralelas si y solo si sus vectores directores son paralelos y dichas rectas no se cortan. \pause Luego, rectas coincidentes no son paralelas.
% \end{block}\pause

% \begin{exampleblock}{Ejemplo 1}
% \justifying
% % Determine, en cada caso, si los tres puntos dados son colineales.
% Determine si los tres puntos dados son colineales: $P(1, 2,-3), Q(3,-4, 2)$ y $R(1, 1, 2)$
% % \begin{enumerate}
% % \item $P(1, 2,-3), Q(3,-4, 2)$ y $R(1, 1, 2)$.
% % \item $P(1, 2,-3), Q(3,-4, 2)$ y $R(-1, 8,-8)$.
% % \end{enumerate}
% \end{exampleblock}\pause

% \begin{exampleblock}{Ejemplo 2}
% \justifying
% ¿Es paralela la recta $\displaystyle \frac{x-1}{2}=\frac{y-3}{3}=\frac{z-4}{1}$ con $x=4t, y=6t, z=2t$, $t\in\mathbb{R}$?
% \end{exampleblock}

% \end{frame}

%---------------------------------------------------------------------
\subsection{Perpendicularidad de rectas. Vector Proyección}
\begin{frame}
  \frametitle{\insertsectionhead}
  \framesubtitle{\insertsubsectionhead}
  
\justifying
\footnotesize

\begin{minipage}[t]{0.5\linewidth}
\vspace{-0.3cm}
\begin{block}{Definición}
\justifying
Dos rectas en el espacio (o en el plano) son perpendiculares si y solo si sus vectores directores son perpendiculares y si dichas rectas se cortan en un punto. 
\end{block}\pause
\begin{exampleblock}{Ejemplo 1}
\justifying
¿Es perpendicular la recta $\displaystyle \vec{r}=(1,0,1)+\lambda(1,2,3)$, $\lambda\in\mathbb{R}$, con $x=1+t, y=t, z=1-4t$, $t\in\mathbb{R}$?
\end{exampleblock} \pause
\begin{block}{Definición}
\justifying
El vector $\lambda\vec{b}$ se denomina \textbf{vector proyección ortogonal} de $\vec{x}$ sobre $\vec{b}$ y se denota por $P_{\vec{b}}(\vec{x})$.
\end{block}
\end{minipage}\pause 
\hspace{0.5cm}\begin{minipage}[t]{0.5\linewidth}
\visible<4-5>{\includegraphics[scale=0.4]{Proyeccion1.png}}\pause
\vspace{0.5cm}
Los vectores $\vec{b}$ e $\vec{y}$ son ortogonales en el punto $B$. El vector $\overrightarrow{AB}=\lambda\vec{b}$ es un ponderado de $\vec{b}$ (ya que son paralelos).
\end{minipage}

\end{frame}

% %---------------------------------------------------------------------
% %\subsection{Perpendicularidad de rectas}
% \begin{frame}
%   \frametitle{\insertsectionhead}
%   \framesubtitle{\insertsubsectionhead}
  
% \justifying

% \begin{exampleblock}{Ejemplo 3}
% \justifying
% Considere las rectas 
% \begin{eqnarray*}
% \ell_1 &:& (x, y, z) = (1, 3,-2) + t(4,5,-3),~~t\in\mathbb{R}\\ 
% \ell_2 &:& \dfrac{x-2}{2}=\dfrac{y+3}{3}=\dfrac{5-z}{2}
% \end{eqnarray*}

% \begin{enumerate}[(1)]
% \justifying
% \item Determine si las rectas $\ell_1$ y $\ell_2$ se intersectan.
% \item Encuentre una recta $\ell$ que sea paralela a la recta $\ell_1$ y que pase por el punto $(-2, 3, 5)$.
% \item Determine una recta $\ell$ que sea perpendicular a la recta $\ell_2$ y que pase por el punto $(2, 0, 1)$.
% \end{enumerate}
% \end{exampleblock}

% \end{frame}

%---------------------------------------------------------------------
\subsection{Vector proyección. Propiedades}
\begin{frame}
  \frametitle{\insertsectionhead}
  \framesubtitle{\insertsubsectionhead}
  
\justifying
\footnotesize

\begin{minipage}[t]{0.5\linewidth}
%\vspace{-4cm}
Notamos que 
\vspace{-0.2cm}
$$\lambda\vec{b}+\vec{y}=\vec{x}.$$\pause Como el vector $\vec{y}$ es ortogonal con $\vec{b}$, tenemos que $\vec{y}\cdot\vec{b}=0$. \pause Luego, si a la igualdad anterior le realizamos un producto punto con $\vec{b}$, obtenemos que $$\lambda=\frac{\vec{x}\cdot\vec{b}}{\vec{b}\cdot\vec{b}}=\frac{\vec{x}\cdot\vec{b}}{||\vec{b}||^2}.$$\pause Por lo tanto 
\vspace{-0.3cm}
$$\color{red}\boxed{\color{black}P_{\vec{b}}(\vec{x})=\left(\frac{\vec{x}\cdot\vec{b}}{||\vec{b}||^2}\right)\vec{b}}$$ \pause
\vspace{-0.4cm}
\begin{block}{Nota}
\justifying
El vector $\vec{y}=\vec{x}-P_{\vec{b}}(\vec{x})$ se denomina \textbf{complemento ortogonal} y es muy útil para determinar distancias.
\end{block}
\end{minipage}\pause 
\hspace{0.5cm}\begin{minipage}[t]{0.5\linewidth}
\vspace{-1.8cm}
\begin{exampleblock}{Ejemplo 2}
\justifying
Si $\vec{u}=(2,4)$ y $\vec{v}=(5,3)$, determine $P_{\vec{v}}(\vec{u})$ y $P_{\vec{u}}(\vec{v})$. Dibuje estos vectores y sus complementos ortogonales.
\end{exampleblock} \pause
\vspace{-0.2cm}
\begin{block}{Propiedades} 
\justifying
Si $\displaystyle P_{\vec{b}}(\vec{x})$ denota la proyección ortogonal de $\vec{x}$ sobre $\vec{b}$ entonces: \pause 
\begin{enumerate}[<+->][{(1)}]
\item $\displaystyle P_{\vec{b}}(\vec{b})=\vec{b}$.
\item Si $\vec{a}$ es cualquier vector perpendicular al vector $\vec{b}$ entonces $\displaystyle P_{\vec{b}}(\vec{a})=\vec{0}$.
\item $\displaystyle P_{\vec{b}}(\lambda\vec{x})=\lambda \displaystyle P_{\vec{b}}(\vec{x})$.
\item $\displaystyle P_{\vec{b}}(\vec{x}+\vec{y})=\displaystyle P_{\vec{b}}(\vec{x})+\displaystyle P_{\vec{b}}(\vec{y})$
\end{enumerate}
\vspace{-0.2cm}
\end{block}\pause 
\vspace{-0.2cm}
\begin{exampleblock}{Ejemplo 3}
\justifying
Considere la recta $L:\frac{x-1}{2}=y-2=\frac{z-3}{2}.$ Encuentre la ecuación de la recta perpendicular a $L$ que pasa por el punto $(1,2,1)$.
\end{exampleblock}
\end{minipage}

\end{frame}


% %---------------------------------------------------------------------
% %\subsection{Vector proyección}
% \begin{frame}
%   \frametitle{\insertsectionhead}
%   \framesubtitle{\insertsubsectionhead}
  
% \justifying

% \begin{exampleblock}{Ejemplo 5}
% \justifying
% Considere la recta $$L:\frac{x-1}{2}=y-2=\displaystyle\frac{z-3}{2}.$$ Encuentre la ecuación de la recta perpendicular a $L$ que pasa por el punto $(1,2,1)$.
% \end{exampleblock}

% % \begin{exampleblock}{Ejemplo 6}
% % \justifying
% % Calcule la distancia más corta desde el punto $(2,6)$ a la recta que pasa por el origen y tiene vector director a $\vec{v}=(7,4)$.
% % \end{exampleblock}

% \end{frame}


% %==============================================
% \section{Conclusión}
% %==============================================

% \begin{frame}
%   \frametitle{\insertsectionhead}
%   \framesubtitle{\insertsubsectionhead}
  
%   \justifying
  
%   \begin{itemize}[<+->]
%   \justifying
%       \item A veces existen distintas definiciones de perpendicularidad de 2 rectas en el espacio. Acá elegimos la que requiere ortogonalidad de sus vectores directores e intersección entre ellas.
%       % \item Un excelente ejercicio es realizar el ejemplo 3, parte (3), usando vector proyección y complemento ortogonal.
%   \end{itemize}

% \end{frame}


% %==============================================
% \section{Asistencia}
% %==============================================

% \begin{frame}
%   \frametitle{\insertsectionhead}
%   \framesubtitle{\insertsubsectionhead}
  
%   \justifying
  
%   \textbf{Solo se considerará la asistencia hasta 5 minutos después de terminada la clase.}
  
%   \begin{center}
%   \includegraphics[scale=0.23]{QR}
%   \end{center}
  
% \end{frame}


\end{document}