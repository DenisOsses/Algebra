\documentclass[aspectratio=169]{beamer}
\usepackage{graphicx}
\usepackage{ragged2e}
\usefonttheme{professionalfonts}
\usepackage[spanish]{babel}
\usepackage{advdate}
\usepackage{multicol}
\setbeamercovered{transparent}
\usepackage{array} % for "\newcolumntype" macro
\newcolumntype{C}{>{$\displaystyle}c<{$}}
\newcommand\myfrac[2]{\frac{#1}{#2\mathstrut}}

% Document metadata
\title{Clase 22}
\subtitle{Producto cruz}
\author{Profesor: Denis Osses}
%\date{\AdvanceDate[+1]\today}
%\date{\today}
\date{6 de junio de 2025}

% Image for the title page (use includegraphics option to properly size/place it)
\titlegraphic{\includegraphics[height=\paperheight]{imagen5}}

\usetheme[sectionstyle=style2]{trigon}

% Define logos to use (comment if no logo)
\biglogo{logoFIC} % Used on titlepage only
%\smalllogo{gggg}% Used on top right corner of regular frames

% ------ If you want to change the theme default colors, do it here ------
%\definecolor{tPrim}{HTML}{00843B}   % Green
%\definecolor{tSec}{HTML}{289B38}    % Green light
%\definecolor{tAccent}{HTML}{F07F3C} % Orange


% ------ Packages and definitions used for this demo. Can be removed ------
\usepackage{appendixnumberbeamer} % To use \appendix command
\pdfstringdefDisableCommands{% Fix hyperref translate warning with \appendix
\def\translate#1{#1}%
}
\usepackage{pgf-pie} % For pie charts
\usepackage{caption} % For subfigures
\usepackage{subcaption} % For subfigures
\usepackage{xspace}
\newcommand{\themename}{\textbf{\textsc{Álgebra}}\xspace}
\usepackage[scale=2]{ccicons} % Icons for CC-BY-SA
\usepackage{booktabs} % Better tables


%==============================================================================
%                               BEGIN DOCUMENT
%==============================================================================
\begin{document}

%--------------------------------------
% Create title frame
\titleframe

%--------------------------------------
% Table of contents
\begin{frame}{Temario}
  \setbeamertemplate{section in toc}[sections numbered]
  \tableofcontents%[hideallsubsections]
\end{frame}

%==============================================
\section{Objetivos de hoy}
%==============================================
%\subsection{Charts}
\begin{frame}{\insertsectionhead}
  \framesubtitle{\insertsubsectionhead}
  
\justifying

\begin{itemize}[<+->]
\justifying
    \item Definir producto cruz entre vectores y sus propiedades.
    \item Aplicar las propiedades fundamentales de las operaciones de vectores para la resolución de problemas.
\end{itemize}

\end{frame}

%==============================================
\section{Contenidos}
%==============================================

%---------------------------------------------------------------------
\subsection{Producto cruz}
\begin{frame}
  \frametitle{\insertsectionhead}
  \framesubtitle{\insertsubsectionhead}
  
\justifying
\footnotesize

\begin{minipage}[t]{0.5\linewidth}
Consideremos el siguiente problema: dados dos vectores no colineales $\vec{a}$ y $\vec{b}$ en el espacio, nos interesa obtener un tercer vector $\vec{c}$ que cumpla las siguientes condiciones: \pause

\begin{enumerate}[{(1)}]
    \item $\vec{c}$ es perpendicular a $\vec{a}$. \pause
    \item $\vec{c}$ es perpendicular a $\vec{b}$.
    %\item $\vec{c}$ es igual al 'area del paralel'ogramo que determinan los vectores $\vec{a}$ y $\vec{b}$.
\end{enumerate} \pause 

\begin{block}{Definición}
\justifying
Si $\vec{a}=(a_1,a_2,a_3)$ y $\vec{b}=(b_1,b_2,b_3)$ entonces el \textbf{producto cruz} de $\vec{a}$ y $\vec{b}$ es el vector \pause $$\color{red}\boxed{\color{black}\vec{a}\times\vec{b}=(a_2b_3-a_3b_2,a_3b_1-a_1b_3,a_1b_2-a_2b_1)}$$
\end{block}
\end{minipage}  \pause 
\hspace{0.5cm}\begin{minipage}[t]{0.5\linewidth}
A fin de hacer la definición más fácil de recordar, se usa la notación de determinantes. \pause Un \textbf{determinante de orden 2} se define mediante $$\begin{vmatrix}a&b\\c&d\end{vmatrix}=ad-bc.$$\pause

Un \textbf{determinante de orden 3} se puede definir en términos de determinantes de segundo orden como sigue: \pause $$\begin{vmatrix}a&b&c\\a_1&a_2&a_3\\ b_1&b_2&b_3\end{vmatrix}=a\begin{vmatrix}a_2&a_3\\b_2&b_3\end{vmatrix}-b\begin{vmatrix}a_1&a_3\\b_1&b_3\end{vmatrix}+c\begin{vmatrix}a_1&a_2\\b_1&b_2\end{vmatrix}.$$
\end{minipage}

\end{frame}

%---------------------------------------------------------------------
\subsection{Propiedades}
\begin{frame}
  \frametitle{\insertsectionhead}
  \framesubtitle{\insertsubsectionhead}
  
\justifying
\footnotesize

\begin{minipage}[t]{0.5\linewidth}
Ahora, si definimos los vectores (\textbf{canónicos}) del espacio: $\hat{i}=(1,0,0)$, $\hat{j}=(0,1,0)$ y $\hat{k}=(0,0,1)$, \pause podemos reescribir el producto cruz de $\vec{a}$ con $\vec{b}$ como \pause 
\tiny{$$\color{red}\boxed{\color{black}\vec{a}\times\vec{b}=\begin{vmatrix}\hat{i}&\hat{j}&\hat{k}\\a_1&a_2&a_3\\ b_1&b_2&b_3\end{vmatrix}=\begin{vmatrix}a_2&a_3\\b_2&b_3\end{vmatrix}\hat{i}-\begin{vmatrix}a_1&a_3\\b_1&b_3\end{vmatrix}\hat{j}+\begin{vmatrix}a_1&a_2\\b_1&b_2\end{vmatrix}\hat{k}}$$} \pause
\footnotesize
\begin{exampleblock}{Ejemplo 1}
\justifying
Determine el producto cruz de los vectores $\vec{a}=(1,1,1)$ y $\vec{b}=(-1,0,2)$.
\end{exampleblock}
\end{minipage}\pause 
\hspace{0.8cm}\begin{minipage}[t]{0.47\linewidth}
\vspace{-0.8cm}
\begin{block}{Propiedades}\pause
\justifying
Sean $\vec{a},\vec{b}$ y $\vec{c}$ vectores en $\mathbb{R}^3$, y $\alpha$ un número real. Se verifican: \pause
\begin{enumerate}[<+->][{(1)}]
\justifying
\item $\vec{a} \times \vec{b}$ es perpendicular a $\vec{a}$ y $\vec{b}$.
\item $\vec{a} \times \vec{b}=-(\vec{b}\times \vec{a})$
\item Si $\vec{a}$ y $\vec{b}$ son paralelos entonces $\vec{a}\times\vec{b}=\vec{0}$
\item $\vec{a}\times (\alpha \vec{b})=(\alpha \vec{a})\times \vec{b}=\alpha ( \vec{a} \times \vec{b})$
\item $\vec{a}\times (\vec{b}+\vec{c})=(\vec{a}\times \vec{b})+(\vec{a}\times \vec{c})$
\item $||\vec{a} \times \vec{b}||$ es igual al área del parelelógramo formado por $\vec{a}$ y $\vec{b}$, y si $\theta$ es el ángulo formado por $\vec{a}$ y $\vec{b}$ entonces
$$||\vec{a} \times \vec{b}||=||\vec{a}||||\vec{b}||\sen(\theta).$$
\end{enumerate}
\vspace{-0.5cm}
\end{block}
\end{minipage}

\end{frame}

%---------------------------------------------------------------------
%\subsection{Propiedades}
\begin{frame}
  \frametitle{\insertsectionhead}
  \framesubtitle{\insertsubsectionhead}
  
\justifying
\footnotesize

\begin{minipage}[t]{0.5\linewidth}
\begin{exampleblock}{Ejemplo 2}
\justifying
Calcule el área del triángulo formado por los puntos $A(-1,2,3)$, $B(1,-2,0)$ y $C(2,4,5)$.
\end{exampleblock}\pause
\begin{block}{Nota}
\justifying
Si $\vec{a}$ y $\vec{b}$ son vectores del plano $XY$, entonces los podemos escribir como $\vec{a}=(a_1,a_2,0)$, $\vec{b}=(b_1,b_2,0)$ y \pause $$\vec{a}\times\vec{b}=0\hat{i}-0\hat{j}+\begin{vmatrix}a_1&a_2\\b_1&b_2\end{vmatrix}\hat{k}=(0,0,a_1b_2-a_2b_1)$$ \pause que corresponde a un vector paralelo al eje $Z$.
\end{block}
\end{minipage}\pause
\hspace{0.5cm}\begin{minipage}[t]{0.5\linewidth}
    \vspace{-0.5cm}
\begin{block}{Nota}
\justifying
$\vec{a}\times\vec{b}$ apunta en una dirección perpendicular a $\vec{a}$ y $\vec{b}$ que está dada por la \textbf{regla de la mano derecha}: si los dedos de su mano derecha se curvan en la dirección (en un ángulo menor que $\pi$) desde $\vec{a}$ hacia $\vec{b}$ entonces su dedo pulgar apunta en la dirección de $\vec{a}\times\vec{b}$.
\vspace{0.2cm}
\begin{center}
\visible<6>{\includegraphics[scale=0.5]{ProdCruz1.png}}
\end{center}

\end{block}

\end{minipage}



\end{frame}

% %==============================================
% \section{Conclusión}
% %==============================================

% \begin{frame}
%   \frametitle{\insertsectionhead}
%   \framesubtitle{\insertsubsectionhead}
  
%   \justifying
  
%   \begin{itemize}[<+->]
%   \justifying
%       \item Advierta que el el producto cruz de dos vectores es nuevamente un vector, pero que el producto punto de dos vectores es un número real.
%   \end{itemize}

% \end{frame}


% %==============================================
% \section{Asistencia}
% %==============================================

% \begin{frame}
%   \frametitle{\insertsectionhead}
%   \framesubtitle{\insertsubsectionhead}
  
%   \justifying
  
%   \textbf{Solo se considerará la asistencia hasta 5 minutos después de terminada la clase.}
  
%   \begin{center}
%   \includegraphics[scale=0.23]{QR}
%   \end{center}
  
% \end{frame}


\end{document}