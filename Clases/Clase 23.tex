\documentclass[aspectratio=169]{beamer}
\usepackage{graphicx}
\usepackage{ragged2e}
\usefonttheme{professionalfonts}
\usepackage[spanish]{babel}
\usepackage{advdate}
\usepackage{multicol}
\setbeamercovered{transparent}
\usepackage{array} % for "\newcolumntype" macro
\newcolumntype{C}{>{$\displaystyle}c<{$}}
\newcommand\myfrac[2]{\frac{#1}{#2\mathstrut}}

% Document metadata
\title{Clase 23}
\subtitle{Planos. Posiciones relativas}
\author{Profesor: Denis Osses}
%\date{\AdvanceDate[+1]\today}
%\date{\today}
\date{9 de junio de 2025}

% Image for the title page (use includegraphics option to properly size/place it)
\titlegraphic{\includegraphics[height=\paperheight]{imagen5}}

\usetheme[sectionstyle=style2]{trigon}

% Define logos to use (comment if no logo)
\biglogo{logoFIC} % Used on titlepage only
%\smalllogo{gggg}% Used on top right corner of regular frames

% ------ If you want to change the theme default colors, do it here ------
%\definecolor{tPrim}{HTML}{00843B}   % Green
%\definecolor{tSec}{HTML}{289B38}    % Green light
%\definecolor{tAccent}{HTML}{F07F3C} % Orange


% ------ Packages and definitions used for this demo. Can be removed ------
\usepackage{appendixnumberbeamer} % To use \appendix command
\pdfstringdefDisableCommands{% Fix hyperref translate warning with \appendix
\def\translate#1{#1}%
}
\usepackage{pgf-pie} % For pie charts
\usepackage{caption} % For subfigures
\usepackage{subcaption} % For subfigures
\usepackage{xspace}
\newcommand{\themename}{\textbf{\textsc{Álgebra}}\xspace}
\usepackage[scale=2]{ccicons} % Icons for CC-BY-SA
\usepackage{booktabs} % Better tables


%==============================================================================
%                               BEGIN DOCUMENT
%==============================================================================
\begin{document}

%--------------------------------------
% Create title frame
\titleframe

%--------------------------------------
% Table of contents
\begin{frame}{Temario}
  \setbeamertemplate{section in toc}[sections numbered]
  \tableofcontents%[hideallsubsections]
\end{frame}

%==============================================
\section{Objetivos de hoy}
%==============================================
%\subsection{Charts}
\begin{frame}{\insertsectionhead}
  \framesubtitle{\insertsubsectionhead}
  
\justifying

\begin{itemize}[<+->]
\justifying
    \item Reconocer los elementos de las diferentes representaciones de planos y determinar sus ecuaciones.
    \item Usar las distintas representaciones de planos y sus propiedades para resolver problemas geométricos.
\end{itemize}

\end{frame}

%==============================================
\section{Contenidos}
%==============================================

%---------------------------------------------------------------------
\subsection{Planos}
\begin{frame}
  \frametitle{\insertsectionhead}
  \framesubtitle{\insertsubsectionhead}
  
\justifying
\footnotesize

\begin{minipage}[t]{0.5\linewidth}
Observemos la siguiente figura:
\begin{center}
\includegraphics[scale=0.7]{Plano1.png}
\end{center}
\end{minipage}\pause 
\begin{minipage}[t]{0.5\linewidth}
\vspace{-1cm}
\begin{block}{Definición}
\justifying
Un \textbf{plano} en el espacio $\mathbb{R}^3$ es el conjunto determinado por \pause
\vspace{0.1cm}
\begin{itemize}[<+->]
\justifying
    \item un punto $P_0(x_0, y_0, z_0)$ en el plano
    \item y un vector $\vec{n}$ que es ortogonal al plano. Este vector ortogonal $\vec{n}$ se llama \textbf{vector normal}.
\end{itemize} \pause
\vspace{0.2cm}
Sea $P(x, y, z)$ un punto arbitrario en el plano, y sean $\vec{r}_0$ y $\vec{r}$ los vectores posición de $P_0$ y $P$, respectivamente. \pause Entonces $\overrightarrow{P_0P}=\vec{r}-\vec{r}_0$. \pause El vector normal $\vec{n}$ es ortogonal a todo vector en el plano dado. \pause En particular, $\vec{n}$ es ortogonal a $\vec{r}-\vec{r}_0$, por tanto, se tiene \pause  $$\color{red}\boxed{\color{black}\vec{n}\cdot(\vec{r}-\vec{r}_0)=0}.$$ \pause
Esta ecuación es conocida como \textbf{ecuación vectorial del plano}.

\end{block}
\end{minipage}

\end{frame}


%---------------------------------------------------------------------
%\subsection{Planos}
\begin{frame}
  \frametitle{\insertsectionhead}
  \framesubtitle{\insertsubsectionhead}
  
\justifying
\footnotesize

\begin{minipage}[t]{0.5\linewidth}
Si escribimos los vectores en coordenadas: $\vec{n}=(a,b,c)$, $\vec{r}=(x,y,z)$ y $\vec{r}_0=(x_0,y_0,z_0)$. \pause Reemplazando en la ecuación vectorial, obtenemos $$(a,b,c)\cdot(x-x_0,y-y_0,z-z_0)=0.$$ \pause Equivalentemente, $$\color{red}\boxed{\color{black}a(x-x_0)+b(y-y_0)+c(z-z_0)=0}$$ \pause
que es conocida como \textbf{ecuación general del plano} que contiene al punto $P_0(x_0,y_0,z_0)$ con vector normal $\vec{n}$. \pause
\begin{block}{Nota}
\justifying
Un plano que contiene al punto $P_0(x_0,y_0,z_0)$ es generado por 2 vectores no paralelos $\vec{u}$ y $\vec{v}$, cuya ecuación es $\vec{r}=\vec{r}_0+\mu\vec{u}+\lambda\vec{v}$, con $\mu,\lambda\in\mathbb{R}$.
\end{block}

\end{minipage}\pause 
\hspace{0.5cm}\begin{minipage}[t]{0.5\linewidth}
\vspace{-0.8cm}
\begin{exampleblock}{Ejemplo 1}
\justifying
Determine la ecuación del plano que pasa por los puntos $A(1,1,-1)$, $B(2,-1,3)$ y $C(3,1,1)$.
\end{exampleblock}\pause
\begin{block}{Nota}
\justifying
En el ejemplo anterior reescribimos la ecuación del plano como $$\color{red}\boxed{\color{black}ax+by+cz=d}$$\pause
Esta ecuación es conocida como \textbf{ecuación cartesiana del plano} y es la forma habitual de escribirla.\pause
\vspace{0.2cm}
Además, si 2 de los 3 parámetros $a,b,c$ son 0, entonces tenemos \textbf{planos paralelos a los planos coordenados} $XY$, $XZ$ e $YZ$. 
\end{block}
\end{minipage}

\end{frame}

%---------------------------------------------------------------------
\subsection{Posiciones relativas entre planos}
\begin{frame}
  \frametitle{\insertsectionhead}
  \framesubtitle{\insertsubsectionhead}
  
\justifying
\footnotesize

\begin{minipage}[t]{0.5\linewidth}
\vspace{-0.5cm}
\begin{block}{Definición}
\justifying
Dos planos son \textbf{paralelos} si sus vectores normales son paralelos. \pause Dos planos son \textbf{ortogonales o perpendiculares} si sus vectores normales son ortogonales o perpendiculares. 
\end{block}\pause
\begin{exampleblock}{Ejemplo 2}
\justifying
Encuentre la ecuación del plano $\pi$ sabiendo que:
\begin{enumerate}[{(1)}]
\justifying
\item $(0, 0, 0)\in\pi$,
\item $\pi$ es perpendicular al plano $\pi_1$, donde $\pi_1: x + y - z = 1$,
\item $\pi$ es paralelo a la recta $\ell$, donde $\ell : x = -1 -3, y = 1 + 2t, z = t$, con $t$ en $\mathbb{R}$.
\end{enumerate}
\end{exampleblock}
\end{minipage}\pause
\hspace{0.5cm}\begin{minipage}[t]{0.5\linewidth}
\begin{center}
   \visible<4-5>{\includegraphics[scale=0.75]{Plano2.png}}
\end{center}\pause
\begin{block}{Definición}
\justifying
El ángulo $\theta$ entre dos planos $\pi_1$ y $\pi_2$ corresponde al ángulo formado por sus vectores normales $\vec{n}_1$ y $\vec{n}_2$, respectivamente.
\end{block}
\end{minipage}

\end{frame}

%---------------------------------------------------------------------
%\subsection{Posiciones relativas entre planos}
\begin{frame}
  \frametitle{\insertsectionhead}
  \framesubtitle{\insertsubsectionhead}
  
\justifying
\footnotesize

\begin{minipage}[t]{0.5\linewidth}
\begin{exampleblock}{Ejemplo 3}
\justifying
Considere el punto $P(2,0,1)$ y la recta $$L:(x,y,z)=(1,2,3)+t(-2,-1,1)~~,~~t\in\mathbb{R}.$$
\vspace{-0.5cm}
\begin{enumerate}[{(1)}]
\justifying
\item Encuentre la ecuación del plano que contiene al punto $P$ y es perpendicular a $L$.
\item Encuentre la ecuación del plano que contiene a la recta $L$ y al punto $P$.
\end{enumerate}
\end{exampleblock}
\end{minipage}
\begin{minipage}[t]{0.5\linewidth}
\end{minipage}



\end{frame}

% %==============================================
% \section{Conclusión}
% %==============================================

% \begin{frame}
%   \frametitle{\insertsectionhead}
%   \framesubtitle{\insertsubsectionhead}
  
%   \justifying
  
%   \begin{itemize}[<+->]
%   \justifying
%       \item Advierta que la ecuación $ax+by=d$ representa un plano en el espacio, no una recta en el plano. Cada ecuación debe ser entendida en su contexto. 
%   \end{itemize}

% \end{frame}


% %==============================================
% \section{Asistencia}
% %==============================================

% \begin{frame}
%   \frametitle{\insertsectionhead}
%   \framesubtitle{\insertsubsectionhead}
  
%   \justifying
  
%   \textbf{Solo se considerará la asistencia hasta 5 minutos después de terminada la clase.}
  
%   \begin{center}
%   \includegraphics[scale=0.23]{QR}
%   \end{center}
  
% \end{frame}


\end{document}