\documentclass[aspectratio=169]{beamer}
\usepackage{graphicx}
\usepackage{ragged2e}
\usefonttheme{professionalfonts}
\usepackage[spanish]{babel}
\usepackage{advdate}
\usepackage{multicol}
\setbeamercovered{transparent}
\usepackage{array} % for "\newcolumntype" macro
\newcolumntype{C}{>{$\displaystyle}c<{$}}
\newcommand\myfrac[2]{\frac{#1}{#2\mathstrut}}

% Document metadata
\title{Clase 24}
\subtitle{Distancias}
\author{Profesor: Denis Osses}
%\date{\AdvanceDate[+1]\today}
%\date{\today}
\date{13 de junio de 2025}

% Image for the title page (use includegraphics option to properly size/place it)
\titlegraphic{\includegraphics[height=\paperheight]{imagen5}}

\usetheme[sectionstyle=style2]{trigon}

% Define logos to use (comment if no logo)
\biglogo{logoFIC} % Used on titlepage only
%\smalllogo{gggg}% Used on top right corner of regular frames

% ------ If you want to change the theme default colors, do it here ------
%\definecolor{tPrim}{HTML}{00843B}   % Green
%\definecolor{tSec}{HTML}{289B38}    % Green light
%\definecolor{tAccent}{HTML}{F07F3C} % Orange


% ------ Packages and definitions used for this demo. Can be removed ------
\usepackage{appendixnumberbeamer} % To use \appendix command
\pdfstringdefDisableCommands{% Fix hyperref translate warning with \appendix
\def\translate#1{#1}%
}
\usepackage{pgf-pie} % For pie charts
\usepackage{caption} % For subfigures
\usepackage{subcaption} % For subfigures
\usepackage{xspace}
\newcommand{\themename}{\textbf{\textsc{Álgebra}}\xspace}
\usepackage[scale=2]{ccicons} % Icons for CC-BY-SA
\usepackage{booktabs} % Better tables


%==============================================================================
%                               BEGIN DOCUMENT
%==============================================================================
\begin{document}

%--------------------------------------
% Create title frame
\titleframe

%--------------------------------------
% Table of contents
\begin{frame}{Temario}
  \setbeamertemplate{section in toc}[sections numbered]
  \tableofcontents%[hideallsubsections]
\end{frame}

%==============================================
\section{Objetivos de hoy}
%==============================================
%\subsection{Charts}
\begin{frame}{\insertsectionhead}
  \framesubtitle{\insertsubsectionhead}
  
\justifying

\begin{itemize}[<+->]
\justifying
    \item Calcular distancias entre puntos, rectas y planos.
\end{itemize}

\end{frame}

%==============================================
\section{Contenidos}
%==============================================

%---------------------------------------------------------------------
\subsection{Distancia Punto-Plano}
\begin{frame}
  \frametitle{\insertsectionhead}
  \framesubtitle{\insertsubsectionhead}
  
\justifying
\footnotesize

\begin{minipage}[t]{0.5\linewidth}
Queremos determinar la distancia $D$ (más corta) entre el punto $P_1(x_1,y_1,z_1)$ y el plano $\pi: ax+by+cz=d$ \pause 

\begin{center}
\visible<2-10>{\includegraphics[scale=0.7]{distancia1.png}}
\end{center} \pause

Para ello, elegimos cualquier punto $P_0(x_0,y_0,z_0)$ del plano y construimos el vector $\vec{b}=\overrightarrow{P_0P_1}$. \pause Luego lo proyectamos sobre el vector $\vec{n}=(a,b,c)$ normal al plano:
\end{minipage}\pause
\hspace{0.4cm}\begin{minipage}[t]{0.5\linewidth}
\vspace{-0.8cm}
\begin{center}
$D=||P_{\vec{n}}(\vec{b})||=\left|\left|\left(\frac{\vec{b}\cdot\vec{n}}{||\vec{n}||^2}\right)\vec{n}\right|\right|=\frac{|\vec{b}\cdot\vec{n}|}{||\vec{n}||^2}||\vec{n}||$\end{center}\pause Por tanto, 

\hspace{1cm}\begin{minipage}{0.4\textwidth}
\begin{center}
$\color{red}\boxed{\color{black}D=\frac{|\vec{b}\cdot\vec{n}|}{||\vec{n}||}}$
\end{center}
\end{minipage}\pause
\begin{minipage}{0.45\textwidth}
Esta expresión corresponde a la \textbf{distancia} $D$ \textbf{en forma vectorial}
\end{minipage}\pause 

\vspace{0.2cm}

Podemos expresar esta fórmula reemplazando $\vec{b}=(x_1-x_0,y_1-y_0,z_1-z_0)$, $\vec{n}=(a,b,c)$ y simplificando: \pause $$\color{red}\boxed{\color{black}D=\frac{|ax_1+by_1+cz_1-d|}{\sqrt{a^2+b^2+c^2}}}$$\pause

\begin{exampleblock}{Ejemplo 1}
\justifying
Determine la distancia del punto $B(1,-2,3)$ al plano $2x-y-z=2$.
\end{exampleblock}
\end{minipage}

\end{frame}

%---------------------------------------------------------------------
\subsection{Distancia Punto-Recta}
\begin{frame}
  \frametitle{\insertsectionhead}
  \framesubtitle{\insertsubsectionhead}
  
\justifying
\footnotesize

\begin{minipage}[t]{0.5\linewidth}
Queremos determinar la distancia $D$ (más corta) entre el punto $P_1(x_1,y_1,z_1)$ y la recta $L: \vec{r}=\vec{p}_0+\lambda\vec{v}$, $\lambda\in\mathbb{R}$.

\begin{figure} [h]
\centering
\visible<2-7>{\includegraphics[scale=0.28]{distancia2.png}}
\end{figure}
\end{minipage}\pause
\hspace{0.5cm}\begin{minipage}[t]{0.5\linewidth}
Para ello elegimos cualquier punto $P_0(x_0,y_0,z_0)$ de la recta y construimos el vector $\vec{b}=\overrightarrow{P_0P_1}$. \pause Luego lo proyectamos sobre el vector director $\vec{v}$ de la recta $L$, obteniendo $P_{\vec{v}}(\vec{b})$. \pause Notamos que $$P_{\vec{v}}(\vec{b})+\vec{y}=\vec{b}~~\Rightarrow~~\vec{y}=\vec{b}-P_{\vec{v}}(\vec{b}).$$ \pause Como $$D=||\vec{y}||$$\pause concluimos que $$\color{red}\boxed{\color{black}D=\left|\left|\vec{b}-P_{\vec{v}}(\vec{b})\right|\right|}$$\pause 
Note que $D$ corresponde a la norma (longitud) del vector complemento ortogonal de $\vec{b}$ sobre $\vec{v}$.
\end{minipage}

\end{frame}

%---------------------------------------------------------------------
%\subsection{Distancia Punto-Recta}
\begin{frame}
  \frametitle{\insertsectionhead}
  \framesubtitle{\insertsubsectionhead}
  
\justifying
\footnotesize

\begin{minipage}[t]{0.5\linewidth}
Otra forma de determinar la distancia $D$ desde el punto $P_1$ a la recta $L$ es mediante el siguiente proceso: \pause Tenemos que $$P_{\vec{v}}(\vec{b})+\vec{y}=\vec{b}$$ \pause Hacemos producto cruz $\times$ de esta igualdad con $\vec{v}$. \pause Obtenemos $$P_{\vec{v}}(\vec{b})\times\vec{v}+\vec{y}\times\vec{v}=\vec{b}\times\vec{v}.$$\pause Como $\vec{v}$ y $P_{\vec{v}}(\vec{b})$ son paralelos entonces $P_{\vec{v}}(\vec{b})\times\vec{v}=\vec{0}$. \pause Así $\vec{y}\times\vec{v}=\vec{b}\times\vec{v}.$ \pause Ahora tomamos la norma de estos vectores \pause $$||\vec{y}\times\vec{v}||=||\vec{b}\times\vec{v}||~~\Rightarrow~~||\vec{y}||\cdot||\vec{v}||\sen\left(\frac{\pi}{2}\right)=||\vec{b}\times\vec{v}||.$$
Por lo tanto
\end{minipage}\pause
\hspace{0.5cm}\begin{minipage}[t]{0.5\linewidth}
\vspace{-0.8cm} $$\color{red}\boxed{\color{black}D=\frac{||\vec{b}\times\vec{v}||}{||\vec{v}||}}$$ \pause
\vspace{-0.2cm}
\begin{exampleblock}{Ejemplo 2}
\justifying
Determine la distancia desde el punto $A(1,2,3)$ a la recta \begin{eqnarray*}x&=&2t\\ y&=&-1-t~~,~~t\in\mathbb{R} \\ z&=&3\end{eqnarray*}
\end{exampleblock}\pause

\begin{exampleblock}{\color{red}{Ejercicio Propuesto}}
\justifying
Determine una fórmula que permita calcular la distancia más corta entre las rectas de ecuaciones $\ell_1: \vec{r}_1=\vec{p}_1+t_1\vec{v}_1$, $t_1\in\mathbb{R}$ y $\ell_2: \vec{r}_2=\vec{p}_2+t_2\vec{v}_2$, $t_2\in\mathbb{R}$.
\end{exampleblock}
\end{minipage}



\end{frame}

% %==============================================
% \section{Conclusión}
% %==============================================

% \begin{frame}
%   \frametitle{\insertsectionhead}
%   \framesubtitle{\insertsubsectionhead}
  
%   \justifying
  
%   \begin{itemize}[<+->]
%   \justifying
%       \item Advierta que existen varias formas de calcular las distancias.
%   \end{itemize}

% \end{frame}


% %==============================================
% \section{Asistencia}
% %==============================================

% \begin{frame}
%   \frametitle{\insertsectionhead}
%   \framesubtitle{\insertsubsectionhead}
  
%   \justifying
  
%   \textbf{Solo se considerará la asistencia hasta 5 minutos después de terminada la clase.}
  
%   \begin{center}
%   \includegraphics[scale=0.23]{QR}
%   \end{center}
  
% \end{frame}


\end{document}