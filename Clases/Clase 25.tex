\documentclass[handout,aspectratio=169]{beamer}
\usepackage{graphicx}
\usepackage{ragged2e}
\usefonttheme{professionalfonts}
\usepackage[spanish]{babel}
\usepackage{advdate}
\usepackage{multicol}
\setbeamercovered{transparent}
\usepackage{array} % for "\newcolumntype" macro
\newcolumntype{C}{>{$\displaystyle}c<{$}}
\newcommand\myfrac[2]{\frac{#1}{#2\mathstrut}}
\usepackage{caption}

% Document metadata
\title{Clase 25}
\subtitle{Números complejos. Módulo, conjugado y forma polar.}
\author{Profesor: Denis Osses}
%\date{\AdvanceDate[+1]\today}
%\date{\today}
\date{16 de junio de 2025}

% Image for the title page (use includegraphics option to properly size/place it)
\titlegraphic{\includegraphics[height=\paperheight]{imagen5}}

\usetheme[sectionstyle=style2]{trigon}

% Define logos to use (comment if no logo)
\biglogo{logoFIC} % Used on titlepage only
%\smalllogo{gggg}% Used on top right corner of regular frames

% ------ If you want to change the theme default colors, do it here ------
%\definecolor{tPrim}{HTML}{00843B}   % Green
%\definecolor{tSec}{HTML}{289B38}    % Green light
%\definecolor{tAccent}{HTML}{F07F3C} % Orange


% ------ Packages and definitions used for this demo. Can be removed ------
\usepackage{appendixnumberbeamer} % To use \appendix command
\pdfstringdefDisableCommands{% Fix hyperref translate warning with \appendix
\def\translate#1{#1}%
}
\usepackage{pgf-pie} % For pie charts
\usepackage{caption} % For subfigures
\usepackage{subcaption} % For subfigures
\usepackage{xspace}
\newcommand{\themename}{\textbf{\textsc{Álgebra}}\xspace}
\usepackage[scale=2]{ccicons} % Icons for CC-BY-SA
\usepackage{booktabs} % Better tables


%==============================================================================
%                               BEGIN DOCUMENT
%==============================================================================
\begin{document}

%--------------------------------------
% Create title frame
\titleframe

%--------------------------------------
% Table of contents
\begin{frame}{Temario}
  \setbeamertemplate{section in toc}[sections numbered]
  \tableofcontents%[hideallsubsections]
\end{frame}

%==============================================
\section{Objetivos de hoy}
%==============================================
%\subsection{Charts}
\begin{frame}{\insertsectionhead}
  \framesubtitle{\insertsubsectionhead}
  
\justifying

\begin{itemize}[<+->]
\justifying
    \item Utilizar las distintas representaciones de los números complejos y sus operaciones.
\end{itemize}

\end{frame}

%==============================================
\section{Contenidos}
%==============================================

%---------------------------------------------------------------------
\subsection{Números complejos}
\begin{frame}
  \frametitle{\insertsectionhead}
  \framesubtitle{\insertsubsectionhead}
  
\justifying
\footnotesize

\begin{minipage}[t]{0.5\linewidth}
\vspace{1cm}
Algunas ecuaciones cuadráticas no tienen
solución real. Por ejemplo $x^2+1=0$ no tiene raíces reales porque no existe $x\in\mathbb{R}$ tal que $x^2=-1$.\pause

\vspace{0.2cm}

Para resolver todas las ecuaciones cuadráticas, se crea un sistema numérico expandido llamado conjunto de \textbf{números complejos} $\mathbb{C}$, en el cual se define el número \textbf{imaginario} $i$ con la propiedad que $$i^2=-1~~\Leftrightarrow~~i=\sqrt{-1}.$$
\end{minipage}\pause
\hspace{0.5cm}\begin{minipage}[t]{0.5\linewidth}
\vspace{-0.5cm}
\begin{block}{Definición}
\justifying
Un \textbf{número complejo} es una expresión de la forma $$z=a+bi$$\pause
donde $a$ y $b$ son números reales y $i^2=-1$. La \textbf{parte real} de este número complejo es $a$ y la \textbf{parte imaginaria} es $b$. \pause 
Dos números complejos son iguales si y solo si sus partes reales son iguales y sus partes imaginarias son iguales. \pause

\vspace{0.2cm}
El número complejo $a+bi$ también puede estar representado por el par ordenado $(a,b)$ y determinado como un punto en un plano (llamado plano de Argand) como en la figura siguiente. \pause De este modo, el número complejo $i=0+1\cdot i$ se identifica con el punto $(0,1)$.
\end{block}
\end{minipage}

\end{frame}

%---------------------------------------------------------------------
\subsection{Álgebra y operaciones de números complejos}
\begin{frame}
  \frametitle{\insertsectionhead}
  \framesubtitle{\insertsubsectionhead}
  
\justifying
\footnotesize

\begin{minipage}[t]{0.5\linewidth}
\begin{center}
    \includegraphics[scale=0.6]{Complejo1}
\end{center}\pause

\begin{block}{Definición}
\justifying
La parte real del complejo $z=a+bi$ se anota como $Re(z)=a$ y la parte imaginaria como $Im(z)=b$. \pause Note que $Re(z), Im(z)\in\mathbb{R}$. \pause A partir de esto, el eje horizontal del plano de Argand se denomina \textbf{eje real} y el eje vertical se llama \textbf{eje imaginario}.
\end{block}
\end{minipage}\pause
\hspace{0.5cm}\begin{minipage}[t]{0.5\linewidth}
\vspace{-3.5cm}
\begin{block}{Definición}
\justifying
Si $z=a+bi$ y $w=c+di$ entonces \pause

\begin{enumerate}[<+->][{(1)}]
\justifying
    \item su \textbf{adición} está dada por: $z+w=(a+c)+(b+d)i$.
    \item su \textbf{resta} está dada por: $z-w=(a-c)+(b-d)i$.
    \item su \textbf{producto} está dado por: $zw=(ac-bd)+(bc+ad)i$.
    \item su \textbf{división} está dada por: $\dfrac{z}{w}=\dfrac{(ac+bd)+(bc-ad)i}{c^2+d^2}$.
    \item el \textbf{módulo} de $z$ está dado por: $|z|=\sqrt{a^2+b^2}$.
    \item el \textbf{conjugado} de $z$ está dado por: $\overline{z}=a-bi$.
\end{enumerate}
\end{block}
\end{minipage}

\end{frame}

%---------------------------------------------------------------------
%\subsection{Álgebra y operaciones de números complejos}
\begin{frame}
  \frametitle{\insertsectionhead}
  \framesubtitle{\insertsubsectionhead}
  
\justifying
\footnotesize

\begin{minipage}[t]{0.5\linewidth}
\hspace{0.7cm}\begin{minipage}{0.4\textwidth}
\includegraphics[scale=0.4]{Complejo2}
% \captionof{figure}{Conjugación}
\end{minipage}
\hspace{0.3cm}\begin{minipage}{0.3\textwidth}
\includegraphics[scale=0.35]{Complejo3}
% \captionof{figure}{Módulo}
\end{minipage}\pause

\begin{block}{Propiedades}
\justifying
Si $z=a+bi$ y $w=c+di$ entonces \pause
\begin{multicols}{2}
\begin{enumerate}[<+->][{(a)}]
\justifying
    \item $z\overline{z}=|z|^2$
    \item $\dfrac{z}{w}=\dfrac{z\overline{w}}{|w|^2}$
    \item $\overline{z+w}=\overline{z}+\overline{w}$
    \item $\overline{zw}=\overline{z}\;\overline{w}$
    \item $z+\overline{z}=2Re(z)$
    \item $z-\overline{z}=2iIm(z)$
\end{enumerate}
\end{multicols}
\end{block}
\end{minipage}\pause
\hspace{0.5cm}\begin{minipage}[t]{0.5\linewidth}
\vspace{-1.5cm}
\begin{exampleblock}{Ejemplo 1}
\justifying
Dados los números complejos $z_1=\frac{1}{4}+\frac{2}{3}i$ y $z_1=\frac{1}{8}+\frac{1}{6}i$, calcule
\begin{multicols}{2}
\begin{enumerate}[{(1)}]
\item $z_1+z_2$
\item $|z_1-z_2|$
\item $z_1:z_2$
\item $\bar{z_1}\cdot z_2$
\end{enumerate}
\end{multicols}
\end{exampleblock}\pause 

\begin{exampleblock}{Ejemplo 2}
\justifying
Determine la parte real e imaginaria de los siguientes números complejos
\begin{multicols}{3}
\begin{enumerate}
%\item $\dfrac{1}{2+i}$
\item $\dfrac{3+2i}{3-2i}$
\item $(1+i)^4$
\item $\dfrac{8}{(1-i)^5}$
\end{enumerate}
\end{multicols}
\end{exampleblock}
\end{minipage}

\end{frame}

%---------------------------------------------------------------------
\subsection{Forma polar}
\begin{frame}
  \frametitle{\insertsectionhead}
  \framesubtitle{\insertsubsectionhead}
  
\justifying
\footnotesize

\begin{minipage}[t]{0.5\linewidth}
\begin{center}
    \includegraphics[scale=0.6]{Complejo4}
\end{center}\pause 
\begin{block}{Definición}
\justifying
Si $z=a+bi$, entonces su \textbf{forma polar} es $$z=r(\cos(\theta)+i\sen(\theta))$$ \pause donde $r=|z|=\sqrt{a^2+b^2}$ y $\tan(\theta)=\frac{b}{a}$. \pause El ángulo $\theta$ se llama \textbf{argumento} de $z$ y escribimos $\theta=\arg(z)$. \pause Note que $\arg(z)$ no es único.
\end{block}
\end{minipage}
\hspace{0.5cm}\begin{minipage}[t]{0.5\linewidth}
\vspace{-1cm}
\begin{exampleblock}{Ejemplo 3}
\justifying
Calcule el m\'odulo y el argumento de los n\'umeros $\dfrac{1+i}{1-i}$ y $\dfrac{\sqrt{2}}{1-i}$.
\end{exampleblock}\pause
\begin{exampleblock}{\color{red}Ejercicio Propuesto}
\justifying
\vspace{0.2cm}
Escriba en forma polar el n\'umero complejo $\dfrac{i^5-i^{-4}}{-\sqrt{2}i}$
\end{exampleblock}

\end{minipage}

\end{frame}

% %==============================================
% \section{Conclusión}
% %==============================================

% \begin{frame}
%   \frametitle{\insertsectionhead}
%   \framesubtitle{\insertsubsectionhead}
  
%   \justifying
  
%   \begin{itemize}[<+->]
%   \justifying
%       \item Los números complejos son una extensión de los números reales, pero no poseen orden: no podemos decir si un complejo es positivo o negativo. 
%   \end{itemize}

% \end{frame}


% %==============================================
% \section{Asistencia}
% %==============================================

% \begin{frame}
%   \frametitle{\insertsectionhead}
%   \framesubtitle{\insertsubsectionhead}
  
%   \justifying
  
%   \textbf{Solo se considerará la asistencia hasta 5 minutos después de terminada la clase.}
  
%   \begin{center}
%   \includegraphics[scale=0.23]{QR}
%   \end{center}
  
% \end{frame}


\end{document}