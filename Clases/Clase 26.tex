\documentclass[aspectratio=169]{beamer}
\usepackage{graphicx}
\usepackage{ragged2e}
\usefonttheme{professionalfonts}
\usepackage[spanish]{babel}
\usepackage{advdate}
\usepackage{multicol}
\setbeamercovered{transparent}
\usepackage{array} % for "\newcolumntype" macro
\newcolumntype{C}{>{$\displaystyle}c<{$}}
\newcommand\myfrac[2]{\frac{#1}{#2\mathstrut}}
\usepackage{caption}

% Document metadata
\title{Clase 26}
\subtitle{Forma Polar de un complejo. Potencias y Raíces $n$-ésimas}
\author{Profesor: Denis Osses}
%\date{\AdvanceDate[+1]\today}
\date{19 de junio de 2025}

% Image for the title page (use includegraphics option to properly size/place it)
\titlegraphic{\includegraphics[height=\paperheight]{imagen5}}

\usetheme[sectionstyle=style2]{trigon}

% Define logos to use (comment if no logo)
\biglogo{logoFIC} % Used on titlepage only
%\smalllogo{gggg}% Used on top right corner of regular frames

% ------ If you want to change the theme default colors, do it here ------
%\definecolor{tPrim}{HTML}{00843B}   % Green
%\definecolor{tSec}{HTML}{289B38}    % Green light
%\definecolor{tAccent}{HTML}{F07F3C} % Orange


% ------ Packages and definitions used for this demo. Can be removed ------
\usepackage{appendixnumberbeamer} % To use \appendix command
\pdfstringdefDisableCommands{% Fix hyperref translate warning with \appendix
\def\translate#1{#1}%
}
\usepackage{pgf-pie} % For pie charts
\usepackage{caption} % For subfigures
\usepackage{subcaption} % For subfigures
\usepackage{xspace}
\newcommand{\themename}{\textbf{\textsc{Álgebra}}\xspace}
\usepackage[scale=2]{ccicons} % Icons for CC-BY-SA
\usepackage{booktabs} % Better tables


%==============================================================================
%                               BEGIN DOCUMENT
%==============================================================================
\begin{document}

%--------------------------------------
% Create title frame
\titleframe

%--------------------------------------
% Table of contents
\begin{frame}{Temario}
  \setbeamertemplate{section in toc}[sections numbered]
  \tableofcontents%[hideallsubsections]
\end{frame}

%==============================================
\section{Objetivos de hoy}
%==============================================
%\subsection{Charts}
\begin{frame}{\insertsectionhead}
  \framesubtitle{\insertsubsectionhead}
  
\justifying

\begin{itemize}[<+->]
\justifying
\item Representar un complejo en forma polar.
    \item Interpretar geométrica y algebraicamente las raíces $n$-ésimas de números complejos.
\end{itemize}

\end{frame}

%==============================================
\section{Contenidos}
%==============================================

%---------------------------------------------------------------------
\subsection{Forma polar}
\begin{frame}
  \frametitle{\insertsectionhead}
  \framesubtitle{\insertsubsectionhead}
  
\justifying
\footnotesize

\begin{minipage}[t]{0.5\linewidth}
\begin{center}
    \includegraphics[scale=0.6]{Complejo4}
\end{center}\pause 
\begin{block}{Definición}
\justifying
Si $z=a+bi$, entonces su \textbf{forma polar} es $$z=r(\cos(\theta)+i\sen(\theta))=r\textrm{cis}(\theta)$$ \pause donde $r=|z|=\sqrt{a^2+b^2}$ y $\tan(\theta)=\frac{b}{a}$. \pause El ángulo $\theta$ se llama \textbf{argumento} de $z$ y escribimos $\theta=\arg(z)$. \pause Note que $\arg(z)$ no es único.
\end{block}
\end{minipage}
\hspace{0.5cm}\begin{minipage}[t]{0.5\linewidth}
\vspace{-1cm}
\begin{exampleblock}{Ejemplo 1}
\justifying
Calcule el m\'odulo y el argumento de $\dfrac{1+i}{1-i}$ 
% y $\dfrac{\sqrt{2}}{1-i}$.
\end{exampleblock}
% \begin{exampleblock}{\color{red}Ejercicio Propuesto}
% \justifying
% \vspace{0.2cm}
% Escriba en forma polar el n\'umero complejo $\dfrac{i^5-i^{-4}}{-\sqrt{2}i}$
% \end{exampleblock}

\end{minipage}

\end{frame}


%---------------------------------------------------------------------
\subsection{Multiplicación en forma polar y Teorema de De Moivre}
\begin{frame}
  \frametitle{\insertsectionhead}
  \framesubtitle{\insertsubsectionhead}
  
\justifying
\footnotesize

\begin{minipage}[t]{0.5\linewidth}
\begin{center}
    \includegraphics[scale=0.5]{Complejo5}
\end{center}\pause
La forma polar de números complejos da idea de la multiplicación y la división. \pause Si $z_1=r_1(\cos(\theta_1)+i\sen(\theta_1))$ y $z_2=r_2(\cos(\theta_2)+i\sen(\theta_2))$ entonces \pause $$\color{red}\boxed{\color{black}z_1z_2=r_1r_2(\cos(\theta_1+\theta_2)+i\sen(\theta_1+\theta_2))}$$
\end{minipage}\pause
\hspace{0.5cm}\begin{minipage}[t]{0.5\linewidth}
\vspace{-3cm}
\begin{block}{Teorema (De Moivre)}\pause
\justifying
Si $z=r(\cos(\theta)+i\sen(\theta))$ y $n\in\mathbb{N}$ entonces  $$z^n=r^n(\cos(n\theta)+i\sen(n\theta)).$$
\end{block} \pause
\begin{exampleblock}{Ejemplo 2}
\justifying
Calcule $(1+i)^{100}$.
\end{exampleblock}
\end{minipage}

\end{frame}

%---------------------------------------------------------------------
\subsection{Raíces $n$-ésimas de un complejo}
\begin{frame}
  \frametitle{\insertsectionhead}
  \framesubtitle{\insertsubsectionhead}
  
\justifying
\footnotesize

\begin{minipage}[t]{0.5\linewidth}
El teorema de De Moivre también se puede usar para hallar las raíces $n$-ésimas de
números complejos. \pause 

\begin{block}{Definición}
\justifying
Una raíz $n$-ésima del número complejo $z$ es un número complejo $w$ tal que $w^n=z$.
\end{block} \pause

\begin{block}{Teorema}
\justifying
Sea $z=r(\cos(\theta)+i\sen(\theta))$ y $n\in\mathbb{N}$. Entonces $z$ tiene las $n$ raíces distintas $$w_k=r^{1/n}\left[\cos\left(\frac{\theta+2k\pi}{n}\right)+i\sen\left(\frac{\theta+2k\pi}{n}\right)\right]$$ para $k=0,1,2,\ldots,n-1.$
\end{block} 
\end{minipage}\pause
\hspace{0.5cm}\begin{minipage}[t]{0.5\linewidth}
\vspace{-0.5cm}
\begin{block}{Nota}
\justifying
Las raíces $n$-ésimas de $z$ tienen módulo $|w_k|=r^{1/n}$, es decir, $w_k$ se ubican en el círculo de radio $r^{1/n}$ del plano complejo. \pause También,
como el argumento de cada raíz sucesiva $w_k$ excede al argumento de la raíz previa en
$2\pi/n$, estas se encuentran igualmente espaciadas en este círculo. \pause Ver \href{https://www.geogebra.org/m/bterebwm}{\textbf{Geogebra}}
\end{block} \pause

\begin{exampleblock}{Ejemplo 3}
\justifying
Calcule las raíces sextas de $-8$.
\end{exampleblock} \pause

\begin{exampleblock}{Ejemplo 4}
\justifying
Determine las soluciones complejas de $z^4-i=0$
\end{exampleblock}
\end{minipage}

\end{frame}

% %==============================================
% \section{Conclusión}
% %==============================================

% \begin{frame}
%   \frametitle{\insertsectionhead}
%   \framesubtitle{\insertsubsectionhead}
  
%   \justifying
  
%   \begin{itemize}[<+->]
%   \justifying
%       \item Desde el punto de vista polar, la multiplicación de complejos se puede interpretrar como rotaciones.
%   \end{itemize}

% \end{frame}


% %==============================================
% \section{Asistencia}
% %==============================================

% \begin{frame}
%   \frametitle{\insertsectionhead}
%   \framesubtitle{\insertsubsectionhead}
  
%   \justifying
  
%   \textbf{Solo se considerará la asistencia hasta 5 minutos después de terminada la clase.}
  
%   \begin{center}
%   \includegraphics[scale=0.23]{QR}
%   \end{center}
  
% \end{frame}


\end{document}