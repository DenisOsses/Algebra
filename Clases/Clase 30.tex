\documentclass[aspectratio=169]{beamer}
\usepackage{graphicx}
\usepackage{ragged2e}
\usefonttheme{professionalfonts}
\usepackage[spanish]{babel}
\usepackage{advdate}
\usepackage{multicol}
\setbeamercovered{transparent}
\usepackage{array} % for "\newcolumntype" macro
\newcolumntype{C}{>{$\displaystyle}c<{$}}
\newcommand\myfrac[2]{\frac{#1}{#2\mathstrut}}
\usepackage{caption}

% Document metadata
\title{Clase 27}
\subtitle{Polinomios. Algoritmo de la División y Teorema del Resto}
\author{Profesor: Denis Osses}
%\date{\AdvanceDate[+1]\today}
%\date{\today}
\date{25 de junio de 2025}

% Image for the title page (use includegraphics option to properly size/place it)
\titlegraphic{\includegraphics[height=\paperheight]{imagen5}}

\usetheme[sectionstyle=style2]{trigon}

% Define logos to use (comment if no logo)
\biglogo{logoFIC} % Used on titlepage only
%\smalllogo{gggg}% Used on top right corner of regular frames

% ------ If you want to change the theme default colors, do it here ------
%\definecolor{tPrim}{HTML}{00843B}   % Green
%\definecolor{tSec}{HTML}{289B38}    % Green light
%\definecolor{tAccent}{HTML}{F07F3C} % Orange


% ------ Packages and definitions used for this demo. Can be removed ------
\usepackage{appendixnumberbeamer} % To use \appendix command
\pdfstringdefDisableCommands{% Fix hyperref translate warning with \appendix
\def\translate#1{#1}%
}
\usepackage{pgf-pie} % For pie charts
\usepackage{caption} % For subfigures
\usepackage{subcaption} % For subfigures
\usepackage{xspace}
\newcommand{\themename}{\textbf{\textsc{Álgebra}}\xspace}
\usepackage[scale=2]{ccicons} % Icons for CC-BY-SA
\usepackage{booktabs} % Better tables


%==============================================================================
%                               BEGIN DOCUMENT
%==============================================================================
\begin{document}

%--------------------------------------
% Create title frame
\titleframe

%--------------------------------------
% Table of contents
\begin{frame}{Temario}
  \setbeamertemplate{section in toc}[sections numbered]
  \tableofcontents%[hideallsubsections]
\end{frame}

%==============================================
\section{Objetivos de hoy}
%==============================================
%\subsection{Charts}
\begin{frame}{\insertsectionhead}
  \framesubtitle{\insertsubsectionhead}
  
\justifying

\begin{itemize}[<+->]
\justifying
    \item Definir polinomios junto a sus elementos y operaciones principales.
\end{itemize}

\end{frame}

%==============================================
\section{Contenidos}
%==============================================

%---------------------------------------------------------------------
\subsection{Polinomios}
\begin{frame}
  \frametitle{\insertsectionhead}
  \framesubtitle{\insertsubsectionhead}
  
\justifying
\footnotesize

\begin{minipage}[t]{0.5\linewidth}
\begin{block}{Definición}
\justifying
Un polinomio $f$ es una expresión (función) de la forma $$f(x)=a_nx^n+a_{n-1}x^{n-1}+\cdots a_2x^2+a_1x+a_0~~,~~x\in\mathbb{R}$$ donde los $a_i\in\mathbb{R}$ se denominan \textbf{coeficientes} del polinomio y $n\in\mathbb{N}$ es su \textbf{grado} u \textbf{orden}.
\end{block}\pause

% \begin{exampleblock}{Ejemplo}
% \justifying
% Las funciones constante, lineal, cuadrática y cúbica son polinomios.
% \end{exampleblock}

\begin{block}{Definición}
\justifying
Un polinomio $f$ es \textbf{divisible} por el polinomio $p$ si existe otro polinomio $q$ tal que $$f(x)=p(x)q(x).$$
\end{block}
\end{minipage}\pause 
\hspace{0.5cm}\begin{minipage}[t]{0.5\linewidth}
\begin{exampleblock}{Ejemplo}
\justifying
$x^4-16$ es divisible por $x^2-4$, $x^2+4$, $(x-2)$ y $x+2$, pero no es divisible por $x^2+3x+1.$
\end{exampleblock}\pause 

\begin{block}{División de polinomios}
\justifying
Si $f(x)$ y $p(x)$ son polinomios y si $p(x)\neq0$, entonces existen polinomios únicos $q(x)$ y $r(x)$ tales que $$f(x)=p(x)q(x)+r(x)$$
donde ya sea $r(x)=0$ o el grado de $r(x)$ es menor que el grado de $p(x)$. El polinomio $q(x)$ es el \textbf{cociente} y $r(x)$ es el \textbf{residuo} en la división de $f(x)$ entre $p(x)$.
\end{block}
\end{minipage}


\end{frame}

%---------------------------------------------------------------------
\subsection{Teorema del Resto}
\begin{frame}
  \frametitle{\insertsectionhead}
  \framesubtitle{\insertsubsectionhead}
  
\justifying
\footnotesize

\begin{minipage}[t]{0.5\linewidth}
\begin{exampleblock}{Ejemplo}
\justifying
Realice la división de $x^4-16$ entre $x^2+3x+1.$
\end{exampleblock}\pause 
\begin{block}{Teorema del Resto}
\justifying
Si un polinomio $f(x)$ se divide entre $x-c$, entonces el residuo (resto) es $f(c)$
\end{block}\pause 

\begin{block}{Teorema del Factor}
\justifying
Un polinomio $f(x)$ tiene un factor $x-c$ si y sólo si $f(c)=0$.
\end{block}
\end{minipage}\pause 
\hspace{0.5cm}\begin{minipage}[t]{0.5\linewidth}
\begin{exampleblock}{Ejemplo}
\justifying
Demuestre que $x-2$ es un factor del polinomio $~~~~f(x)=x^3-4x^2+3x+2$.
\end{exampleblock}\pause 

Para aplicar el teorema del resto es necesario dividir un polinomio $f(x)$ entre $x-c$. El método de \textbf{división sintética} se puede usar para simplificar este
trabajo. \pause 

\begin{exampleblock}{Ejemplo}
\justifying
Use división sintética para hallar el cociente $q(x)$ y residuo $r$ si el polinomio $2x^4+5x^3-2x-8$ se divide entre $x+3$.
\end{exampleblock}

\end{minipage}

\end{frame}

%---------------------------------------------------------------------
\subsection{División Sintética}
\begin{frame}
  \frametitle{\insertsectionhead}
  \framesubtitle{\insertsubsectionhead}
  
\justifying
\footnotesize

\begin{minipage}[t]{0.5\linewidth}
\begin{center}
    \includegraphics[scale=0.53]{sintetica}
\end{center}
\end{minipage}\pause 
\begin{minipage}[t]{0.5\linewidth}
\vspace{-5cm}
Los teoremas del factor y el resto se pueden extender al conjunto de números complejos. Así, un número complejo $c=a+bi$ es un cero de un polinomio $f(x)$ si y sólo si $x-c$ es un factor de $f(x)$. Excepto en casos especiales,
los ceros de polinomios son muy difíciles de hallar. El siguiente teorema expresa que hay al menos un cero $c$ y, en consecuencia, por el teorema del factor, $f(x)$ tiene un factor de la forma $x-c$.
\end{minipage}

\end{frame}

%---------------------------------------------------------------------
\subsection{Teorema Fundamental del Álgebra}
\begin{frame}
  \frametitle{\insertsectionhead}
  \framesubtitle{\insertsubsectionhead}
  
\justifying
\footnotesize

\begin{minipage}[t]{0.5\linewidth}
\begin{block}{Teorema Fundamental del Álgebra}
\justifying
    Si un polinomio $f(x)$ tiene un grado positivo y coeficientes complejos, entonces $f(x)$ tiene al menos un cero complejo.
\end{block}\pause 
\begin{block}{Factorización
completa para polinomios}
\justifying
Si $f(x)$ es un polinomio de grado $n>0$, entonces existen $n$ números complejos $c_1,c_2,\ldots,c_n$ tales que $$f(x)=a(x-c_1)(x-c_2)\cdots(x-c_n)$$
donde $a$ es el coeficiente principal de $f(x)$. Cada número $c_k$ es un cero de $f(x).$
\end{block}
\end{minipage}\pause 
\hspace{0.5cm}\begin{minipage}[t]{0.5\linewidth}
\begin{exampleblock}{Ejemplo}
    \justifying Factorice completamente el polinomio $$p(x)=5x^3-30x^2+65x$$
\end{exampleblock}\pause

\begin{block}{Teorema}
    \justifying
    Todo polinomio con coeficientes reales y $n$ de grado positivo se pueden expresar como un producto de polinomios lineales y cuadráticos con coeficientes reales tales que los factores cuadráticos son irreducibles sobre $\mathbb{R}$.
\end{block}

\end{minipage}

\end{frame}


% %==============================================
% \section{Conclusión}
% %==============================================

% \begin{frame}
%   \frametitle{\insertsectionhead}
%   \framesubtitle{\insertsubsectionhead}
  
%   \justifying
  
%   \begin{itemize}[<+->]
%   \justifying
%       \item Desde el punto de vista polar, la multiplicación de complejos se puede interpretrar como rotaciones.
%   \end{itemize}

% \end{frame}


% %==============================================
% \section{Asistencia}
% %==============================================

% \begin{frame}
%   \frametitle{\insertsectionhead}
%   \framesubtitle{\insertsubsectionhead}
  
%   \justifying
  
%   \textbf{Solo se considerará la asistencia hasta 5 minutos después de terminada la clase.}
  
%   \begin{center}
%   \includegraphics[scale=0.23]{QR}
%   \end{center}
  
% \end{frame}


\end{document}